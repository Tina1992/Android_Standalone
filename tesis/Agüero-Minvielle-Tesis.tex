%% LyX 2.1.4 created this file.  For more info, see http://www.lyx.org/.
%% Do not edit unless you really know what you are doing.
\documentclass[oneside,american,spanish,english]{book}
\usepackage[T1]{fontenc}
\usepackage[utf8]{luainputenc}
\setcounter{secnumdepth}{3}
\setcounter{tocdepth}{3}
\usepackage{color}
\usepackage{float}
\usepackage{calc}
\usepackage{graphicx}
\usepackage[numbers]{natbib}
\usepackage{subscript}

\makeatletter

%%%%%%%%%%%%%%%%%%%%%%%%%%%%%% LyX specific LaTeX commands.
%% Because html converters don't know tabularnewline
\providecommand{\tabularnewline}{\\}

%%%%%%%%%%%%%%%%%%%%%%%%%%%%%% Textclass specific LaTeX commands.
\newenvironment{lyxlist}[1]
{\begin{list}{}
{\settowidth{\labelwidth}{#1}
 \setlength{\leftmargin}{\labelwidth}
 \addtolength{\leftmargin}{\labelsep}
 \renewcommand{\makelabel}[1]{##1\hfil}}}
{\end{list}}

%%%%%%%%%%%%%%%%%%%%%%%%%%%%%% User specified LaTeX commands.
\usepackage{multirow}
\usepackage{multicol} 
\usepackage{colortbl}
\usepackage[printonlyused]{acronym}
\usepackage{pslatex}
\usepackage{txfonts}
\usepackage[small,bf]{caption}
\usepackage{acronym}
\usepackage{mathpazo}
\usepackage{listings}
\usepackage{algorithm}
\usepackage{algpseudocode}
\usepackage[Lenny]{fncychap}
\usepackage{fancyhdr}
\usepackage{graphicx}
\lstloadlanguages{Java}
\restylefloat{algorithm}
\floatstyle{plain}
\definecolor{gray}{gray}{0.7}

\makeatother

\usepackage{babel}
\addto\shorthandsspanish{\spanishdeactivate{~<>}}

\usepackage{listings}
\addto\captionsamerican{\renewcommand{\lstlistingname}{Listing}}
\addto\captionsenglish{\renewcommand{\lstlistingname}{Listing}}
\addto\captionsspanish{\renewcommand{\lstlistingname}{Listado de código}}
\renewcommand{\lstlistingname}{Listing}

\begin{document}
\thispagestyle{empty}

\selectlanguage{spanish}%
\begin{center}
Universidad Nacional del Centro de la Provincia De Buenos Aires\\
Facultad de Ciencias Exactas - Departamento de Computación y Sistemas\\
Ingeniería de Sistemas
\par\end{center}

\begin{center}
\includegraphics[scale=0.6]{images/unicen}
\par\end{center}

\vspace{1.5cm}


\begin{center}
{\huge{}Técnicas de aprendizaje para predecir atributos no funcionales
en componentes de aplicaciones Android}\vspace{1cm}

\par\end{center}

\begin{center}
{\large{}por}\\
{\Large{}Agüero, Silvana}
\par\end{center}{\Large \par}

\begin{center}
{\Large{}Minvielle, Martina}
\par\end{center}{\Large \par}

\vspace{1cm}


\begin{center}
{\Large{}\ \ Director: Dr. Alejandro Zunino }
\par\end{center}{\Large \par}

\begin{center}
{\Large{}Co-Director: Ing. Emiliano Sanchez}
\par\end{center}{\Large \par}

\vspace{0.5cm}


\noindent \begin{center}
Trabajo final de carrera presentado como requisito parcial\\
para optar por el título de\\
Ingeniera de Sistemas
\par\end{center}

\vspace{1cm}


\begin{center}
{\large{}Tandil, Marzo de 2017}
\par\end{center}{\large \par}

\newpage{}

\selectlanguage{english}%
\addcontentsline{toc}{chapter}{Resumen}




\chapter*{Resumen}

Durante los últimos años el desarrollo de aplicaciones mobiles ha
ido creciendo exponencialmente debido a la proliferación de los dispositivos
mobiles y el aumento de su capacidad de cómputo, haciendo que la reutilización
de software mediante la integración de componentes de terceros se
torne una práctica muy común. De este modo la misma funcionalidad
suele ser ofrecida por componentes alternativos que difieren en sus
propiedades no-funcionales o atributos de calidad. Por lo tanto, la
elección del servicio más adecuado para ejecutarse en un determinado
contexto, no es tarea sencilla. 

Por razones prácticas es costosa la ejecución o prueba de todos los
servicios para determinar, luego, el más adecuado de acuerdo a su
calidad de servicio y a la funcionalidad requerida. En este punto,
es donde toma importancia definir un proceso para la medición y predicción
de las propiedades no-funcionales. 

A través de la aplicación de modelos predictivos basados en la experiencia,
es posible determinar el servicio que más satisface las condiciones
del entorno, en base a un conjunto de restricciones del contexto de
ejecución, a un conjunto de características de entrada y un conjunto
de servicios que ofrecen la misma funcionalidad. Ya que estas condiciones
o necesidades varían, el objetivo del aprendizaje de máquina es el
desarrollo de sistemas que puedan cambiar su comportamiento (decisión)
de manera autónoma basados en su experiencia (información generalizada).

Se proponen dos herramientas para asistir la aplicación del enfoque.
Una de ellas facilita la obtención de indicadores de propiedades de
componentes Android y la otra herramienta permite utilizar diferentes
técnicas de regresión sobre las mediciones obtenidas previamente.
Así, se logra construir modelos predictivos sobre alguna propiedad
de interés de algún componente específico. 

\selectlanguage{spanish}%
\newpage{}\foreignlanguage{american}{\addcontentsline{toc}{chapter}{Agradecimientos}}


\chapter*{Agradecimientos}

\newpage{}

\tableofcontents{}

\newpage{}\addcontentsline{toc}{chapter}{Índice de Figuras}

\listoffigures


\selectlanguage{american}%
\newpage{}

\addcontentsline{toc}{chapter}{Índice de Cuadros}

\selectlanguage{spanish}%
\listoftables


\newpage{}

\addcontentsline{toc}{chapter}{Glosario}
\pagestyle{plain}

\cleardoublepage{}


\chapter*{Glosario}

\selectlanguage{english}%
\begin{acronym}
\acro{ADB}{Android Debug Bridge}
\acro{ART}{Android Runtime}
\acro{API}{Application Programmatic Interface}
\acro{CC}{Correlation Coefficient}
\acro{CPU}{Central Processing Unit}
\acro{CSV}{Comma-Separated Values}
\acro{DVM}{Dalvik Virtual Machine}
\acro{EPM}{Empirical Performance Model}
\acro{FN}{False Negative}
\acro{FP}{False Positive}
\acro{FTP}{File Transfer Protocol}
\acro{GC}{Garbage Collector}
\acro{GLMM}{Generalized Linear Mixed Models}
\acro{GPU}{Graphics Processor Unit}
\acro{HTTP}{Hypertext Transfer Protocol}
\acro{HTTPS}{Hypertext Transfer Protocol Secure}
\acro{IDE}{Integrated Development Environment}
\acro{iOS}{Iphone Operating System}
\acro{JAR}{Java ARchive}
\acro{Java SE}{Java Standard Edition}
\acro{Java ME}{Java Micro Edition}
\acro{JDBC}{Java Database Connectivity}
\acro{JDK}{Java Development Kit}
\acro{JMH}{Java Microbenchmark Harness}
\acro{JMS}{Java Message Service}
\acro{JNI}{Java Native Interface}
\acro{JRE}{Java Runtime Environment}
\acro{JSON}{JavaScript Object Notation}
\acro{JVM}{Java Virtual Machine}
\acro{LDAP}{Lightweight Directory Access Protocol}
\acro{MAE}{Mean Absolute Error}
\acro{MARS}{Multivariate Adaptive Regression Splines}
\acro{MLP}{MultiLayer Perceptron}
\acro{POP3}{Post Office Protocol}
\acro{REST}{Representational State Transfer}
\acro{RMSE}{Root Mean Absolute Error}
\acro{SMO}{Sequential Minimal Optimization}
\acro{RAE}{Relative Absolute Error}
\acro{RBF}{Radial Basis Function}
\acro{RRSE}{Root Relative Squared Error}
\acro{SGD}{Stochastic Gradient Descendent}
\acro{SQL}{Structured Query Language}
\acro{SVM}{Support Vector Machine}
\acro{TP}{True Positive}
\acro{TSP}{Travelling Salesman Problem}
\acro{UML}{Unified Modeling Language}
\acro{VM}{Virtual Machine}
\acro{XML}{eXtensible Markup Language}
\end{acronym}

\newpage{}

\selectlanguage{american}%
\mainmatter
\renewcommand
\thepage{\arabic{page}}
\setcounter{page}{1}
\pagestyle{fancy}  
\fancyhead{}  
\fancyhead[RO]{\rightmark} 
\fancyhead[LE]{\leftmark}
\acresetall

\selectlanguage{spanish}%

\chapter{Introducci�n}

En los �ltimos a�os las aplicaciones m�viles se han puesto en el centro
de la escena debido a la proliferaci�n de los dispositivos m�viles
y su creciente capacidad de c�mputo y almacenamiento\cite{Zunino2011}.
Estos dispositivos pasaron de ser terminales con capacidades limitadas,
generalmente de prop�sito espec�fico, como agendas electr�nicas o
tel�fonos celulares, a ser peque�as computadoras de prop�sito general
con grandes capacidades de procesamiento, almacenamiento y acceso
a Internet, como tablets y smartphones. 

Los dispositivos m�viles de hoy en d�a utilizan sistemas operativos
similares a las computadoras personales, como Android y iOS. Por lo
tanto, pueden ejecutar software al que hace unos a�os atr�s solo se
encontraba en dichas computadoras. Esto, sumado a su costo accesible,
peque�o tama�o, movilidad y ubicuidad de las conexiones m�viles de
alta velocidad, ha impulsado el desarrollo de las aplicaciones m�viles
para una amplia variedad de fines, incluyendo entretenimiento, juegos,
comunicaciones, redes sociales, comercio electr�nico, turismo, educaci�n,
y mucho m�s. 

Para reducir los costos y tiempos de desarrollo, es com�n la reutilizaci�n
de software mediante la integraci�n de componentes de terceros. Un
componente\cite{Petritsch2016} es una entidad de software en tiempo
de ejecuci�n que encapsula un servicio, es decir, un conjunto de funciones
y datos, a trav�s de una interfaz espec�fica \ac{API}. Los componentes
se pueden clasificar de acuerdo al ambiente en el que residen durante
su ejecuci�n, distingui�ndose as� aquellos que se ejecutan en nodos
remotos, como los denominados Servicios Web \cite{Erickson2009},
de aquellos que residen en el dispositivo, como procesos en segundo
plano, bibliotecas de enlace din�mico, o simples objetos Java. 

Las funcionalidades que implementan estos componentes buscan satisfacer
las necesidades comunes de muchas aplicaciones, como el procesamiento
de texto e im�genes, almacenamiento de datos en la nube, identificaci�n
de usuarios, algoritmos de optimizaci�n, etc. La misma funcionalidad
suele ser ofrecida por componentes alternativos que difieren en sus
propiedades no-funcionales o atributos de calidad \cite{Addison2003}.
Estas propiedades son los aspectos que utilizan los desarrolladores
para juzgar su funcionamiento, tales como performance (por ej., tiempo
de respuesta), disponibilidad (por ej., tasa de errores) o precisi�n
de la respuesta (para el caso de aquellos componentes que procesan
datos de entrada para obtener un resultado de salida). 

Los dispositivos m�viles tienen limitaciones en conflicto como la
energ�a, el acceso a la red y la capacidad de c�lculo que determinan
el contexto de ejecuci�n de estos componentes y que afecta considerablemente
los atributos de calidad de los mismos y de las aplicaciones que los
invocan. Por lo tanto, es importante elegir los componentes adecuados
de acuerdo con su calidad de servicio adem�s de la funcionalidad requerida. 

Sin embargo, estos componentes suelen ser \emph{cajas negras} para
los desarrolladores de aplicaciones m�viles, que tienen acceso a la
definici�n de sus \ac{API}s pero no as� a su implementaci�n interna,
por lo que no cuentan con informaci�n de sus atributos din�micos para
elegir el componente adecuado en cada contexto de ejecuci�n.


\section{Motivaci�n\label{sec:Motivaci=0000F3n}}

Para predecir la performance de un sistema, algunos enfoques definen
un modelo del mismo como funci�n de agregaci�n que considera el desempe�o
individual de sus componentes. As�, por ejemplo, el tiempo de respuesta
de un \emph{mashup} de servicios Web puede determinarse como la suma
de los tiempos de respuesta de los servicios invocados \cite{Rosenberg2009}.
Lo mismo se puede aplicar sobre otras propiedades de arquitecturas
y aplicaciones que involucran el ensamblado de diferentes componentes
\cite{Crnkovic2011}\cite{Sanchez2015}. 

Muchos estudios se enfocan en el an�lisis y predicci�n de los aspectos
din�micos de componentes individuales, siendo las t�cnicas de aprendizaje
de m�quina, y en particular las de regresi�n, las m�s utilizadas.
La mayor parte de estos estudios se enfocan en algoritmos de optimizaci�n
e inteligencia artificial \cite{Hutter2014}, donde se destaca el
trade-off entre los tiempos de ejecuci�n y la precisi�n o calidad
de las respuestas obtenidas. En estos trabajos, se obtienen mediciones
a partir de sucesivas ejecuciones de los algoritmos y se entrenan
modelos con estas mediciones. Los modelos son funciones que dado determinado
algoritmo e instancia de un problema permiten estimar su desempe�o.
Generalmente, diferentes algoritmos se desempe�an mejor que otros
para distintas instancias, por lo que los modelos resultan �tiles
como criterio de decisi�n en la selecci�n y parametrizaci�n autom�tica
de algoritmos, la asignaci�n �ptima de tareas en contextos Grid, entre
otros escenarios. 

La predicci�n de performance tambi�n se ha extendido a servicios Web
\cite{Zheng2013} en una modalidad conocida como predicci�n colaborativa,
en donde las mediciones son recolectadas y compartidas por m�ltiples
nodos distribuidos en todo el mundo. En este caso, los modelos permiten
determinar latencia, disponibilidad, y otras propiedades de un servicio
a partir de la ubicaci�n geogr�fica del cliente que lo invoca y otras
caracter�sticas del contexto. 

En las aplicaciones m�viles, donde el contexto de ejecuci�n y disponibilidad
de los recursos puede variar r�pidamente (cambio de red de acceso
a internet, reducci�n de la bater�a, uso limitado de memoria y \ac{CPU},
etc) y diferentes componentes consumen estos recursos de diferente
forma, es interesante el uso de estos modelos como criterio de calidad
para la selecci�n de servicios y componentes, tanto en tiempo de desarrollo
como en tiempo de ejecuci�n. Por esta raz�n, se plantea la posibilidad
de desarrollar un enfoque para construir y evaluar modelos de predicci�n
en el contexto de aplicaciones m�viles. 


\section{Objetivos y Soluci�n propuesta\label{sec:Objetivos-Soluci=0000F3n-propuesta}}

El objetivo del trabajo consiste en desarrollar un enfoque para la
elaboraci�n de modelos de predicci�n de propiedades din�micas de componentes
accedidos por aplicaciones m�viles. Bas�ndonos en el hecho de que
Android es el sistema operativo m�s difundido para dispositivos m�viles,
el enfoque se pondr� a prueba sobre diferentes casos de estudio en
este sistema operativo. 

El enfoque se basa en un proceso de aprendizaje de m�quina. Dado un
conjunto de componentes que implementan la misma funcionalidad y de
los cuales queremos predecir propiedades din�micas, el m�todo propuesto
es el siguiente: 
\begin{enumerate}
\item Usar conocimiento del dominio para seleccionar caracter�sticas del
contexto y de los datos de entrada del componente que puedan ser indicativos
de su desempe�o. 
\item Generar un conjunto de datos de entrada representativos del espacio
de entrada para la evaluaci�n de los componentes. 
\item Ejecutar los componentes con las entradas generadas y tomar mediciones
de las caracter�sticas identificadas en el punto 1 m�s las propiedades
de inter�s a predecir: tiempo de respuesta, calidad de la respuesta,
etc. 
\item Usar estas mediciones con t�cnicas de aprendizaje de m�quina para
entrenar y evaluar modelos de predicci�n. 
\end{enumerate}
El enfoque propuesto se pondr� a prueba sobre grupos de componentes
y servicios reales que implementan funcionalidades de inter�s para
desarrolladores de aplicaciones m�viles. Esta evaluaci�n no s�lo involucrar�
diferentes dominios, sino tambi�n diferentes t�cnicas, como regresiones
y redes neuronales, sobre diferentes propiedades de inter�s. Para
llevar a cabo la medici�n de los componentes se implementar� una herramienta
de test de performance para la plataforma Android, y se utilizar�
software de aprendizaje de m�quina como Weka \cite{Hall2009}. 

En conclusi�n, se espera poder proveer un an�lisis de los dominios
y t�cnicas consideradas, como as� tambi�n compararlas tanto en t�rminos
de precisi�n como tambi�n en cuanto a su generalizaci�n a diferentes
contextos de ejecuci�n (dispositivos) y datos de entrada


\section{Organizaci�n\label{sec:Organizaci=0000F3n}}

El resto del trabajo se organiza en 5 cap�tulos. A continuaci�n se
da un breve resumen de los temas que se abordan en cada uno de ellos. 

En el cap�tulo\ref{chap:Marco-Teorico} se presenta el marco te�rico,
donde se definen los conceptos utilizados a lo largo de todo el informe,
tales como: Android, desempe�o, precisi�n, aprendizaje de m�quina,
regresi�n, modelos, componentes. 

En el cap�tulo\ref{chap:Trabajos-Relacionados} se presentan herramientas
de benchmarks tanto para el sistema Android como para Java y se exponen
algunos trabajos relacionados desde diferentes perspectivas, predicciones
que incluyen modelos de performance y precisi�n como aproximaci�n
de �xito de los resultados, no s�lo en algoritmos de optimizaci�n
sino tambi�n en servicios Web. 

En el cap�tulo \ref{chap:Enfoque-y-Herramientas} se describe el enfoque
y las herramientas utilizadas. Se detalla la arquitectura e implementaci�n
de las mismas, ahondando en las decisiones de dise�o que se consideran
importantes para el entendimiento y reutilizaci�n de las mismas. 

En el cap�tulo \ref{chap:Evaluaci=0000F3n} se presenta la evaluaci�n
de los modelos obtenidos a partir de todo el framework presentado.
Se presentan las variables consideradas en los escenarios evaluados,
los modelos que han sido analizados y las conclusiones alcanzadas,
presentando justificativo para la selecci�n de los mejores modelos
predictivos alcanzados.

Finalmente, en el cap�tulo \ref{chap:Conclusiones} se exponen las
conclusiones del trabajo realizado, las limitaciones actuales del
enfoque y la herramienta, y posibles l�neas de trabajo futuro.
\acresetall


\chapter{Marco Teórico\label{chap:Marco-Teorico}}

En este capítulo se presentan los conceptos fundamentales del dominio,
el cual está centrado en torno a la predicción de performance y precisión
de componentes de software en dispositivos móviles.


\section{Dispositivos Móviles\label{sec:Dispositivos-m=0000F3viles}}

Los dispositivos móviles son artefactos electrónicos pequeños que
se alimentan a través de una batería de litio. En este contexto, un
smartphone o teléfono inteligente es un teléfono móvil con una mayor
capacidad de cómputo y conectividad que un teléfono móvil convencional.
Mientras el teléfono móvil es un dispositivo inalámbrico electrónico
utilizado para acceder y utilizar los servicios de la red de telefonía
celular, el término \emph{inteligente} hace referencia a la capacidad
de usarlo también como una computadora de bolsillo.

Una de las características más destacadas de los smartphones reside
en la posibilidad que brindan de instalar aplicaciones mediante las
cuales el usuario final logra ampliar las capacidades y funcionalidades
del equipo, obteniendo así una personalización total del dispositivo.
Otras características importantes son la capacidad multitarea, el
acceso y conectividad a Internet vía WiFi o red móvil, el soporte
de clientes de email, la eficaz administración de datos y contactos,
la posibilidad de lectura de archivos en diversos formatos como .pdf
o .doc, y la posibilidad de obtener datos del ambiente a través de
sensores especializados como el acelerómetro y el sistema de posicionamiento
global conocido como GPS por sus siglas en inglés, entre otros. 

Para poder ejecutar aplicaciones en los dispositivos móviles, los
mismos poseen, al igual que las computadoras, sistemas operativos.
Un sistema operativo es un intermediario entre el usuario de un dispositivo
y el hardware del mismo. El objetivo de un sistema operativo es proveer
un ambiente en el cual el usuario pueda ejecutar programas en un manera
conveniente y eficiente. Un sistema operativo es software que administra
el hardware del dispositivo. El hardware debe proveer mecanismos apropiados
para asegurar la operación correcta de un sistema y evitar a los usuarios
interferir con el funcionamiento apropiado del mismo. 

Entre los sistemas operativos más populares en dispositivos móviles
se encuentra Android. En las secciones siguientes se describe detalladamente
este sistema y la framework que provee para el desarrollo de aplicaciones. 


\subsection{Android \label{sec:Android}}

Android es un sistema operativo de código abierto diseñado para dispositivos
móviles tales como smartphones y tablets. Este sistema operativo está
basado en un kernel Linux y es desarrollado por Google. El mismo cuenta
con un middleware extensible y aplicaciones de usuario. Adicionalmente,
posee una plataforma de distribución de aplicaciones disponible a
partir de la versión 2.2 del sistema, denominada Google Play\footnote{https://play.google.com},
que permite a los usuarios navegar y descargar aplicaciones que más
se ajusten a sus necesidades y preferencias, personalizando de esta
forma el dispositivo sencillamente. Por otro lado, también provee
un framework para el desarrollo de aplicaciones\footnote{https://developer.android.com}
que utiliza Java como lenguaje de su interfaz (API). 

La plataforma Android utiliza la máquina virtual Dalvik (\ac{DVM})
para ejecutar aplicaciones programadas en Java a partir de la versión
5. Debido al escaso poder de procesamiento y memoria limitada de los
dispositivos que ejecutan Android, no fue posible utilizar la máquina
virtual Java estándar por lo que la compañía Google tomó la decisión
de crear una nueva, la \ac{DVM}, que fue optimizada para requerir
poco uso de memoria y diseñada para ejecutar en simultáneo múltiples
instancias de la máquina virtual, delegando en el sistema operativo
Android subyacente el soporte para el aislamiento de procesos, la
gestión de memoria e hilos de ejecución. 

En la figura \ref{fig:Android-architecture} se puede ver la arquitectura
en capas empleada por el sistema Android. Las diferentes capas de
la arquitectura son descritas a continuación:

\begin{figure}
\begin{centering}
\includegraphics[scale=0.77]{images/Android-architecture}
\par\end{centering}

\caption{Diagrama de la arquitectura en capas empleada por Android\label{fig:Android-architecture}}
\end{figure}

\begin{itemize}
\item Kernel Linux: Android utiliza el núcleo de Linux como una capa de
abstracción de hardware para los dispositivos móviles. Esta capa contiene
los drivers necesarios para que cualquier componente de hardware pueda
ser utilizado mediante las llamadas correspondientes, sólo debe considerarse
al momento de incluir un nuevo componente de hardware que los fabricantes
hayan desarrollado los drivers correspondientes. Además del soporte
de drivers, la capa es responsable de proporcionar otros servicios
como la seguridad, el manejo de la memoria, procesos, etc. 
\item Entorno de ejecución de Android: como se ha adelantado previamente,
cada aplicación corre en su propio proceso Linux con su propia instancia
de la máquina virtual Dalvik, la cual interpreta un lenguaje ligeramente
diferente al tradicional bytecode de \ac{JVM}. A partir de la versión
5.0 de Android, Dalvik es reemplazada por \ac{ART} la cual logra
reducir el tiempo de ejecución del código Java hasta un 33\%. También
se incluye en el entorno un módulo de librerías nativas con la mayoría
de librerías disponibles en lenguaje Java. Estas bibliotecas, si bien
resultan diferentes a las ofrecidas por \ac{Java SE} y \ac{Java ME},
proveen prácticamente la misma funcionalidad. 
\item Bibliotecas: incluye un conjunto de bibliotecas nativas escritas en
lenguaje C y C++ usadas en varios componentes de Android que proporcionan
la mayor parte de las características de Android.
\item Marco o framework de Aplicaciones: este es el framework que proporciona
Android para el desarrollo de aplicaciones, servicios y otros componentes.
Todo el conjunto de funciones del sistema operativo y bibliotecas
nativas está disponible a través de este framework, cuya API esta
escrita en el lenguaje Java. El framework permite que los desarrolladores
tengan acceso a las mismas \ac{API} utilizadas por las aplicaciones
base del sistema. El foco principal del diseño de esta capa ha sido
simplificar la re-utilización de componentes: las aplicaciones pueden
publicar sus capacidades y otras pueden hacer uso de ellas (sujetas
a restricciones de seguridad), un mecanismo que permite a los usuarios
reemplazar fácilmente componentes. 
\item Aplicaciones: Este nivel contiene todas las aplicaciones de usuario,
tanto las incluidas por defecto en Android así como como aquellas
que el usuario vaya añadiendo posteriormente ya sean de terceros o
de su propio desarrollo. Todas estas aplicaciones utilizan los servicios,
las \ac{API} y bibliotecas de los niveles inferiores. Se brindará
una descripción más detallada en la siguiente sección.
\end{itemize}

\subsection{Aplicaciones Android \label{sec:Aplicaciones-Android}}

Además de las características técnicas, es importante resaltar que
la popularidad de Android ha crecido muy rápidamente desde su lanzamiento.
Esto se debe a la versatilidad que Android otorga a los dispositivos
a través de las aplicaciones, que permiten adaptarlos según las necesidades
de los usuarios. Android también permite que las aplicaciones se adapten
a las características del dispositivo (pantalla, sensores, etc.),
aprovechando las capacidades particulares de cada uno. 

Un aspecto clave del diseño de las aplicaciones en Android es que
éstas pueden reutilizar componentes de otras aplicaciones instaladas
en el dispositivo. Por ejemplo, si una aplicación desea tomar una
fotografía, es probable que ya exista una aplicación que cumpla esa
funcionalidad, entonces, la nueva aplicación puede utilizar la existente
sin necesidad de desarrollar una actividad propia para utilizar la
cámara, esta invocación se realiza de modo tal que sea transparente
para el usuario final.

El sistema Android provee cinco tipos de componentes básicos para
el desarrollo de aplicaciones: Activity, Service, Content Provider,
Broadcast Receiver e Intent. El componente \emph{Activity }representa
una pantalla simple que provee interfaz de usuario. Una aplicación
puede estar formada por una o mas actividades que trabajan en conjunto
y representan diferentes pantallas o vistas. 

Los\emph{ Content Providers} son los encargados de administrar la
información compartida por las aplicaciones. Las aplicaciones pueden
almacenar sus datos en el sistema de archivos, en una base de datos
SQLite, en la Web, o en cualquier otro lugar de acceso. A través del
\emph{content provider}, otras aplicaciones pueden consultar, o incluso
modificar estos datos. 

Las aplicaciones también pueden iniciar servicios que se ejecutan
en segundo plano para realizar operaciones que requieran gran cantidad
de tiempo, o interactúen con procesos remotos, por ejemplo la reproducción
de música en segundo plano o la descarga de datos mientras el usuario
interactúa con una aplicación diferente. 

El componente \emph{Intent} es un objeto de acción que facilita la
comunicación entre componentes. Un intent puede verse como un mensaje
entre componentes, por ejemplo, para iniciar una actividad o servicio,
o solicitar una acción. Por último, el componente \emph{Broadcast
Receiver} funciona como puerta de enlace a otros componentes, respondiendo
a los anuncios (Intents) originados por el sistema u otras aplicaciones,
por ejemplo, cuando la pantalla se apaga, la batería es baja, o al
capturar una fotografía. 


\section{Componentes de software\label{sec:Componentes-de-software}}

En el marco del desarrollo de software nos encontramos con la posibilidad
de reutilizar código previamente desarrollado, testeado y deployado
que cumple con una determinada funcionalidad, ahorrando tiempo y esfuerzo
al desarrollador. Estas piezas de códigos ya implementadas se distribuyen
como componentes de software que encapsulan a un conjunto de funciones
y datos relacionados. 

La re-utilización es uno de los objetivos principales al momento de
diseñar un componente de software de calidad para ser usado en diferentes
programas. La comunicación entre componentes se realiza a través de
interfaces. Cuando un componente ofrece servicios al resto del sistema,
el mismo proporciona una interfaz que especifica los servicios que
otros componentes pueden utilizar y la manera en que pueden hacerlo.
La interfaz puede verse como una firma del componente ya que el cliente
no necesita conocer el procesamiento interno del componente para utilizarlo,
condición que respeta el principio de encapsulamiento. Por otro lado,
cuando un componente necesita de otro para su funcionamiento, el mismo
establece las interfaces requeridas donde especifica los servicios
que necesita. De acuerdo al lenguaje de modelado \ac{UML}, las interfaces
proporcionadas por componentes son representadas con símbolos de lollipop
en el borde del componente y las interfaces requeridas por medio de
sockets abiertos en el borde externo, como se ilustra en la figura
\ref{fig:UML-component}. 

\begin{figure}
\begin{centering}
\includegraphics[scale=0.9]{images/UML-component}
\par\end{centering}

\caption{Componente UML con interfaces proveídas y requeridas.\label{fig:UML-component}}
\end{figure}


Otra de las características fundamentales de los componentes es su
capacidad de ser sustituidos tanto en tiempo de diseño como en tiempo
de ejecución, pudiendo ser reemplazados por actualizaciones u otras
alternativas sin romper el sistema en el que los componentes funcionan.
El reemplazo es posible si el componente sucesor provee al menos la
misma funcionalidad que el componente a reemplazar y si requiere a
lo sumo las mismas funciones que el componente inicial. 


\section{Atributos de calidad \label{sec:Atributos-de-calidad}}

Además de su interfaz y funcionalidad, los componentes de software
se caracterizan por un conjunto de propiedades que representan los
aspectos no-funcionales o de calidad de servicio, llamados propiedades
o atributos de calidad. Estas propiedades son características medibles
que utilizan los usuarios para juzgar su funcionamiento~\cite{McGraw2012}.
Algunos ejemplos de estas propiedades son: el tiempo de respuesta,
la disponibilidad, etc. 

Al diseñar un sistema de software no sólo se espera cumplir con los
objetivos de negocio sino también alcanzar un determinado grado de
calidad de software capaz de satisfacer al grupo de usuarios y/o diseñadores.
El diseño de la arquitectura de un sistema depende mayormente de los
atributos de calidad demandados por los stakeholders. Largos tiempos
de respuesta, caídas del sistema, interfaces confusas, no son características
deseables en un sistema, por lo que toda decisión respecto al diseño
de la arquitectura debe conducir al cumplimiento de ciertos atributos
de calidad al mismo tiempo que cumple con la funcionalidad requerida.

Estos atributos de calidad se pueden dividir en dos grupos en base
al momento en el cual son medidos. Un grupo incluye atributos cuantificados
durante el tiempo de diseño como la escalabilidad, modificabilidad,
etc. El segundo grupo incluye atributos cuantificables mientras el
sistema se ejecuta como usabilidad, seguridad, etc. Este trabajo se
enfoca en la predicción de propiedades dinámicas o en tiempo de ejecución.A
continuación se describen las dos propiedades consideradas en la evaluación
del enfoque. 


\subsection*{Tiempo de respuesta\label{subsec:Atributos-de-calidad-Performance}}

Se puede medir la calidad de un sistema a través de su desempeño (performance)
evaluando la efectividad del uso de los recursos disponibles en tiempo
de ejecución. Dependiendo el contexto, el desempeño puede medirse
a través de varias propiedades, como el tiempo de respuesta o la latencia.
El rendimiento de un sistema engloba, generalmente, el tiempo de los
eventos que se producen y que el sistema debe responder a ellos. Estos
eventos pueden ser muy variados tales como alarmas, mensajes, peticiones
a usuarios o procesamiento, pero básicamente se considera de todos
ellos el tiempo que tarda el sistema para responder al evento. La
complejidad para el manejo de estos eventos radica en su fuente, ya
que pueden provenir desde una solicitud de usuario, de otros sistemas
o desde el interior del propio sistema. 


\subsection*{Precisión\label{subsec:Atributos-de-calidad-Precisi=0000F3n}}

Cabe destacar que no existe una definición estándar sobre el significado
de precisión en un sistema, ya que se trata de una medida que evalúa
qué tan exacta es la respuesta de una función (operación) de un componente,
y cada función está ligada a resolver un problema o funcionalidad
particular. Por ejemplo, en un problema de detección de rostros, la
precisión puede medirse como la cantidad de rostros correctamente
detectados sobre los rostros totales presentes, y en un problema de
optimización puede significar el grado de cercanía del valor de la
solución encontrada respecto a la solución óptima.


\section{Aprendizaje de máquina \label{sec:Aprendizaje-de-maquina}}

El aprendizaje de máquina  o aprendizaje automático es una rama de
la inteligencia artificial cuyo objetivo es desarrollar técnicas que
permitan a las computadoras \emph{aprender} a partir de datos suministrados
en forma de ejemplo. El aprendizaje a partir de datos es la base para
comprender el proceso de aprendizaje de máquina ya que los datos son
la única herramienta de la que se dispone y conoce a ciencia cierta
sobre las características de un dominio cualquiera. El aprendizaje
de máquina puede entenderse haciendo una analogía con el aprendizaje
humano basado en la experiencia, en donde el hombre basa su conocimiento
en tres partes: \emph{i}) recuerdo, el hombre reconoce cuando ha sido
la última vez que estuvo en una determinada situación (dataset), \emph{ii})
adaptación, reconoce la última vez que se probó una acción (salida
producida) y \emph{iii}) generalización, reconoce si ha funcionado
o no esta acción (si fue correcta o no). El término generalización
refiere a la similitud entre diferentes situaciones de manera tal
que las opciones que han sido aplicadas en casos previos pueden ser
usadas en nuevos casos.

El aprendizaje de máquina, entonces, es un proceso para que las computadoras
modifiquen o adapten sus acciones (predictivas o de control) para
que sus resultados sean más precisos, precisión que refleja la proximidad
respecto a las acciones correctas. El aprendizaje de máquina reúne
ideas de neurociencia, biología, estadística, matemática y física,
para generar técnicas y hacer que la computadora aprenda. Un área
importante relacionada con el aprendizaje de máquina es la minería
de datos, el proceso de extraer información útil de un conjunto de
datos masivos por medio de algoritmos eficientes. 

Si se define el aprendizaje de máquina como la mejora de tareas a
través de la experiencia, surge el cuestionamiento de como la computadora
puede saber si está aprendiendo mejor o de qué forma podría mejorar
ese aprendizaje. De aquí, surgen diferentes tipos de técnicas o algoritmos
de aprendizaje. Por ejemplo, se le puede indicar a un algoritmo la
respuesta correcta para un problema, así, la próxima vez que se aplique
su desempeño será mejor. También, podría indicarse un conjunto de
respuestas correctas para que el algoritmo \emph{adivine} la forma
de obtener estas respuestas para otros problemas (generalización).
Alternativamente, se puede indicar si la respuesta obtenida es correcta
o no sin señalar la respuesta real, que el algoritmo debería ser capaz
de encontrar. Una variante podría ser asignarle un puntaje a la respuesta
obtenida por el algoritmo que indique cuán correcta resulta ser. 

Estas diferentes alternativas proveen una forma de clasificar las
diferentes métodos de aprendizaje que será detallada en la siguiente
sección. Cabe destacar que por más que existan distintos tipos de
aprendizaje, todos los métodos comparten el mismo objetivo de generalización:
la técnica debe producir salidas sensibles para datos de entrada que
no fueron encontrados durante el aprendizaje, teniendo en cuenta también
que el algoritmo debe lidiar con ruido en los datos, es decir, imprecisión
en los valores que es inherente a la medición de cualquier proceso
real. 


\subsection{Clasificación de las técnicas de aprendizaje}



El modo de aprendizaje que una técnica particular puede realizar queda
determinado por la naturaleza de los datos de entrada, es decir, los
datos de entrenamiento (\emph{dataset}). Básicamente las técnicas
de aprendizaje se clasifican en tres grandes grupos: aprendizaje supervisado,
no supervisado, y por refuerzo.

El aprendizaje supervisado utiliza un conjunto de datos basado en
dos pares de objetos: los datos de entrada o conjunto de ejemplos
del dominio y las respuestas correctas (\emph{targets}) para una propiedad
determinada . A través de las respuestas correctas provistas y basado
en el conjunto de datos la técnica de aprendizaje generaliza el comportamiento
para responder a todas las posibles entradas. Este modo de aprendizaje,
entonces, es un proceso que se realiza mediante un entrenamiento controlado
por un agente externo que determina la respuesta que debería generar
la técnica a partir de una entrada determinada. 

Dentro del aprendizaje supervisado, las técnicas pueden separarse
en dos grupos de acuerdo a la naturaleza de la propiedad o respuesta. 
\begin{description}
\item [{Clasificación}] Consiste en asignar a cada ejemplo una etiqueta
o clase a la que pertenece basado en el entrenamiento de ejemplares
de cada clase. Los datos de entrenamiento son instancias que pertenecen
a una única clase y el conjunto de clases cubre todas las salidas
posibles, por eso se considera al proceso de clasificación como un
proceso discreto. El algoritmo de clasificación tiene como objetivo
encontrar umbrales de decisión que sirvan para identificar las diferentes
clases.
\item [{Regresión}] El proceso de regresión predice valores numéricos de
atributos a partir de funciones matemáticas polinomiales que describan
o se ajusten lo más posible a todos los puntos del dominio, es decir,
todos los valores del conjunto de entrenamiento correspondientes a
la propiedad que se quiere predecir. Generalmente, se considera un
problema de aproximación de función o interpolación al encontrar un
valor numérico entre los valores conocidos. Por lo tanto, el eje primordial
del proceso de regresión es encontrar la función que mejor represente
al conjunto de puntos, ya que funciones con distintos grados de polinomios
producen diferentes efectos.
\end{description}
En el aprendizaje no supervisado, la máquina simplemente recibe los
datos de entrada sin etiquetas o respuestas correctas como en el método
supervisado, ni valores de recompensa desde el ambiente. Aun así,
es posible desarrollar un framework formal para llevar a cabo aprendizaje
no supervisado basado en la noción de que el objetivo es construir
una representación de la entrada que puede ser usada para tomar decisiones,
predecir futuras entradas, comunicar eficientemente entradas para
otras máquinas, entre otras posibilidades. 

El aprendizaje no supervisado puede entenderse como la búsqueda de
patrones en los datos independientemente del ruido presente en los
mismos. Por ejemplo, las técnicas de agrupamiento (clustering) son
técnicas de aprendizaje no supervisado que agrupan un conjunto de
objetos de modo tal que los objetos pertenecientes a un mismo grupo
(\emph{cluster}) comparten algún tipo de similitud entre ellos, de
igual sentido que se diferencian con los objetos de otro grupo. A
diferencia del proceso de clasificación, los grupos o clases no son
conocidos fehacientemente antes del entrenamiento, un claro método
de aprendizaje no supervisado. 

Por ultimo, el aprendizaje por refuerzo se basa en la idea de no disponer
de ejemplos completos del comportamiento deseado por el algoritmo,
es decir, no indicar durante el entrenamiento exactamente la salida
que se desea proporcione el clasificador ante una determinada entrada,
sólo se le indica si la salida obtenida se ajusta a la deseada y en
función a ello se re configuran los pasos. 


\subsection{Técnicas contempladas\label{subsec:Funciones-contempladas}}

El foco principal del trabajo es el desarrollo de un enfoque para
predecir propiedades no-funcionales de componentes de software en
ejecución, como el tiempo de respuesta, precisión de las respuestas,
entre otros. Estas propiedades son valores continuos, motivo por el
cual se entrenan y evalúan técnicas de regresión para su predicción.
Las técnicas utilizadas se describen a continuación. 




\subsubsection{Regresión Lineal\label{sub:Regresi=0000F3n-Lineal}}

La regresión es la predicción de un valor desconocido a través del
cálculo de una función matemática a partir de los valores conocidos.
Si se considera esta función como una función lineal, la salida será
la suma de cada valor conocido multiplicado por una constante, lo
cual define una línea recta (plano en 3D o hiperplano en dimensiones
mayores) que circundan los puntos, como puede observarse en la Figura
\ref{fig:regression-lineal}. 

\begin{figure}
\begin{centering}
\includegraphics[scale=0.85]{images/regression-lineal}
\par\end{centering}

\caption{Regresiones lineales en dos y tres dimensiones.. \label{fig:regression-lineal}}
\end{figure}


Para encontrar la recta (función lineal) que mejor se \emph{ajusta}
a los datos, se intenta minimizar la distancia entre cada punto y
dicha recta. Esta distancia se mide a través de una línea auxiliar
que atraviesa el punto y tope con la función. Luego, se intentará
minimizar la función de error que que se calcula como la suma de las
distancias. Si se minimiza la suma de los cuadrados de las distancias,
se obtiene la minimización más común llamada optimización de mínimos
cuadrados. 

La minimización de este error, para ajustar la función lineal, puede
realizarse con distintas técnicas de regresión lineal, como \emph{ridge-regression
}y \emph{gradiente estocástico descendiente}. La primera aplica una
penalización (\emph{ridge}) a cada constante. La segunda, aplica un
diferencial sobre la función obteniendo el gradiente el cual por definición,
es la dirección en la que incrementa o disminuye en mayor medida.
Ya que el propósito del aprendizaje es minimizar el error de predicción,
se debe seguir la función en dirección del gradiente negativo en la
cual la función disminuye. 


\subsubsection{Red neuronal}

La técnica de red neuronal presenta un modelo matemático sobre el
comportamiento de una neurona.El mismo representa una célula nerviosa
como \emph{i}) un conjunto de entradas valoradas (w) que corresponde
a las sinapsis, \emph{ii}) un sumador que une las señales entrantes
y \emph{iii}) una función de activación (inicialmente una función
umbral) que decide sobre la activación de la célula en base a las
entradas actuales. 

\begin{figure}
\begin{centering}
\includegraphics[scale=0.85]{images/McCulloch-Pitts}
\par\end{centering}

\caption{Modelo matemático neuronal . \label{fig:McCulloch-Pitts}}
\end{figure}


Como puede observarse en la figura \ref{fig:McCulloch-Pitts}, el
modelo neuronales un dispositivo límite binario, las entradas son
multiplicadas por los pesos y sumando sus valores; si la suma es mayor
a un determinado umbral (produce salida 1) la célula se activa, de
lo contrario (produce salida 0) se mantiene desactivada.

El perceptron es una colección de neuronas con un conjunto de entradas
y pesos que unen las neuronas con dichas entradas. Las neuronas del
Perceptron son completamente independientes entre sí, el estado de
una neurona no infiere sobre las demás compartiendo sólo las entradas.

La esencia del aprendizaje de la red neuronal perceptron está centrada
en los valores de pesos. La red debe ser entrenada para que los pesos
se adapten y generen las respuestas correctas (\emph{targets}).

El entrenamiento de \ac{MLP} consiste en dos partes, primero se obtienen
las salidas con las entradas brindadas y los pesos actuales (\emph{forwards}),
y luego se actualizan los pesos considerando el error como la diferencia
entre el valor obtenido y el real (propagación hacia atrás del error
- \emph{backwards}). 

\begin{figure}
\begin{centering}
\includegraphics[scale=0.85]{images/Perceptron-multilayer}
\par\end{centering}

\caption{Red perceptrón multicapa\label{fig:perceptron-neural-network-multilayer}}
\end{figure}



\subsubsection{K-means clusterer}

El algoritmo K - means aplica clustering sobre los datos de entrenamiento
y recibe un parámetro \emph{K} para dividir estos datos en \emph{K}
categorías. El algoritmo intenta localizar k centros en el espacio
de entrada de modo tal que estos centros estén, como su nombre lo
indica, en el centro de una categoría (\emph{cluster}). La dificultad
se presenta ya que al desconocer la categorización de estos grupos,
resulta aún más difícil determinar la localización de cada centro. 

El objetivo de determinar estos centros, en principio, a causa de
incertidumbre total se posicionan los centros de forma aleatoria en
el espacio de entrada. Una vez distinguidos los clusters, se determinan
los puntos que pertenecen al mismo a través del cálculo de la distancia
entre el punto y todos los centros localizados, asignándose entonces,
al cluster cuyo centro sea el más cercano. Finalmente, para cada centro
se actualiza su ubicación utilizando la media antes definida. Estos
pasos se realizan de forma incremental hasta que los centros dejan
de modificar su ubicación. 

Ya que este algoritmo sin duda es un método de clasificación y no
de regresión, se considera importante resaltar la adaptación del mismo
para utilizarlo con este fin. Así, una vez realizada la clasificación
del punto que se quiere predecir, se realizará la predicción calculando
el promedio de los valores de la propiedad a predecir de los otros
puntos que están en el cluster. 


\subsubsection{Maquina de vector de soporte}

La técnica de máquina de vector soporte \ac{SVM}. es un método propiamente
relacionado con problemas de clasificación y regresión. Dado un conjunto
de ejemplos de entrenamiento (dataset) se pueden etiquetar las clases
y entrenar un SVM para construir un modelo que prediga la propiedad
de un nuevo dataset. Intuitivamente, un SVM es un modelo que representa
a los puntos de muestra en el espacio, separando las clases en dos
espacios lo más amplios posibles mediante un hiperplano de separación.
Analíticamente, se toma la distancia existente entre la línea y el
primer punto interceptado (en dirección perpendicular), si se ubica
una ‘zona desierta’ alrededor de la línea, ningún punto ubicado en
dicha zona puede ser clasificado ya que se encuentra demasiado cerca
de la línea. El radio máximo que puede tener esta región es llamado
margen, señalado como M y los puntos de cada clase más cercanos a
la línea de clasificación se denominan vectores de soporte

\begin{figure}
\begin{centering}
\includegraphics[scale=0.9]{images/SVM-metology}
\par\end{centering}

\caption{Metodología de operación del algoritmo SVM\label{fig:SVM-metology}}
\end{figure}



\subsection{Evaluación de modelos}

Una vez entrenado un modelo de predicción con una técnica, la evaluación
del modelo es importante para medir el nivel de acierto de las predicciones.
Esta evaluación consiste en probar el modelo con un conjunto de datos
de prueba y medir el error u otras métricas sobre los resultados.
Estas métricas de evaluación permite comparar el desempeño de modelos
entrenados con diferentes técnicas. 

Existen distintas formas para llevar a cabo esta evaluación. La mas
simple consiste en usar como datos de prueba el mismo conjunto de
datos utilizado para el entrenamiento de los modelos. Otro método
consiste en separar los datos del problema entre datos de entrenamiento
y datos de prueba. Por ultimo, también se puede validar el modelo
de forma cruzada. 

La validación cruzada (cross-validation) es una técnica utilizada
para evaluar los resultados de un análisis estadístico y garantizar
que son independientes de la partición entre datos de entrenamiento
y prueba. Consiste en repetir y calcular la media aritmética obtenida
de las medidas de evaluación sobre diferentes particiones. Por ejemplo,
si consideramos diez subconjuntos para validación, los datos de entrada
se dividen en diez partes, donde una se reserva para las pruebas y
las otras nueve para el entrenamiento. Este proceso se repite diez
veces y se calcula el promedio de las métricas de evaluación. Esto
ayuda a determinar el nivel al que un modelo se podría generalizar
para nuevos conjuntos de datos. 

El presente trabajo contempla las siguientes métricas de evaluación
para los modelos de regresión: 
\begin{itemize}
\item CC (Coeficiente de correlación de Pearson): el coeficiente de correlación
de Pearson es un índice que puede utilizarse para medir el grado de
relación de dos variables siempre y cuando ambas sean cuantitativas.
En el presente trabajo se considera la correlación entre las variables
del dataset con respecto a la propiedad a predecir.
\item RMSE (Root Mean Absolute Error): el \ac{RMSE} representa la raiz
cuadrática del promedio de la distancia euclídea entre el valor de
la propiedad obtenida por la técnica y el valor real. 
\end{itemize}

\subsection{Ajuste del modelo: Overfitting y Underfitting\label{sub:Ajuste-del-modelo:}}

Cuando se genera (o entrena) un modelo de predicción, su desempeño
es incierto hasta su evaluación o aplicación. En algunas ocasiones,
la calidad del modelo es pobre generando respuestas imprecisas, de
modo tal que se le deben aplicar acciones correctivas comprendiendo
cómo se comporta y ajusta el modelo. 

Los modelos pueden presentar dos problemas indeseables: \emph{overfitting}
y \emph{underfitting}. El \emph{overfitting }describe una función
que se ajusta estrechamente a los datos de entrenamiento. El modelo
aprendió los detalles y el ruido en los datos impactando negativamente
en el desempeño del modelo. Este efecto es causado porque el ruido
o las fluctuaciones aleatorias en los datos de entrada fueron usados
para el aprendizaje. 

Por otro lado, los modelos pueden presentar problemas de \emph{underfitting},
cuando no interpretan bien los datos de entrenamiento, por lo que
son incapaces de generalizar correctamente nuevos datos. Este efecto
es causado porque la función o técnica elegida no es el indicada para
representar el comportamiento de los datos. El efecto underfitting
se caracteriza por sobre generalizar los datos. La incorporación de
nuevos datos al conjunto de entrenamiento podría solucionar o apaciguar
este efecto. 

El modelo deseado, sin dudas, sería aquel que se encuentre en un punto
de equilibrio entre un problema y otro, aunque este equilibrio es
muy difícil de alcanzar en la práctica. La Figura \ref{fig:under-overfitting}
presenta tres modelos de regresión para un mismo grupo de datos que
permiten interpretar gráficamente los problemas de underfitting y
overfitting.

\begin{figure}
\begin{centering}
\includegraphics[scale=0.85]{images/under-overfitting}
\par\end{centering}

\caption{Contraste entre distintos efectos del modelo sobre los datos de entrenamiento.\label{fig:under-overfitting}}
\end{figure}




  
\acresetall
\chapter{Trabajos Relacionados\label{chap:Trabajos-Relacionados}}

El enfoque propuesto es un proceso de aprendizaje de máquina que incluye
una herramienta para recolectar mediciones de performance en Android,
y otra herramienta para entrenar y evaluar modelos de predicción con
diferentes técnicas de aprendizaje automático. En este capítulo se
presentan trabajo relacionados, divididos en dos secciones. Primero,
en la sección \ref{sec:Herramientas-de-benchmarks-para-Android},
se describe un conjunto de herramientas para monitorear y realizar
pruebas de performance en Android. Luego, en la sección \ref{sec:Predicci=0000F3n-de-propiedades-no-funcionales-con-aprendizaje-de-m=0000E1quina},
se presentan algunos trabajos relacionados que también abordan el
problema de predicción de atributos no-funcionales con técnicas de
aprendizaje de máquina.


\section{Herramientas para Android\label{sec:Herramientas-de-benchmarks-para-Android}}

Las herramientas de monitoreo y pruebas de performance para Android
ofrecen a los desarrolladores la posibilidad de analizar los aspectos
no-funcionales de una aplicación Android arrojando datos numéricos
acerca de su desempeño y consumo de recursos. A continuación se describen
algunas de estas herramientas. 


\subsection*{Performance Monitors de Android \label{sub:Performance-Monitors-de-Android }}

Es una herramienta integrada en el ambiente de desarrollo \emph{Android
Studio}\footnote{\emph{https://developer.android.com/studio/profile/android-monitor.html}}
y cuenta con varias sub herramientas que monitorean y proveen información
en tiempo real sobre la aplicación. Los datos capturados se almacenan
en archivos para luego analizarlos en diferentes vistas. Pueden monitorearse
tanto aplicaciones en dispositivos reales conectados o simulados a
través de un emulador.

A través de cinco vistas diferentes se accede a la información sobre
la aplicación evaluada. \emph{LogCat} monitorea las excepciones y
mensajes de log emitidos por la aplicación y el sistema operativo,
útil para deputar la aplicación durante su desarrollo. Las restantes
vistas proveen un monitoreo del consumo de memoria, CPU, GPU y red
por parte de la aplicación.


\subsection*{Benchit\label{sub:Benchit}}

\emph{Benchit}\footnote{\emph{https://github.com/T-Spoon/Benchit}}
es una biblioteca Open Source implementada en lenguaje Java para realizar
mediciones de performance en Android. Benchit es rápido y sencillo
de usar, ya que permite analizar áreas de código para determinar el
tiempo de respuesta o latencia de la operación. Una forma sencilla
de utilizar esta librería es a través de iteraciones sobre una misma
sección de código. De esta forma, con cada iteración, la herramienta
va almacenando el tiempo de ejecución del código (diferencia entre
el tiempo de comienzo y tiempo de fin). Al término de las iteraciones,
se podrá extraer el promedio, rango y desviación estándar de las mediciones.
La herramienta, también provee la posibilidad de realizar comparaciones
de código mostrando los resultados de manera ordenada. Esta opción
es útil, por ejemplo, al momento de comparar el desempeño de distintos
algoritmos o porciones de código que realicen la misma acción de forma
diferente. 


\subsection*{Google Caliper \label{sub:Google-caliper}}

\emph{Google Caliper} es un framework open source para implementar,
ejecutar y visualizar resultados de microbenchmarks en aplicaciones
Java, aunque brinda soporte para proyectos Android. Los microbenchmarks
son mediciones realizadas a funciones simples como llamadas al kernel.
Caliper permite obtener diferentes medidas del código Java, principalmente
microbenchmarks, pero también tiene soporte para otros tipos de medidas
incluyendo memoria disponible, u otras medidas arbitrarias de dominio
específico como por ejemplo el radio de compresión. 


\subsection*{JMeter\label{sub:JMeter}}

\emph{Apache JMeter} es una herramienta Open Source implementada en
lenguaje Java para realizar pruebas de carga y rendimiento de una
variedad de servicios, con énfasis en aplicaciones y protocolos Web.
Apache JMeter se puede utilizar para simular un gran volumen de carga
en un servidor o grupo de servidores y probar su resistencia o analizar
el rendimiento general bajo diferentes tipos de carga. En un principio,
fue diseñada para realizar pruebas de rendimiento sobre aplicaciones
Web pero luego ha sido extendida a otras funciones para cubrir diferentes
categorías de testing, tal es el caso de los análisis de carga, de
funcionalidad, desempeño, regresión, entre otros. JMeter es una aplicación
de escritorio con una interfaz gráfica amigable para el usuario. Puede
ser ejecutada bajo cualquier entorno o estación que acepte la máquina
virtual de Java como es el caso de los sistemas operativos Windows,
Linux y Mac.

A grandes rasgos, la herramienta simula un grupo de usuarios que envían
peticiones a un servidor y retorna un conjunto de estadísticas sobre
el desempeño y funcionalidad por medio de gráficos, tablas, etc, tanto
sobre recursos estáticos y dinámicos. 

El cuadro \ref{tab:benchmarks-tools} presenta y compara las principales
características de las herramientas antes mencionadas.

\begin{table}[h]
\begin{centering}
\includegraphics[scale=0.47]{images/benchmarks_tools}
\par\end{centering}

\caption{Información resumida de herramientas de pruebas de performance para
aplicaciones Android y servicios Web.}
\label{tab:benchmarks-tools}
\end{table}



\section{Predicción de propiedades no-funcionales con aprendizaje de máquina
\label{sec:Predicci=0000F3n-de-propiedades-no-funcionales-con-aprendizaje-de-m=0000E1quina}}

La predicción de propiedades no-funcionales ayuda a los arquitectos
de software en la evaluación de sus sistemas durante la etapa de diseño,
y a guiar las decisiones respecto a que componentes integrar a la
arquitectura del sistema de acuerdo a sus requerimientos de calidad.
Diferentes trabajos han recurrido al uso de técnicas de aprendizaje
de máquina para construir modelos de predicción de performance y otros
atributos de calidad dinámicos. En \citet{Hutter2014}se refieren
a estos modelos como modelos empíricos de performance (\emph{EPM}
por sus siglas en inglés) ya que requieren la recolección de mediciones
empíricas sobre los componentes. 

La principal aplicación de estos modelos probablemente es el problema
de selección de algoritmos, introducido en 1976 por John R. Rice \citep{Rice1976}.
Este problema consiste en seleccionar el algoritmo, o configuración
de algoritmo, de un portafolio de alternativas que minimice el tiempo
de respuesta, según la instancia de datos de entrada. La predicción
del tiempo de respuesta ha sido abordada con éxito usando técnicas
de aprendizaje supervisado, principalmente de regresión \citep{Hutter2014}.
En estos trabajos, los datos empíricos de entrenamiento son obtenidos
en contextos de ejecución controlados, para enfocarse en la correlación
entre la propiedad a predecir y las propiedades de los parámetros
de entrada del algoritmo. Una limitación de estos modelos es que no
generalizan la predicción de las propiedades de performance al contexto
de ejecución, es decir, no consideran características del dispositivo
y el ambiente de ejecución que influyen sobre el desempeño del componente,
como su capacidad de cómputo y la disponibilidad de recursos. 

Los modelos de predicción no sólo se utilizan para estimar el tiempo
de respuesta del desempeño de los componentes, sino también para estimar
otras propiedades. Por ejemplo, en \citet{Roberts07learnedmodels}
se proponen modelos para la predicción del tiempo de respuesta y para
la probabilidad de éxito de diferentes algoritmos de planeamiento
utilizando técnicas de aprendizaje de máquina incluidas en la librería
Weka. Para el aprendizaje fueron utilizados los resultados de 4726
instancias de problemas de planning ejecutadas sobre 28 algoritmos
de planeamiento conocidos. Por cada algoritmo y cada problema se registra
si un plan fue encontrado exitosamente o no, y el tiempo (en segundos)
requerido en completar la ejecución. A partir de esta información
se entrenan y evalúan modelos con diferentes técnicas de aprendizaje
supervisado. El trabajo presentado en \citet{Beveridge2009} también
analiza la precisión de distintos algoritmos, en este caso, para el
reconocimiento de rostros en imágenes, utilizando una técnica de regresión
lineal denominada GLMM, por Generalized Linear Mixed Models. 

Otros autores en \citet{Mersmann_anovel} predicen la optimalidad
o razón de aproximación de algoritmos para el problema del viajante
utilizando una técnica de regresión no lineal llamada \emph{MARS},
por sus siglas en inglés. El modelo \emph{MARS} se aplicó con éxito
para predecir la optimalidad del algoritmo de búsqueda local llamado
2-opt, independientemente del tamaño de la instancia del problema.
Se argumenta que es sencillo aplicar la misma metodología a otros
algoritmos y utilizar estos modelos para derivar una estrategia para
el problema de selección de algoritmos en el contexto del problema
del viajante.

La predicción de propiedades dinámicas de servicios Web, como su tiempo
de respuesta, y probabilidad de fallos, es más compleja de abordar
con respecto a algoritmos y componentes ejecutados localmente ya que
depende de factores propios de la infraestructura de red y el proveedor
del servicio, que no se puede monitorear desde el dispositivo cliente.
Un enfoque ingenioso para abordar este problema fue propuesto en el
articulo de Zheng\citep{Zheng2013}. En este trabajo, los autores
se basan en la premisa de que las propiedades de performance de los
servicios Web varían respecto a características del contexto, como
la ubicación geográfica y el momento del día y la semana en el que
se realiza una solicitud al servicio. De esta forma, recolectan la
información del consumo de servicios de múltiples clientes alrededor
del mundo para generalizar modelos de predicción. Este enfoque es
conocido como predicción colaborativa ya que los datos empíricos de
entrenamiento son brindados por múltiples nodos de manera distribuida.
Los autores llevaron a cabo un experimento a gran escala que involucró
más de 30 millones de invocaciones a servicios Web en más de 80 países,
por usuarios distribuidos en más de 30 países. La observación experimental
indica que diferentes usuarios pueden tener diferentes experiencias
de uso sobre el mismo servicio, influenciados por la conexión de red
y los ambientes heterogéneos entre usuarios y proveedores. Los datos
se encuentran disponibles públicamente y han sido utilizados en varios
trabajos para la construcción y comparación de modelos de performance
con diferentes técnicas de aprendizaje de máquina \citep{Albu2013}\citep{Kumar2015}. 
\acresetall


\chapter{Enfoque y Herramientas\label{chap:Enfoque-y-Herramientas}}

En este cap�tulo se describe el enfoque propuesto para extraer informaci�n
y conocimiento de un conjunto de caracter�sticas inherentes a componentes
de software y propiedades est�ticas y din�micas del dispositivo android
de ejecuci�n para analizar las relaciones y dependencias y predecir
atributos no funcionales en base a dichas propiedades. Para el dise�o
se hizo �nfasis en la optimizaci�n combinada de los par�metros de
los algoritmos de aprendizaje autom�tico. 

El enfoque plantea llevar a cabo la predicci�n de propiedades no-funcionales
mediante un proceso de aprendizaje de m�quina a trav�s de \emph{i})
la recolecci�n de indicadores o mediciones tomados a partir de la
informaci�n provista de la ejecuci�n de piezas de software, considerando
atributos de componentes, atributos inherentes al problema de entrada,
y atributos de la operaci�n y resultados de la ejecuci�n, y \emph{ii})
la construcci�n de modelos como un proceso interactivo con el usuario
a partir de la configuraci�n inicial de los datos, la optimizaci�n
autom�tica de los par�metros de acuerdo a las tasas de error arrojadas
por m�tricas de evaluaci�n y el ajuste final del modelo a trav�s del
an�lisis de las curvas de aprendizaje. 

Como soporte para el enfoque, se desarrollaron dos herramientas independientes
entre s� y dise�adas para efectuar el objetivo y conexi�n de las dos
fases propuestas. Por un lado, se desarroll� una herramienta denominada
\emph{Android Performance Testing and Prediction} cuyo dise�o se adapta
f�cilmente a la implementaci�n de cualquier dominio computacional
del que se quiera obtener indicadores de desempe�o. Al tratarse de
un framework implementado para el sistema Android, permite obtener
de manera directa los benchmarks del dispositivo de inter�s para el
an�lisis. Por otro lado, se desarroll� una herramienta standalone
denominada \emph{Nekonata} dise�ada para brindar soporte al uso de
las funciones de cualquier librer�a que realice aprendizaje autom�tico
y miner�a de datos escritas en lenguaje Java y que consiste en dos
fases, una primer etapa para la construcci�n del modelo a partir del
conjunto fuente de benchmarks mediante un proceso de automatizaci�n
de los algoritmos en complemento de informaci�n gr�fica para la colaboraci�n
interactiva del usuario y finalmente una segunda etapa de ajustes
al modelo teniendo en cuenta los efectos de overfitting y underfitting
consecuentes del entrenamiento. 

Estas cuestiones se describen en detalle de la siguiente manera. En
la secci�n \ref{sec:Aplicaciones-de-la} se enumeran algunos de los
usos pr�cticos de los modelos incluyendo aplicaciones de la propuesta.
Luego, en la secci�n \ref{sec:Etapas-del-m=0000E9todo} se profundiza
sobre las distintas etapas del enfoque y flujo de trabajo. En la secci�n
\ref{sec:Herramientas} se describen cada uno de los frameworks desarrollados,
presentando la herramienta para la recolecci�n de datos en la subsecci�n
\ref{subsec:Framework-de-medici=0000F3n} y finalmente, en la secci�n
\ref{subsec:Herramienta-de-entrenamiento} se presenta la herramienta
para la construcci�n de modelos evaluativos. 


\section{Aplicaciones de la propuesta \label{sec:Aplicaciones-de-la}}

Los problemas de clase NP - Completos est�n presentes en la mayor�a
de �mbitos computacionales. Afortunadamente, en la medida que estos
problemas resultan dif�ciles de resolver frente a los peores casos
de entrada, se hace m�s factible resolverlos a�n considerando problemas
de grandes instancias. 

Desafortunadamente, estos algoritmos pueden exhibir variaciones extremas
de ejecuci�n a trav�s de las instancias con distribuciones reales,
incluso, aunque la dimensi�n del problema se mantuviera constante,
la misma instancia puede requerir dram�ticamente diferentes tiempos
de ejecuci�n en funci�n del algoritmo utilizado. Existe una escasa
comprensi�n te�rica de las causas de esta variaci�n. Durante la �ltima
d�cada, una cantidad considerable de trabajo ha intentado demostrar
c�mo utilizar las t�cnicas de aprendizaje autom�tico supervisado para
la construcci�n de modelos de regresi�n que proporcionen respuestas
aproximadas a esta pregunta en base a los datos de rendimiento del
algoritmo analizado, en otras palabras, podr�a creerse que es posible
predecir el tiempo que requerir� un determinado algoritmo para ejecutarse
bajo una entrada en particular construyendo un modelo de tiempo de
ejecuci�n del algoritmo como una funci�n de las caracter�sticas espec�ficas
de cada instancia del problema. 

La construcci�n de tales modelos conocidos como \emph{modelos de actuaci�n
emp�rica }(EPM por sus siglas en ingl�s) ha ido creciendo y motivada
debido a la utilidad que presentan frente a una gran variedad de contextos
pr�cticos. A continuaci�n, se detallan algunos:
\begin{description}
\item [{Selecci�n~de~algoritmos}] Como se ha tratado en algunos trabajos
\citet{Hutter2014}, los modelos de predicci�n son �tiles para la
selecci�n autom�tica de algoritmos y la configuraci�n en una variedad
de formas (un problema cl�sico de selecci�n del mejor algoritmo entre
un determinado conjunto); a trav�s de \ac{EPM} se logra predecir
el rendimiento de cada uno de estos algoritmos candidatos y mediante
un an�lisis comparativo, seleccionar el m�s apropiado considerando
la instancia del problema y las caracter�sticas del hardware. 
\item [{Ajustes~de~par�metros~y~configuraci�n~autom�tica~del~algoritmo}] \ac{EPM}
sirve a dos prop�sitos fundamentales, por un lado, modelar el comportamiento
o funcionalidad de un algoritmo parametrizado en base a la configuraci�n
de tales par�metros, en cuyo caso se puede alternar entre el aprendizaje
del modelo y su uso para identificar configuraciones interesantes
para evaluar posteriormente. Por otro lado, se puede modelar el rendimiento
del algoritmo basado conjuntamente en las caracter�sticas de las instancias
del problema y la configuraci�n de sus par�metros. Tales modelos pueden
utilizarse para ajustar los valores de tales par�metros y obtener
una mejor predicci�n basada en la instancia particular. 
\item [{Generaci�n~de~benchmarks~fuertes}] Un modelo predictivo para
uno o m�s algoritmos se puede utilizar para establecer los par�metros
de los generadores de benchmarks existentes con el fin de crear instancias
asociadas al algoritmo particular. 
\item [{Obtener~una~visi�n~general~de~las~instancias~y~el~rendimiento~de~los~algoritmos}] \ac{EPM}
se puede utilizar para evaluar las caracter�sticas de la instancia
y los valores de los par�metros del algoritmo que m�s impactan en
el rendimiento. Algunos modelos son compatibles con este tipo de evaluaciones
directamente. Para otros modelos, existen m�todos de selecci�n de
atributos (caracter�sticas gen�ricas) para identificar un grupo m�s
reducido de entradas del modelo que son claves, y describen el rendimiento
del algoritmo casi tan bien como todo el conjunto de entradas. 
\item [{Selecci�n~de~Servicio~y~composici�n}] Cuando varios servicios
web implementan la misma funcionalidad, los modelos de rendimiento
resultan ser un buen criterio para escoger el mejor candidato entre
ellos. Incluso en tiempo de ejecuci�n, los proveedores de servicio
pueden cambiarse si las condiciones del contexto y los par�metros
de entrada se modifican. 
\item [{Programaci�n~de~tareas~en~redes~m�viles}] Suponiendo un conjunto
de tareas que deben asignarse entre un conjunto de dispositivos, los
modelos de rendimiento podr�an obtener una medida exacta del tiempo
de respuesta que cada tarea requerir� sobre cada dispositivo con el
fin de minimizar el tiempo total de secuenciaci�n de las tareas. 
\item [{Otros.}]~
\end{description}

\section{Etapas del m�todo\label{sec:Etapas-del-m=0000E9todo}}

El enfoque propuesto se conceptualiza como un proceso de tres fases
complementarias. Este ciclo o flujo de trabajo da lugar a tres etapas
bien definidas por cada dominio o escenario de estudio, desde la obtenci�n
de indicadores hasta la predicci�n de propiedades no funcionales en
entornos de aplicaci�n. La figura \ref{fig:method-stages} muestra
un esquema conceptual del enfoque, cuyas etapas se describen a continuaci�n:

\begin{figure}
\begin{centering}
\includegraphics[scale=0.55]{C:/Users/usuario/Tesisworkspace/Tesis_Standalone/tesis/images/method-stages}
\par\end{centering}

\caption{Esquema conceptual del enfoque en fases. \label{fig:method-stages}}
\end{figure}

\begin{description}
\item [{Testing}] El proceso comienza con la creaci�n de datasets. Cada
dataset pertenece a un escenario o dominio diferente del cual se extraen
todas las caracter�sticas que podr�an influir sobre el desempe�o del
componente. Durante la ejecuci�n de cada servicio o pieza de software
se van realizando las mediciones o m�tricas sobre distintos aspectos
de la operaci�n y registrando cada uno de estos benchmarks en un archivo
para su posterior an�lisis. 
\item [{Learning}] A partir del conjunto de benchmarks obtenidos, se aplican
t�cnicas de aprendizaje de m�quina para la extracci�n de conocimiento
de estos datos y construir, consecuentemente, modelos predictivos
que mejor se ajusten a la generalizaci�n de la informaci�n mediante
un proceso de entrenamiento y evaluaci�n de los mismos. 
\item [{Predict}] Finalmente, se pretende utilizar estos modelos de predicci�n
en entornos de aplicaci�n que permitan la selecci�n del componente
m�s adecuado en base a un conjunto de propiedades del problema de
entrada, las propiedades internas del dispositivo en el cual se llevar�
a cabo la ejecuci�n, y un conjunto de restricciones que deben satisfacerse,
a trav�s de un proceso automatizado que determine al usuario la opci�n
m�s favorable evitando la ejecuci�n de cada componente. 
\end{description}

\section{Herramientas\label{sec:Herramientas}}

El trabajo presentado conforma dos de las tres fases propuestas para
el enfoque global del desarrollo. La primer fase se lleva a cabo en
un framework particular para la medici�n de propiedades de componentes
Android que ser� detallada en la secci�n \ref{subsec:Framework-de-medici=0000F3n}.
La segunda fase para la construcci�n de modelos predictivos a trav�s
de t�cnicas de aprendizaje de m�quina se desarrolla en una segunda
herramienta la cual ser� detallada en la secci�n \ref{subsec:Herramienta-de-entrenamiento}. 


\subsection{Framework de medici�n para Android\label{subsec:Framework-de-medici=0000F3n}}


\subsubsection*{Enfoque general}

\textquotedblleft Android Performance Testing and Prediction\textquotedblright{}
es un framework dise�ado para realizar mediciones de performance de
componentes ejecutados bajo el sistema Android. Es una herramienta
de testing que permite \textquotedblleft correr\textquotedblright{}
diferentes piezas de software y evaluar propiedades influyentes y
caracter�sticas del entorno de ejecuci�n. 

La herramienta cuenta con el soporte necesario para adaptarse a cualquier
dominio inform�tico y exportar las mediciones correspondientes en
archivos de formato CSV, mediciones que servir�n de fuente para herramientas
de predicci�n a trav�s de t�cnicas de aprendizaje de m�quina. 


\subsubsection*{Objetivos}

El rendimiento de los componentes de ejecuci�n (algoritmos, servicios
Web, procesos ejecut�ndose en segundo plano, etc.) dependen de varios
factores: \textquotedblleft el contexto de ejecuci�n\textquotedblright{}
donde el componente est� funcionando, los \textquotedblleft par�metros
de entrada\textquotedblright{} que requiere el componente de la operaci�n,
y su \textquotedblleft implementaci�n interna\textquotedblright .
Sin embargo, al utilizar componentes de terceros, generalmente se
desconoce su implementaci�n actuando como \textquotedbl{}cajas negras\textquotedbl{}
a los desarrolladores m�viles, que s�lo conocen sus interfaces de
aplicaciones, pero no su trabajo interno. 

Para elaborar un an�lisis del rendimiento, t�cnicas de aprendizaje
autom�tico sobre los datos recogidos emp�ricamente pueden ser utilizados
para construir modelos de predicci�n del tiempo de ejecuci�n del componente,
como una funci�n de los par�metros de entrada y las caracter�sticas
espec�ficas del contexto de ejecuci�n: configuraci�n de algoritmos,
selecci�n de servicios, planificaci�n de trabajos, por citar algunos
ejemplos. 


\subsubsection*{Enfoque}

El enfoque principal de esta herramienta est� centrado en dos propiedades:
el tiempo de respuesta y la precisi�n de los resultados. 

El tiempo de respuesta refiere al tiempo total que demanda un componente
en ejecutar una operaci�n o tarea, es decir, el tiempo para responder
a una solicitud con una entrada determinada. Por otro lado, la precisi�n
es una medida que eval�a la calidad de los resultados o salida de
un componente y tiene diferentes significados sem�nticos dependiendo
de la funcionalidad requerida, por ejemplo, en problemas de clasificaci�n,
se toma el concepto de precisi�n como la medida estad�stica de la
eficacia de un clasificador, la precisi�n est� relacionada a la identificaci�n
o exclusi�n correctamente de una condici�n. En problemas de optimizaci�n,
tambi�n conocido como optimalidad, la precisi�n es una relaci�n entre
la soluci�n de salida obtenida y la soluci�n �ptima conocida. 

Para construir un modelo de predicci�n del tiempo de respuesta sobre
un componente particular, se deben considerar dos variables al momento
de evaluar, un conjunto acerca de las caracter�sticas del contexto
de ejecuci�n (E,) por ejemplo, n�cleos de CPU, tipo de red, etc. y
otro conjunto acerca de las caracter�sticas de la entrada en particular
(I), por ejemplo, en tama�o expresado en bytes. Por lo tanto, el modelo
resulta una funci�n R(c) que depende tanto de E como de I. 



Del mismo modo se construye un modelo de predicci�n de precisi�n sobre
un componente, como en la mayor�a de los casos esta medida no depende
de las caracter�sticas del contexto en el que se ejecuta, el modelo
responder�a a una funci�n como la siguiente:




\subsubsection*{Componentes}

El an�lisis de rendimiento que se propone en esta herramienta se basa
en entidades de software individuales de ejecuci�n que proveen servicios
a trav�s de una interfaz espec�fica. De aqu� en m�s, estas entidades
se denotar�n componentes. 

\begin{figure}
\begin{centering}
\includegraphics[scale=0.55]{C:/Users/usuario/Tesisworkspace/Tesis_Standalone/tesis/images/android-component}
\par\end{centering}

\caption{Esquema conceptual de componentes Android considerados. \label{fig:android-component}}
\end{figure}


En la figura \ref{fig:android-component} se pueden observar los tres
tipos de componentes generalizados por la herramienta para obtener
los resultados de mediciones adecuados, incluyendo servicios Web y
servicios espec�ficos de la plataforma Android, para toda pieza de
software remanente los componentes son tratados simplemente como objetos
Java. Las instancias de objetos hacen referencia a cualquier componente
residente en el espacio de memoria de una aplicaci�n, el componente
espec�fico para servicios web incluye cualquier componente remoto
fuera del dispositivo y accedidos a trav�s de protocolos de comunicaci�n
Web, como \ac{HTTP}. Por �ltimo el componente espec�fico para servicios
Android incluye cualquier proceso ejecutado en segundo plano residente
en el mismo dispositivo y accedidos a trav�s de objetos Intent (como
�nico mecanismo de comunicaci�n entre procesos del sistema Android). 


\subsubsection*{Dise�o e implementaci�n}

El framework \textquotedblleft Android-Performance-Testing-and-Prediction\textquotedblright{}
est� conformado por tres proyectos individuales. El proyecto \textquotedblleft Android-Testing-Tool\textquotedblright{}
constituye el modelo base para llevar a cabo el proceso de medici�n
de propiedades de cualquier tipo de componente. Los proyectos restantes
han sido implementados para desarrollar diferentes dominios de aplicaci�n,
los cuales incluyen la dependencia al proyecto modelo. El proyecto
\textquotedblleft Evaluation-of-Face-Detection-Services\textquotedblright{}
fue dise�ado para obtener mediciones sobre servicios que ofrecen reconocimiento
facial y el proyecto \textquotedblleft Examples-Android-Testing-Tool\textquotedblright{}
fue dise�ado para dominios de problemas NP, incluyendo a problemas
de la clase P y NP - Completos. 

A continuaci�n se expone en la figura \ref{fig:TestingTool-Diagram}
los principales factores que han servido de gu�a para el dise�o del
proyecto base \textquotedblleft Android-Testing-Tool\textquotedblright . 

\begin{figure}
\begin{centering}
\includegraphics[scale=0.1]{C:/Users/usuario/Tesisworkspace/Tesis_Standalone/tesis/images/TestingTool-Diagram}
\par\end{centering}

\caption{Diagrama estructural de la herramienta template \textquoteleft Android
Testing Tool\textquoteright .\label{fig:TestingTool-Diagram}}


\selectlanguage{english}%
\selectlanguage{english}%
\end{figure}


La forma de organizar diferentes modelos de evaluaci�n se realiza
a trav�s de la creaci�n de planes de prueba, que se configuran y posteriormente
se ejecutan para obtener un archivo de formato CSV con el resultado
de todas las mediciones incorporadas al plan de pruebas. La herramienta
implementa por defecto un objeto llamado \textquotedblleft TestPlan\textquotedblright{}
que define una ejecuci�n sistem�tica de las operaciones y m�tricas
sobre ellos. B�sicamente, un modelo de plan de pruebas se compone
de un conjunto posible de componentes, un conjunto posible de objetos
de entrada, y un conjunto posible de m�tricas para realizar las mediciones,
dando la posibilidad de configurar estas propiedades a los requerimientos
deseados. Una vez configurado y ejecutado el plan de pruebas, los
resultados con las mediciones son almacenados en un objeto denominado
\textasciiacute Results\textquoteright{} y los datos son exportados
a un archivo de formato CSV para su posterior procesamiento. 

La herramienta utiliza una representaci�n simplificada de los componentes,
pudiendo ser �stos servicios Web o simplemente piezas de software
implementadas por el programador. Los par�metros que recibe son dos
objetos del tipo Input y Output, siendo �stos, instancias de objetos
que representan la entrada y salida del problema respectivamente.
Cada uno de los componentes asociados a un problema espec�fico ejecutan
una operaci�n o tarea a trav�s de la llamada al m�todo execute del
componente. Esta operaci�n es responsable de evaluar la ejecuci�n
de la instancia de entrada en el componente y retornar un objeto de
salida o respuesta del problema. Tanto \textasciiacute Input\textasciiacute{}
como \textasciiacute Output\textasciiacute{} son dos conceptos abstractos
que representan instancias reales del problema. Una instancia de entrada
puede encapsular no s�lo los par�metros requeridos para realizar la
ejecuci�n del componente, sino la configuraci�n de los mismos. Por
otro lado, una instancia de salida encapsula la respuesta o resultado
de esa operaci�n. Durante la ejecuci�n del plan de pruebas (ejecuci�n
de cada uno de los componentes asociados), se eval�an los indicadores
que fueron configurados como m�tricas en el plan. Se distinguieron
cuatro tipos de m�tricas en funci�n del elemento de medici�n, determinando
el par�metro que recibe: 

1. M�tricas globales sobre caracter�sticas est�ticas del contexto:
cualidades del entorno de ejecuci�n que se mantienen est�ticas (sin
cambio) durante el plan de pruebas, por ejemplo, modelo del dispositivo,
arquitectura de la CPU, cantidad de n�cleos de CPU, tama�o de memoria,
etc. 

2. M�tricas generales sobre caracter�sticas de la entrada: atributos
inherentes al dominio del problema, por ejemplo, en el problema de
detecci�n de rostros algunas caracter�sticas podr�an ser el nombre
de imagen, tama�o o contraste, formato de archivo, etc.

3. M�tricas generales del componente: propiedades del componente como
el nombre, su ubicaci�n, etc.

4. M�tricas de operaci�n: Son medidas que act�an sobre la ejecuci�n
del componente, por lo cual distinguen en tres tipos diferentes: \textbullet{}
Caracter�sticas de la salida: caracter�sticas del resultado de la
operaci�n como su tama�o, calidad, etc. al igual que las entidades
de entrada, son inherentes al dominio del problema. En la detecci�n
de rostros, por ejemplo, la salida podr�a ser un vector con la ubicaci�n
de los puntos detectados y una caracter�stica de salida interesante
podr�a ser, el tama�o de ese vector, es decir, el n�mero de rostros
detectados. \textbullet{} Caracter�sticas din�micas del contexto:
caracter�sticas del entorno de ejecuci�n que pueden variar de una
operaci�n a otra, como el uso de CPU, el n�mero de procesos en ejecuci�n,
tipo de conexi�n, ubicaci�n del dispositivo, etc. \textbullet{} Caracter�sticas
de desempe�o: medidas de inter�s sobre el rendimiento que var�an de
una operaci�n a otra, como el tiempo de respuesta, el consumo de bater�a,
operaciones ejecutadas con �xito o con error, etc.

El framework de medici�n ha sido dise�ado para soportar distintos
dominios sobre los cu�les tomar las medidas deseadas. Por el momento
s�lo implementa dos tipos de dominios: por un lado, problemas cl�sicos
del tipo NP para el an�lisis del desempe�o de diferentes algoritmos
de resoluci�n y por otro lado, aplicaciones de prop�sito general,
en este caso, aplicaciones que ofrecen reconocimiento facial para
la evaluaci�n de diferentes servicios que ofrecen esta misma funcionalidad,
como ya ha sido anticipado. Los problemas de complejidad NP se contemplan
bajo un proyecto individual llamado \textquotedblleft Examples-Android-Testing-Tool\textquotedblright{}
para la evaluaci�n de desempe�o de distintos algoritmos considerando
dos aspectos fundamentales: tiempo (aproximaci�n del n�mero de pasos
de ejecuci�n que emplea el algoritmo) y espacio (aproximaci�n de la
cantidad de memoria utilizada). Los problemas implementados pertenecen
tanto a la clase P como a la clase NP-Completos. Por otro lado, el
objetivo de la incorporaci�n de dominios que implican el uso de servicios
remotos se basa en la idea de automatizar el testeo continuo de uno
o varios servicios, en este caso, aplicaciones que ofrecen reconocimiento
facial, teniendo en cuenta las caracter�sticas que estos pueden proveer
y aquellas que el usuario de la aplicaci�n desee considerar, de esta
manera determinar el servicio m�s adecuado (seg�n el contexto) para
la ejecuci�n de una tarea. Este dominio ha sido dise�ado a trav�s
de un proyecto individual bajo el nombre de \textquotedblleft Evaluation-of-Faces-Detection-Services\textquotedblright .
Este proyecto se implement� como un modelo b�sico y gen�rico sobre
el proceso de detecci�n de rostros. Tal proyecto conforma una estructura
general para almacenar y acceder a todos los atributos posibles que
cualquier servicio que brinde la funcionalidad de detecci�n de rostros
pudiera ofrecer. Es importante destacar algunos detalles sobre los
componentes, entradas y m�tricas que fueron tomados en cuenta en el
dise�o del proyecto. Cada componente representa un servicio particular,
cuya performance es evaluada y analizada; en base a los par�metros
de configuraci�n que acepta cada uno de estos servicios, se generan
diferentes instancias de componentes posibles, un servicio que admite
dos variables de configuraci�n, por ejemplo, da lugar a cuatro tipos
de componentes como producto de la combinaci�n de opciones configurables
posibles. En cuanto a las entradas del dominio, pueden a�adirse al
plan de pruebas a trav�s de archivos o de manera est�tica especificando
la ruta de cada imagen. Finalmente, fueron consideradas m�tricas respecto
al proceso u operaci�n (rostros correctamente detectados, rostros
detectados y Tiempo de respuesta) y respecto a la imagen de entrada
(contraste, entrop�a, formato, intensidad promedio, cantidad de rostros
en la imagen, cantidad de p�xeles de la imagen, tama�o y nombre).
Respecto al servicio, s�lo se registra el nombre del mismo. 


\subsection{Herramienta de entrenamiento y evaluaci�n de modelos\label{subsec:Herramienta-de-entrenamiento}}

El eje que ha guiado el dise�o de la herramienta ha sido el brindar
soporte para el uso de cualquier librer�a que ofrezca aprendizaje
de m�quina a trav�s de la implementaci�n de todos los conceptos involucrados
como objetos independientes a los cuales adaptar las funcionalidades
de las librer�as. La herramienta \emph{Nekonata} lleva a cabo dos
tipos de procesos en el desarrollo de sus funciones, procesos automatizados
y procesos que requieren la colaboraci�n del usuario con el fin de
mejorar los resultados a partir de vistas gr�ficas sobre el comportamiento
y caracter�sticas de los modelos y adem�s, para incluir en el proceso
el inter�s y criterio del usuario. 

El enfoque y objetivo de la herramienta se concluyen en dos etapas,
una fase de entrenamiento de modelos y una fase de evaluaci�n y mejora
de los mismos. 

En la figura \ref{fig:prediction-tool} se muestran las vistas iniciales
de la herramienta , presentaci�n y configuraci�n de datos. 

\begin{figure}
\begin{centering}
\includegraphics[scale=0.55]{C:/Users/usuario/Tesisworkspace/Tesis_Standalone/tesis/images/prediction-tool}
\par\end{centering}

\caption{Captura de pantalla de la herramienta: (A) Presentaci�n, (B) Configuraci�n
de datos, (C) Men� de selecci�n de opciones.\label{fig:prediction-tool}}
\end{figure}



\subsubsection{Entrenamiento de modelos}

En esta secci�n se brindar� un enfoque global y expresar�n las consideraciones
de los conceptos relacionados al proceso de entrenamiento. Posteriormente,
se describe en detalle el flujo de desarrollo del proceso. Estos conceptos
han sido el eje del dise�o de la herramienta. A continuaci�n, se presenta
un listado de los objetos tratados: 


\subsubsection*{Librerias}

Al d�a de hoy se han implementado bibliotecas que realizan aprendizaje
autom�tico en m�ltiples lenguajes de programaci�n. A pesar de la gran
variaci�n que existe entre unos y otros, el uso de t�cnicas de aprendizaje
de m�quina se rige bajo los mismos conceptos: durante la etapa de
entrenamiento, un conjunto de datos que servir�n como base de datos
del proceso y una lista de algoritmos categorizados seg�n realicen
una funci�n de regresi�n o clasificaci�n (distinci�n que lleva a cada
biblioteca definir los tipos de atributos num�ricos y categ�ricos),
finalmente durante la etapa de evaluaci�n, es necesario el uso de
m�tricas o indicadores matem�ticos para evaluar la calidad del clasificador.
Estos principios permiten generalizar el t�rmino \emph{librer�a} independizando
la implementaci�n espec�fica de cada biblioteca a�adida al sistema. 

La herramienta Nekonata utiliza la biblioteca Weka, una plataforma
de software para el aprendizaje autom�tico y la miner�a de datos escrito
en Java y desarrollado en la Universidad de Waikato y popularmente
conocida ya que contiene una extensa colecci�n de t�cnicas para preprocesamiento
de datos y modelado. 


\subsubsection*{Base de datos}

Los archivos dataset usados para el entrenamiento, a menudo son vistos
como una grilla de valores cuyas columnas representan cada una de
las caracter�sticas del dominio com�nmente denominadas clases o atributos
y cuyas filas denominadas instancias representan cada uno de los ejemplos
del escenario de estudio. Bajo estas consideraciones, a nivel conceptual
existe un objeto �nico como dataset que obtiene los datos a trav�s
de un proceso de parseo del archivo fuente a los objetos correspondientes
de la librer�a utilizada, generalizando toda funcionalidad requerida
mediante el acceso a la representaci�n de los atributos e instancias. 


\subsubsection*{Instancias}

Tal como se adelant� anteriormente, cada dataset est� formado por
un conjunto de ejemplos tomados de un dominio en particular. Cada
ejemplo constituye una instancia individual del problema representada
por un conjunto de dos valores, el nombre del atributo y su respectivo
indicador.


\subsubsection*{Modelos}

La herramienta Nekonata est� orientada al aprendizaje supervisado
de predicci�n mediante funciones de regresi�n e incluye la t�cnica
adaptada de agrupamiento a trav�s de cluster como alternativa. Ambos
casos fueron dise�ados para incorporar los algoritmos que se deseen
asegurando el correcto uso de sus funciones.


\subsubsection*{Par�metros}

Los algoritmos de aprendizaje autom�tico se rigen bajo f�rmulas matem�ticas
que a menudo incluyen constantes o coeficientes cuyo valor incide
directamente en la calidad y desempe�o del clasificador. Algunos algoritmos
no tienen par�metros adicionales m�s que los atributos del dominio,
otros en cambio, tienen par�metros simples o complejos. Un par�metro
simple es aquel conformado por un �nico valor y un par�metro complejo
aquel constituido por una serie de valores, en la mayor�a de casos
se trata de algoritmos internos del algoritmo principal, tal es el
caso de la funci�n Kernel que utilizan los algoritmos de vectores
de soporte. 


\subsubsection*{Optimizaci�n}

Los algoritmos pueden aplicarse utilizando los valores por defecto
de los par�metros sin embargo pueden variarse conjuntamente para adaptarse
mejor a los datos de entrenamiento y producir modelos predictivos
m�s adecuados. Un proceso de optimizaci�n define rangos de valores
v�lidos para cada uno de los par�metros admitidos y ejecuta a trav�s
de un proceso evaluativo, la combinaci�n cruzada de cada opci�n, evaluando
cada alternativa posible. Como resultado, se obtiene la mejor de todas
las configuraciones para el algoritmo en cuesti�n.

La herramienta Nekonata considera la optimizaci�n de algoritmos de
regresi�n, clusterers y funciones kernel. 

Las consideraciones mencionadas anteriormente son el punto de partida
en el mecanismo de aprendizaje. El proceso de entrenamiento de modelos
(Figura \ref{fig:prediction-workflow}) b�sicamente lleva a cabo la
configuraci�n de todos los datos requeridos para ejecutar los algoritmos
clasificadores y obtener, posteriormente los modelos. Estos datos
est�n directamente afectados por la biblioteca de aprendizaje autom�tico
que se use por lo que se dispondr� de diferentes opciones de selecci�n
variando entre una librer�a y otra. Esta disposici�n dio lugar a un
flujo determinado en el orden de las actividades. 

\begin{figure}
\begin{centering}
\includegraphics[scale=0.55]{C:/Users/usuario/Tesisworkspace/Tesis_Standalone/tesis/images/prediction-workflow}
\par\end{centering}

\caption{Diagrama de flujo del proceso de entrenamiento de modelos.\label{fig:prediction-workflow}}
\end{figure}


La selecci�n de la librer�a de aprendizaje autom�tico es el punto
de partida condicionando el resto de los datos configurables. Luego,
al seleccionar el archivo fuente con los benchmarks medidos en formato
CSV, a trav�s de un proceso de parseo se obtiene el archivo com�nmente
llamado dataset adaptado al formato de instancias de la librer�a.
Con este nuevo formato se extraen los atributos para seleccionar aquel
que se pretenda predecir y junto con la selecci�n de algoritmos se
inicia la optimizaci�n de los mismos. 

Los algoritmos de aprendizaje se habilitan al momento de elegir la
librer�a a utilizar ofreciendo s�lo aquellos que la librer�a dispone.
La multi selecci�n de clasificadores brinda la posibilidad al usuario
de elegir los algoritmos de mayor preferencia o inter�s para optimizar
en ese momento evitando el proceso de optimizaci�n de todos los clasificadores
lo cual significar�a un ahorro en el uso de los recursos computacionales. 

Finalmente, como resultado de esta etapa se obtiene el conjunto de
clasificadores seleccionados cuyas configuraciones son las m�s favorables
para los datos de entrenamiento 


\subsubsection{Evaluaci�n de modelos}

En esta secci�n se detallar�n las consideraciones y los enfoques generales
del proceso de evaluaci�n y m�tricas de error. Tambi�n, se especificar�
el flujo de las actividades llevadas a cabo para calificar el desempe�o
del clasificador y realizar ajustes de ser necesario. 


\subsubsection*{Evaluaci�n}

El dataset de origen, tambi�n denominado conjunto de datos de aprendizaje
o entrenamiento, es utilizado para obtener el modelo predictivo que
generalice esos datos adecuadamente. El t�rmino generalizar hace referencia
a obtener una funci�n que se ajuste a los datos en cuanto minimice
el \emph{error emp�rico} que es el error producido por el algoritmo.
Se asume que los datos de entrenamiento constituyen una muestra lo
suficientemente representativa para aplicar un clasificador.  

El proceso de evaluaci�n, entonces, debe utilizar nuevos datos de
entrada (preferentemente distintos al conjunto de entrenamiento) y
predecir la salida a partir de �stos arrojando ciertamente una tasa
de error en la predicci�n hecha por el clasificador. Existen dos m�todos
cl�sicos de evaluaci�n en base al origen de los datos usados para
la validaci�n de modelos. El m�s sencillo resulta de utilizar los
mismos datos que fueron usados para la fase de entrenamiento. El m�todo
restante conocido bajo el nombre de validaci�n cruzada (\textquoteleft Cross
validation\textquoteright{} por su nombre en ingl�s) tiene una metodolog�a
m�s compleja. 

El m�todo de validaci�n cruzada utiliza un coeficiente \emph{K} de
repetici�n y divisi�n; los datos de entrada se dividen en \emph{K}
subconjuntos, utilizando uno de ellos como dato de prueba y el resto
(\emph{K-1}) como datos de entrenamiento. El proceso es repetido durante
\emph{K} iteraciones, con cada uno de los posibles subconjunto de
datos de prueba. Finalmente se calcula la media aritm�tica de los
resultados de cada iteraci�n para obtener un �nico resultado. Los
resultados de la evaluaci�n son contemplan mediante un conjunto de
indicadores que se describir�n m�s adelante. Nekonata implementa ambos
m�todos e incluye la funcionalidad necesaria para acceder a todos
los valores predictivos. 


\subsubsection*{M�tricas}

Las m�tricas usadas para la evaluaci�n de los algoritmos simplemente
son f�rmulas matem�ticas o composici�n de ellas que involucran �nicamente
a dos variables, los valores reales del dominio y los valores predictivos.
En la herramienta se ha implementado el conjunto de indicadores m�s
popular brindando, conjuntamente, la posibilidad de utilizar las m�tricas
ofrecidas por cada biblioteca incorporada. Tambi�n, se han definido
e incorporado cuatro caracter�sticas globales asociadas a las m�tricas
para brindar soporte a otras funcionalidades. Toda m�trica \emph{i})
brinda alg�n tipo de informaci�n, acerca del error de predicci�n o
estad�sticas o atributos de los datos; \emph{ii}) tiene una representaci�n
particular de los indicadores, los valores pueden estar normalizados
en una escala del 0 al 1, expresados en porcentajes o simplemente
adaptados a la escala de los datos; \emph{iii}) definen por su naturaleza
un requerimiento minimo o maximo de su indicador y \emph{iv}) son
aplicables a un tipo de funci�n ya sea regresi�n o clasificaci�n. 

Con el fin de extremar las mejoras y asegurar que el modelo resultante
sea el m�s favorable para el conjunto de datos usados en el entrenamiento
el proceso de evaluaci�n de modelos fue dividido en dos fases. Luego
de la etapa de entrenamiento, los algoritmos clasificadores optimizados
son expuestos a una serie de m�tricas dispuestos de manera tal que
permita visualizar las diferencias de desempe�o entre clasificadores
con la misma m�trica considerada. Esta vista es un recurso ofrecido
al usuario para decidir aquel modelo que a su criterio tenga mejor
calidad a trav�s de la comparaci�n simult�nea y continuar mejorando
el modelo elegido a partir del an�lisis de los efectos de underfit
y overfit. 


\subsubsection*{Fase 1: Comparaci�n de modelos}

La optimizaci�n de un algoritmo clasificador para obtener la configuraci�n
m�s apropiada para el conjunto de datos es conveniente,sin embargo,
resulta m�s �til aplicar un proceso de optimizaci�n a un conjunto
de algoritmos candidatos para luego analizar cu�l de ellos es el que
mejor generaliza los datos. La comparaci�n entre clasificadores se
realiza por medio de m�tricas que arrojan una estimaci�n acerca del
error de predicci�n, es decir, una medida que refleja la diferencia
entre los datos reales del dominio y los valores predichos por el
algoritmo. Es conveniente trasladar el an�lisis a la mayor cantidad
de m�tricas posibles, ya que un mismo indicador podr�a arrojar valores
cercanos entre un algoritmo y otro favoreciendo equivocadamente a
uno de ellos, error que podr�a notarse al compararlos simult�neamente
con otras m�tricas. La metodolog�a de operaci�n de esta fase se muestra
en la figura \ref{fig:models-comparison-workflow}. 

Nekonata hace foco en este detalle y ofrece al usuario una vista de
im�genes con los indicadores medidos de cada algoritmo seleccionado
por el usuario para aplicar el proceso de optimizaci�n y elegir, posteriormente,
el que resulte m�s adecuado. La informaci�n gr�fica complementaria
que se brinda, detalla los algoritmos representados por medio de barras
y agrupados en categor�as separadas de acuerdo a cada m�trica a modo
de facilitar la comparaci�n. Adicionalmente las barras se colorean
en dos tonos diferentes para acentuar, por cada m�trica, los clasificadores
que significar�an las mejores opciones para ese indicador. 

\begin{figure}
\begin{centering}
\includegraphics[scale=0.55]{C:/Users/usuario/Tesisworkspace/Tesis_Standalone/tesis/images/models-comparison-workflow}
\par\end{centering}

\caption{Diagrama de flujo de la fase de comparaci�n de modelos. \label{fig:models-comparison-workflow}}
\end{figure}


La disposici�n de todas las m�tricas de evaluaci�n se han distribuido
en im�genes dispares para agruparlas seg�n la clase de informaci�n
que representan, ya sean indicadores del error de predicci�n o caracter�sticas
sobre los datos. En el primer caso tambi�n se distinguen entre aquellas
m�tricas interpretadas a valores normalizados entre cero y uno y m�tricas
a valores de escala del atributo. De esta forma se agrupan en conjuntos
los indicadores proclives a compararse mutuamente como puede observarse
en la figura \ref{fig:screenshot-errors}. 

El aporte del usuario se incorpor� para personificar el inter�s y
criterio para determinar los indicadores m�s significativos para basar
la elecci�n del modelo m�s favorable. De esta manera, cada usuario
puede basar su elecci�n analizando y comparando los indicadores que
a su criterio son m�s relevantes. 

\begin{figure}
\begin{centering}
\includegraphics[scale=0.55]{C:/Users/usuario/Tesisworkspace/Tesis_Standalone/tesis/images/screenshot-errors}
\par\end{centering}

\caption{Captura de pantalla de la vista de indicadores sobre el error de predicci�n
normalizados.\label{fig:screenshot-errors}}
\end{figure}



\subsubsection*{Fase 2: Ajustes al modelo}

A trav�s del uso de m�tricas se adquiere una idea estimativa del desempe�o
del clasificador frente al conjunto de datos de entrenamiento, sin
embargo, podr�a resultar �til conocer el comportamiento general del
algoritmo, contrastando cada uno de los valores reales del atributo
clase con los valores predichos por el clasificador, y obtener as�
una vista exacta de la manera en que el clasificador se ajusta a los
datos (Figura \ref{fig:screenshot-error-curve}). 

Este recurso es usado en la herramienta como un gr�fico de dos l�neas
continuas de distinto color para representar el conjunto de datos
de entrenamiento y el conjunto de datos predichos. Cada punto del
dominio corresponde a cada instancia y la uni�n entre puntos s�lo
se realiza con fines ilustrativos. 

\begin{figure}
\begin{centering}
\includegraphics[scale=0.55]{C:/Users/usuario/Tesisworkspace/Tesis_Standalone/tesis/images/screenshot-error-curve}
\par\end{centering}

\caption{Captura de pantalla de la vista del error de predicci�n. \label{fig:screenshot-error-curve}}
\end{figure}


El an�lisis del error de predicci�n es la base para comprender la
calidad del modelo construido hasta el momento aunque no es el �nico
objetivo, ya que la vista permite extraer conocimiento de c�mo se
comporta el conjunto de datos de entrenamiento, existencia de valores
extremos, la variaci�n de los valores tomados, entre otros. Es un
recurso gr�fico para complementar la informaci�n que se le brinda
al usuario y encaminarlo a un correcto proceso de ajuste del modelo. 

\begin{figure}
\begin{centering}
\includegraphics[scale=0.55]{C:/Users/usuario/Tesisworkspace/Tesis_Standalone/tesis/images/adjustment-workflow}
\par\end{centering}

\caption{Diagrama de flujo de la fase de ajuste.\label{fig:adjustment-workflow}}
\end{figure}


El flujo de desarrollo de esta fase se plante� como un proceso de
car�cter opcional e iterativo para el usuario. Se puede optar simplemente
por almacenar el modelo elegido durante la etapa de comparaci�n anterior
o repetir las veces que se desee un procedimiento de an�lisis de las
curvas de aprendizaje del modelo y las acciones consecuentes para
reparar los posibles efectos, como se muestra en la figura \ref{fig:adjustment-workflow}.


A continuaci�n se explicar� el funcionamiento de la fase brindando
una visi�n general del planteamiento de las curvas de aprendizaje
con las respectivas consideraciones que se tuvieron en cuenta y la
manera en que se ha incluido la participaci�n del usuario para la
transformaci�n del conjunto de datos. 


\paragraph*{Curvas de aprendizaje}

Las curvas de aprendizaje se componen por dos l�neas de puntos representadas
en el mismo plano cartesiano. Ambas l�neas grafican el error de predicci�n
utilizando los dos m�todos de evaluaci�n conocidos, aplicando el conjunto
de entrenamiento o el m�todo de validaci�n cruzada.Las curvas de aprendizaje
implementadas por la herramienta son mostradas en la figura 4.15. 

El procedimiento considera una cantidad de instancias determinadas,
cantidad que se incrementa en un factor constante para aplicar el
modelo y calcular el error cuadr�tico medio, es decir, la diferencia
cuadr�tica entre el valor real del atributo y el predicho por el modelo. 

En un primer paso, la herramienta toma en cuenta las primeras cinco
instancias, luego las primeras diez, las primeras quince y as� sucesivamente
para conformar cada punto del dominio cuyo valor de ordenada es el
error cuadr�tico, error a�adido en la herramienta como parte del conjunto
de m�tricas consideradas. Por cuestiones de facilitar la vista se
impuso un l�mite en los valores del dominio reducido en trescientas
(300) instancias en caso de que el conjunto de entrenamiento supere
dicha cantidad. 




\paragraph*{Opciones para el usuario}

En Nekonata se han incluido dos acciones posibles para reparar, en
caso de evidenciar, los efectos de overfit o underfit del modelo.
Estas dos acciones no son mutuamente excluyentes de manera que el
usuario puede elegir libremente alguna o ambas acciones a la vez,
sin embargo, la herramienta le recomienda la acci�n que deber�a tomar
en caso de un efecto u otro. Ambas acciones realizan una transformaci�n
en la base de datos, cuando el usuario decide incorporar nuevas instancias
o ejemplos del dominio introduce un archivo el cu�l es parseado al
formato conocido por la librer�a de uso y es unido al conjunto original
siendo ahora, el nuevo conjunto de datos. Por otro lado, cuando el
usuario decide aumentar el grado del polinomio que emplea el modelo,
elige un n�mero entre uno y cuatro para modificar este polinomio y
as�, a�adir a la base de datos los nuevos atributos originarios del
nuevo polinomio completo para ese grado. 


\subsubsection{Dise�o e implementaci�n}

A pesar que la herramienta fue pensada para la predicci�n de componentes
de android,se desarroll� como una aplicaci�n de escritorio debido
a la limitaci�n del hardware de los dispositivos m�viles para ejecutar
los procesos de optimizaci�n que consumen un gran porcentaje de recursos
computacionales. Esta implementaci�n desvincula la herramienta de
medici�n con la herramienta de entrenamiento y evaluaci�n de modelos,
permitiendo la ejecuci�n independiente entre ambas por lo que el puente
de comunicaci�n entre ellas es la correspondencia entre la salida
de la primer herramienta con la entrada de la segunda; la herramienta
de medici�n crea archivos de formato \ac{CSV}, los cuales ser�n usados
en la herramienta como el conjunto de datos fuente para el entrenamiento. 


\subsubsection*{Entorno y tecnolog�as}

Para la implementaci�n del framework Nekonata se utiliz� el entorno
de desarrollo Eclipse cuyo lenguaje de programaci�n es Java. El proyecto
ha sido configurado con la versi�n \ac{JDK} 1.7 y \ac{JRE} 1.8 y
adem�s, se han incorporado tecnolog�as de terceros para el entorno
las cu�les cumplen distintos roles en la herramienta. 

Para el dise�o de la interfaz gr�fica se utiliz� mayormente la librer�a
SWING de java incorporada en el \ac{JDK} aunque tambi�n se incluyeron
componentes de la librer�a nativa AWT. La principal ventaja de la
librer�a SWING es brindar una interfaz adaptada a cada sistema operativo
sin necesidad de cambio de c�digo, es decir independiente a la plataforma.
Complementariamente se incorporaron las dos siguientes librer�as:
\emph{jgoodies-forms-1.8.0.jar} y \emph{miglayout15-swing.jar}

Por otro lado, para la creaci�n de vistas para el usuario se incorpor�
la biblioteca gr�fica de Java JFREECHART que facilita la creaci�n
de varios tipos de gr�ficos profesionales, en el caso de la herramienta
se han utilizado los gr�ficos de barras y lineales para representar
la informaci�n requerida por el usuario. Esta biblioteca ha sido elegida
por brindar objetos de alta calidad y ofrecer una extensa gama de
funciones para reforzar y mejorar la informaci�n mostrada en los gr�ficos.
Se utiliz� la versi�n 1.0.19 de la biblioteca en complemento con la
librer�a JCOMMON en la versi�n 1.0.23. 

Respecto al uso de t�cnicas de aprendizaje de m�quina, actualmente
existen muchas tecnolog�as y frameworks que proveen esta funcionalidad
para utilizar en entornos Java, sin embargo, como se ha adelantado
anteriormente en la herramienta s�lo se incluy� la biblioteca \emph{Weka}
en la versi�n 3.8 y se a�adi� tambi�n un paquete para la optimizaci�n
de par�metros perteneciente a la misma librer�a denominado MULTISEARCH
en su versi�n m�s actual del mes de agosto del 2016. 


\subsubsection*{Dise�o}

Considerando todo lo anteriormente descrito, el concepto principal
de la herramienta fue la extensibilidad de la misma desde todos los
enfoques posibles. Esto significa que la herramienta pudiera aceptar
diferentes tipos de librer�as de aprendizaje de m�quina, datasets,
modelos de regresi�n, par�metros y m�tricas; lo que conlleva a un
gran grado de abstracci�n de las clases que conforman la aplicaci�n
y, por lo tanto, se consider� importante la asociaci�n de mismas,
para un entendimiento mayor de parte de un nuevo desarrollador.

La finalidad de la etapa de implementaci�n fue un buen framework de
trabajo para el futuro desarrollo de modelos, librer�as y m�tricas.
Si bien la herramienta s�lo trabaja con Weka actualmente y la mayor�a
de su funcionalidad es enteramente parser , se considera que el grado
de abstracci�n es lo bastante elevado para poder soportar la finalidad
deseada.

Aplicando los conceptos te�ricos antes presentados, se exponen a continuaci�n
los paquetes y las clases principales que componen el desarrollo de
Nekonata. 


\paragraph*{Databases}

Como ya se mencion� anteriormente se desea poder administrar, leer
y escribir datasets de forma din�mica y segura. Para esto, se cre�
una estructura para almacenar cada individuo (instancia) con una estructura
\emph{Hashtable}, con los nombres de los atributos como claves de
los valores num�ricos que representan a cada uno de ellos.

Asimismo, se desarroll� un conjunto para representar los datasets
ya que son estos las bases de los modelos, del aprendizaje y del framework,
tomando como entrada un csv y creando la base de datos necesaria para
el correcto funcionamiento de la aplicaci�n. 

\begin{figure}
\begin{centering}
\includegraphics[scale=0.55]{C:/Users/usuario/Tesisworkspace/Tesis_Standalone/tesis/images/database-class-diagram}
\par\end{centering}

\caption{Diagrama de clases de las bases de datos implementadas\label{fig:database-class-diagram}}
\end{figure}



\paragraph*{Modelos}

Los modelos que la herramienta presenta son de tipo regresivo pero
no necesariamente son modelos intr�nsecamente predictivos y por eso
el framework presenta una divisi�n clara entre modelos regresivos
y clasificadores (siempre considerando que estos �ltimos se utilizan
para la regresi�n).

\begin{figure}
\begin{centering}
\includegraphics[scale=0.55]{C:/Users/usuario/Tesisworkspace/Tesis_Standalone/tesis/images/modelers-class-diagram}
\par\end{centering}

\caption{Diagrama de clases de los modelos base implementados.\label{fig:modelers-class-diagram}}
\end{figure}


A partir de esto, implementando los m�todos abstractos presentados,
se pueden incluir cualquier m�todo de clasificaci�n requerido, siempre
considerando importante que retorne un valor denso. Actualmente, los
modelos implementados son los presentados en la secci�n \ref{subsec:Funciones-contempladas}.

\begin{figure}
\begin{centering}
\includegraphics[scale=0.55]{C:/Users/usuario/Tesisworkspace/Tesis_Standalone/tesis/images/all-modelers-class-diagram}
\par\end{centering}

\caption{Diagrama de la relaci�n entre los modelos implementados.\label{fig:all-modelers-class-diagram}}
\end{figure}


Como se puede apreciar en la figura \ref{fig:all-modelers-class-diagram}
, se realiz� una abstracci�n intermedia considerando la librer�a Weka.
Esto se debe principalmente al comportamiento en com�n que ten�an
dichos modelos, pero no es mandatoria dicha abstracci�n al agregar
una nueva librer�a o al agregar nuevos modelos. 

Ya que el dise�o fue impulsado por la necesidad de abstracci�n e independencia
de las librer�as subyacentes, se considera importante la posibilidad
de que en un futuro cualquier desarrollador pueda incorporar modelos
implementados de forma particular. 


\paragraph{Par�metros}

Para implementar los modelos, la iniciativa fue modelar el concepto
te�rico que los rige: funciones. Como ya fue explicado, las funciones
est�n moldeados por variables y par�metros. Las variables son aquellos
atributos modelados por la Hashtable en la clase individuo. Debido
a los modelos considerados en el apartado anterior, los par�metros
(o las constantes de una funci�n) fueron modeladas en dos partes:
\begin{enumerate}
\item Par�metros simples: Son aquellos que s�lo tienen un valor num�rico
y que tiene un valor �nico en la funci�n. Rigen en funcionamiento
del m�todo de aprendizaje y determinan la calidad y finalidad que
tendr�n los mismos. Estos tipos de parametros se utilizan en todo
el core de modelos ya que son la idea fundamental de cualquier funci�n
matem�tica. 
\item Par�metros kernel: Son funciones matem�ticas que se emplean en las
\ac{SVM}. Estas funciones son las que le permiten convertir lo que
ser�a un problema de clasificaci�n no lineal en el espacio dimensional
original, a un sencillo problema de clasificaci�n lineal en un espacio
dimensional mayor. Debido a que estas son funciones dentro de los
modelos, es conveniente el modelado de las mismas de forma particular.
\end{enumerate}
Observando ambos puntos, se puede apreciar una composici�n de par�metros
ya que las funciones se conforman por par�metros simples y los par�metros
kernel son funciones internas. Esto se puede observar en la figura
\ref{fig:parameters-class-diagram} . 

\begin{figure}
\begin{centering}
\includegraphics[scale=0.55]{C:/Users/usuario/Tesisworkspace/Tesis_Standalone/tesis/images/parameters-class-diagram}
\par\end{centering}

\caption{Diagrama de clases de los parametros implementados\label{fig:parameters-class-diagram}}
\end{figure}



\paragraph{M�tricas}

Otra parte importante de la implementaci�n de la herramienta es la
capacidad de la misma de cuantificar y valorizar los modelos obtenidos.
Las mismas ya vienen implementadas por la librer�a Weka, pero considerando
la finalidad de abstraer comportamiento, las mismas fueron parseadas
y se crearon nuevas clases.

\begin{figure}
\begin{centering}
\includegraphics[scale=0.55]{C:/Users/usuario/Tesisworkspace/Tesis_Standalone/tesis/images/metrics-class-diagram}
\par\end{centering}

\caption{Diagrama de clases de las m�tricas implementadas.\label{fig:metrics-class-diagram}}
\end{figure}


La implementaci�n presentada surge a partir de la forma de implementaci�n
que tiene la librer�a weka. La misma presenta una clase que representa
la relaci�n entre los valores reales que presenta la base de datos
y los valores calculados por el modelo. Sin embargo, la librer�a presenta
la particularidad de solo calcular m�tricas de los modelos que la
misma toma como de regresi�n. As�, el Simple K clusterer quedar�a
excluido de este grupo y, por lo tanto, no podr�a ser valorado. La
implementaci�n planteada permite que este modelo quede a la altura
de los otros presentados y que, si se desea, se pueden crear nuevas
m�tricas sin tener la clase \emph{AbsEvaluation} que represente esta
relaci�n. S�lo ser�a necesario crear la m�trica heredando de \emph{AbsSimpleMetric}
e implementar la forma de c�lculo requerida.


\paragraph*{Optimizaci�n}

El paquete que se procede a explicar es implementado principalmente
considerando la funci�n que proviene de Weka, permitiendo la prueba
de varios valores para los par�metros sin necesidad de pruebas constantes,
apuntando a una mejor y m�s r�pida optimizaci�n del modelo. Puede
omitirse dicha implementaci�n si se desea agregar un nuevo modelo,
pero es conveniente la explicaci�n del mismo para futuras adaptaciones
de modelos de weka que se quieran agregar o de librer�as que tengan
esta posibilidad tambi�n. 

La idea del paquete de optimizaci�n es que, de forma transparente
para el usuario de la aplicaci�n, el modelo consiga adaptarse a la
base de datos analizada. Esto permite que la aplicaci�n Nekonata pueda
adaptarse a grandes rangos de valores objetivo. Como ya se dijo anteriormente,
no es un paquete necesario en la aplicaci�n ya que los par�metros
pueden ser puestos de forma fija en un modelo, pero esto restringe
el rango y las posibilidades del mismo. 

Los optimizadores cumplen la funci�n de probar valores para los par�metros
y se quedan con aquellos que minimizan el grado de error del modelo.
Asimismo, cabe destacar que Weka proporciona dos tipos de optimizadores.
Los primeros son los simples, que consideran cada par�metro del modelo
desde un valor al otro. Los segundos son los kernels y prueban valores
no solo para los par�metros propios de los modelos sino para los par�metros
que conforman los kernels. 

\begin{figure}
\begin{centering}
\includegraphics[scale=0.55]{C:/Users/usuario/Tesisworkspace/Tesis_Standalone/tesis/images/optimizer-class-diagram}
\par\end{centering}

\caption{Diagrama de clases de los optimizadores implementados.\label{fig:optimizer-class-diagram}}
\end{figure}



\paragraph*{C�mo agregar clases nuevas}

Todo lo anteriormente planteado se mantiene coherente y de forma congruente
gracias a la clase librer�a y lo que conlleva agregarla a la aplicaci�n. 

\begin{figure}
\begin{centering}
\includegraphics[scale=0.55]{C:/Users/usuario/Tesisworkspace/Tesis_Standalone/tesis/images/library-class-diagram}
\par\end{centering}

\caption{Diagrama de clases de las librerias y las relaciones con los otros
objetos.\label{fig:library-class-diagram}}
\end{figure}


Al crear la librer�a, y como se ve en el diagrama de clases, se deben
implementar dos m�todos que devuelven dos conceptos principales. Por
un lado, un objeto de tipo Database que ya se explic� anteriormente.
Por el otro, un objeto MetricsCollection. Este �ltimo representa el
conjunto de m�tricas que sirven para valorizar los modelos. 

Ya implementado se provee el conjunto \emph{SimpleMetricsCollection}
que permiten valorizar cualquier modelo, ademas del \emph{WekaMetricsCollection}
que s�lo contiene la metricas basadas en el \emph{WekaEvaluation}. 

La �ltima funci�n que se debe implementar de forma ineludible retorna
un vector de constantes. Estas constantes deben estar declaradas en
la clase \emph{Config.Modeler} con el formato: 

\fbox{\begin{minipage}[t]{1\columnwidth}%
\begin{center}
Nombre a mostrar en la aplicaci�n: constante 
\par\end{center}%
\end{minipage}}

Las mismas declaran los modelos que son aceptados por la liberia.
Esto se hace a modo de �ndice para los modelos y, al momento de utilizar
la herramienta, las clases de los modelos sean creadas al momento
de ser seleccionadas. Esto se permite implementando la �ltima funci�n: 

\lstinline[language=Java,basicstyle={\footnotesize}]!public abstract Vector<AbsModeler> createModelers(Vector<Integer> selectedModels, int index);!

Esta �ltima funci�n crea la relaci�n entre la constante con las clases
que se deben crear para representar a los modelos. 

\begin{figure}
\begin{centering}
\includegraphics[scale=0.55]{C:/Users/usuario/Tesisworkspace/Tesis_Standalone/tesis/images/config-correlation}
\par\end{centering}

\caption{Diagrama conceptual de la configuraci�n que presenta la herramienta\label{fig:config-correlation}}
\end{figure}

\acresetall%% LyX 2.1.4 created this file.  For more info, see http://www.lyx.org/.
%% Do not edit unless you really know what you are doing.
\documentclass[oneside,spanish]{book}
\usepackage[T1]{fontenc}
\usepackage[latin9]{inputenc}
\setcounter{secnumdepth}{3}
\setcounter{tocdepth}{3}
\usepackage{float}
\usepackage{graphicx}

\makeatletter
%%%%%%%%%%%%%%%%%%%%%%%%%%%%%% Textclass specific LaTeX commands.
\newenvironment{lyxlist}[1]
{\begin{list}{}
{\settowidth{\labelwidth}{#1}
 \setlength{\leftmargin}{\labelwidth}
 \addtolength{\leftmargin}{\labelsep}
 \renewcommand{\makelabel}[1]{##1\hfil}}}
{\end{list}}

\makeatother

\usepackage{babel}
\addto\shorthandsspanish{\spanishdeactivate{~<>}}

\begin{document}

\chapter{Evaluaci�n\label{chap:Evaluaci=0000F3n}}

En este cap�tulo se describe el procedimiento de evaluaci�n que se
lleva a cabo y una descripci�n completa sobre los escenarios considerados
y cada uno de los dataset involucrados. Tambi�n, se presentan los
resultados obtenidos junto con las inferencias y relaciones que se
han determinado entre las t�cnicas y el contexto en el que se aplican. 

El cap�tulo se organiza de la siguiente manera: en la secci�n \ref{sec:Metodolog=0000EDa-de-Evaluaci=0000F3n}
se detalla la metodolog�a de evaluaci�n propuesta para analizar la
eficacia de los modelos de predicci�n, en esta se incluye informaci�n
complementaria y particular que fue tomada en cuenta para llevar a
cabo la evaluaci�n mediante sub-secciones individuales, de esta manera
se especifican, las m�tricas usadas, el formato de los resultados
obtenidos, la determinaci�n empleada para la configuraci�n de los
par�metros m�s fundamentales de cada t�cnica y las caracter�sticas
de los dispositivos m�viles usados para las pruebas. 

En la secci�n \ref{sec:Escenarios} se presentan los escenarios de
estudio considerados, brindando una descripci�n detallada de los dataset
involucrados junto a gr�ficas y an�lisis sobre la distribuci�n de
los datos, con el fin de comprender mejor el desempe�o de las tecnicas
de regresion analizadas. Los modelos de predicci�n obtenidos son expuestos
en la secci�n \ref{sec:Resultados-y-discusi=0000F3n} desglosados
por escenario para analizar c�mo impacta en el desempe�o de las t�cnicas,
las particularidades de cada dominio. Finalmente en la secci�n \ref{sec:Conclusiones}
se obtienen las conclusiones del cap�tulo. 


\section{Metodolog�a de Evaluaci�n\label{sec:Metodolog=0000EDa-de-Evaluaci=0000F3n}}

Existen diversas formas que permiten evaluar la eficacia del clasificador.
La calidad de los modelos ser� evaluada a trav�s de seis m�tricas
de regresi�n que representan de forma distinta el error de predicci�n.
As� mismo, para un an�lisis m�s profundo sobre las distintas t�cnicas
y su desempe�o se usaron cuatro escenarios o contextos dispares entre
s� para determinar, de ser posible, las t�cnicas que mejor se adecuan
a un escenario (por su naturaleza), o por el contrario, no aplican
adecuadamente ante un escenario particular. 

Cada modelo es definido por una t�cnica \emph{X} (regresi�n lineal,
red neuronal, etc.) que se aplica para predecir una propiedad \emph{Y},
(tiempo de respuesta y precisi�n) de un componente \emph{Z}. Cada
uno de los contextos empleados incluyen tres o cuatro datasets, en
algunos casos para individualizar el an�lisis hacia los componentes
espec�ficos, en otros, para separar las pruebas realizadas en m�viles
distintos. 

En todo el conjunto de dataset, se a�adi� el tiempo de respuesta como
propiedad a predecir, y s�lo en aquellos donde el contexto lo permit�a
se incorpor� como propiedad la optimalidad o precisi�n en el resultado
arrojado por el componente. 


\subsection{M�tricas de evaluaci�n\label{subsec:M=0000E9tricas-de-evaluaci=0000F3n}}

Tras construir una serie de modelos de regresi�n diferentes, existe
una gran cantidad de criterios por los cuales pueden ser evaluados
y comparados. Todos los indicadores comparan los valores reales con
sus estimaciones, pero lo hacen de una manera ligeramente diferente. 

\emph{\ac{MAE}} representa el promedio del error absoluto (diferencia
entre los valores predichos y los observados), e indica cu�n grande
es el error que puede esperarse de la predicci�n. Al tratarse de una
m�trica basada en el error medio puede subestimar el impacto de errores
grandes pero infrecuentes. Si el an�lisis se centra demasiado en la
media arrojar� conclusiones precipitadas, por lo tanto para ajustar
errores grandes y raros, se calcula el error cuadr�tico medio (\ac{RMSE}).
Mediante la cuadratura de los errores en lugar de la media y luego
tomar la ra�z cuadrada de la media, se llega a una medida del tama�o
del error que da m�s peso a los errores grandes e infrecuentes que
la media. 

La comparaci�n entre las m�tricas \emph{\ac{RMSE} }y \emph{\ac{MAE}
}puede ser �til para determinar si un dataset contiene errores significativos
y poco frecuentes, cuanto mayor sea la diferencia entre ambos indicadores,
m�s inconsistente es el tama�o del error.

El coeficiente de correlaci�n (\emph{CC}) es analizado en conjunto
con el indicador \emph{\ac{RMSE}} ya que existe una estrecha relaci�n
entre ambos. Por ejemplo, si \emph{\ac{CC}} es 1, \emph{\ac{RMSE}}
debe ser 0, ya que todos los puntos se encuentran en la l�nea de la
funci�n de regresi�n; cuanto m�s cercano sea el valor de \emph{\ac{CC}}
a 1 o -1, m�s pr�ximos son los valores observados a la l�nea de predicci�n,
y por tanto menor ser� el error absoluto reflejado en \emph{\ac{RMSE}}. 

Por otro lado, complementariamente se utilizan indicadores relativos
en lugar de absolutos como los descriptos anteriormente, ya que ponderan
el error de predicci�n respecto a la variaci�n est�ndar de las observaciones.
De esta manera se obtienen indicadores \emph{\ac{RAE}} y \emph{\ac{RRSE}}
de valores entre 0 y 1 para obtener una visi�n de las observaciones
respecto a la media. 


\subsection{Formato de los resultados\label{subsec:Formato-de-los}}

La herramienta devuelve como resultado un archivo en formato TXT almacenando,
por un lado, los valores resultantes de los par�metros fundamentales
de cada t�cnica y, por otro lado, la descripci�n del modelo construido.
La herramienta genera un archivo por cada t�cnica elegida por el usuario
para optimizar, en la direcci�n especificada en la configuraci�n bajo
el nombre de la t�cnica. La figura \ref{fig:linear-regression-result}
permite una descripci�n gr�fica de lo expresado anteriormente, donde
detalla A) los par�metros, B) Nombre del modelo y C) la funci�n definida
para el atributo de predicci�n. 

\begin{figure}[H]
\begin{centering}
\includegraphics[scale=0.55]{C:/Users/usuario/Tesisworkspace/Tesis_Standalone/tesis/images/linear-regression-result}
\par\end{centering}

\caption{Ejemplo de resultado del modelo \textquoteleft Linear Regression\textquoteright .
\label{fig:linear-regression-result}}
\end{figure}


La manera en que los datos son expuestos al usuario no representa
un formato compatible para cualquier herramienta sobre aprendizaje
de m�quina, esto se debe a que no existe actualmente, un formato est�ndar
para almacenar los modelos, y esta discrepancia entre las herramientas
disponibles exige un seteo manual de las t�cnicas por cada evaluaci�n
que se realice


\subsection{Par�metros de configuraci�n de las t�cnicas\label{subsec:Par=0000E1metros-de-configuraci=0000F3nde}}

La optimizaci�n y posterior obtenci�n de un modelo de calidad tiene
lugar a partir del establecimiento de diversos par�metros de configuraci�n. 

Estos par�metros pueden requerir distintos valores adapt�ndose a condiciones
particulares de los datos que optimicen el resultado de la predicci�n.
Para brindar m�s detalle de estos par�metros, el Cuadro \ref{tab:Par=0000E1metros-de-configuraci=0000F3n}
presenta la lista completa de par�metros propios de cada t�cnica,
el valor por defecto adoptado para cada una y el rango de valores
establecido con el cual se determinar� el mejor valor resultante. 

\begin{table}[H]
\begin{centering}
\includegraphics[scale=0.5]{C:/Users/usuario/Tesisworkspace/Tesis_Standalone/tesis/images/configuration-parameters}
\par\end{centering}

\caption{Par�metros de configuraci�n de las t�cnicas de regresi�n \label{tab:Par=0000E1metros-de-configuraci=0000F3n}}
\end{table}


El proceso de optimizaci�n de las t�cnicas utiliza el rango de configuraci�n
establecido para iterar en pasos tomando valores intermedios y evaluando
la calidad de la t�cnica con tales configuraciones, finalmente se
contrastan entre s� y se determina la mejor configuraci�n para un
dataset y atributo a predecir particular. 

Los rangos para cada t�cnica se establecieron tras un an�lisis exhaustivo
de la influencia de estos valores sobre los datos. Los rangos de prueba
se definieron de manera aleatoria en conjunto de fundamentos te�ricos
asociados a cada t�cnica, al mismo tiempo que el an�lisis de un rango
ya propuesto serv�a como base para definir uno nuevo. Los an�lisis
se llevaron a cabo sobre 20 archivos dataset dispuestos para esta
tesis. 

Por otro lado, los valores elegidos por defecto fueron tomados de
los valores propuestos por la herramienta Weka, a excepci�n del par�metro
\emph{Training Time} de la t�cnica \emph{MultiLayer Perceptron} que
fue reducido de 500 a 200 para mejorar la performance de tiempo del
algoritmo. 

Para el caso de la t�cnica \emph{Support Vector Machine} la configuraci�n
es m�s compleja ya que involucra par�metros propios simples y un kernel
con par�metros espec�ficos. Para la t�cnica \emph{\ac{SMO}}, el par�metro
\emph{gamma} y \emph{complexy} fueron analizados en conjunto con el
apoyo de fundamentos teor�a y emp�ricas. La figura \ref{fig:RBF-parameters-SVM}
refleja el impacto que tienen los par�metros sobre la clasificaci�n
de los datos. 

\begin{figure}[H]
\begin{centering}
\includegraphics[scale=0.55]{C:/Users/usuario/Tesisworkspace/Tesis_Standalone/tesis/images/RBF-parameters-SVM}
\par\end{centering}

\caption{Efecto de los par�metros de la t�cnica SVM. \label{fig:RBF-parameters-SVM}}
\end{figure}


Las mejores opciones para configurar los par�metros se obtienen observando
la diagonal en donde los valores se incrementan a la par, la primer
imagen de la cuadr�cula aplica una clasificaci�n muy simple sobre
los datos, dejando en evidencia un alto grado de error en las predicciones.
La �ltima imagen, en cambio, muestra un claro ejemplo de \emph{overfit},
por lo que no resulta ser una buena opci�n a la hora de buscar generalizaci�n.
La imagen central resulta entonces, ser la m�s apropiada para definir
los valores por defecto para estos par�metros. Por otro lado, la optimizaci�n
para el par�metro \emph{complexy} se determin� bajo el rango 0 - 25
debido a pruebas emp�ricas realizadas sobre el conjunto base de dataset. 

Respecto a la optimizaci�n del kernel, los cuatro algoritmos considerados
fueron analizados mediante la t�cnica prueba y error y de esa forma,
se determinaron los valores por defecto para los par�metros propios
de cada uno. Del an�lisis se desprenden algunas conclusiones consecuentes: 
\begin{description}
\item [{Normalized~Polynomial}] El valor por defecto para el par�metro
\emph{exponent} es 2.0 y hasta valores de 5.0 los resultados esperados
son buenos. 
\item [{Poly~Kernel}] El valor por defecto para el par�metro \emph{exponent}
es 1.0. Si el valor se incrementa se obtienen mejores resultados,
sin embargo para valores mayores a 5.0 el error de predicci�n comienza
a aumentar gradualmente. 
\item [{Puk}] El valor por defecto para el par�metro \emph{omega} es 1.0,
la modificaci�n de este valor es indiferente a los resultados por
lo que no aplica a la optimizaci�n. Respecto a \emph{sigma} se adopta
el valor 0.01 ya que el incremento en este valor no produce buenos
resultados. 
\item [{RBF}] Partiendo de un valor \emph{gamma} de 0.01 se obtienen buenos
resultados, e incluso incrementando este valor a raz�n de 0.01 los
resultados tienden a mejorar. 
\end{description}

\subsection{Caracter�sticas de los dispositivos m�viles usados\label{subsec:Caracter=0000EDsticas-de-los}}

En la mayor�a de las pruebas se han utilizado dos dispositivos m�viles
fundamentales, el modelo K3 Note de la marca Lenovo y el modelo Galaxy
S3 de la marca Samsung principalmente por las caracter�sticas del
procesador y memoria interna. El m�vil lenovo es superior debido a
sus 8 n�cleos de procesamiento frente a los 4 n�cleos del Samsung
y a una memoria RAM con el doble de capacidad. 

Tambi�n se utilizaron tres modelos m�s s�lo para el escenario del
problema del viajante. Todas las especificaciones son expuestas en
el cuadro \ref{tab:Especificaciones-de-los}. 

\begin{table}[H]
\begin{centering}
\includegraphics{C:/Users/usuario/Tesisworkspace/Tesis_Standalone/tesis/images/used-mobiles}
\par\end{centering}

\caption{Especificaciones de los dispositivos m�viles utilizados. \label{tab:Especificaciones-de-los}}
\end{table}



\section{Escenarios\label{sec:Escenarios}}

Definidas las m�tricas, la metodolog�a de evaluaci�n y los par�metros
de configuraci�n, en la presente secci�n se detallar�n los escenarios
o dominios estudiados a partir de una descripci�n detallada de los
dataset formados, cada uno de los componentes involucrados y los atributos
de entrada considerados formando un contexto particular para la predicci�n.
Por otro lado, se analizar� el comportamiento o formato que adquieren
los datos sobre los atributos predichos para comprender el desempe�o
de las diferentes t�cnicas sobre los mismos y la manera en que �stos
adecuan sus predicciones. 


\subsection{Escenario 1: Algoritmos para el problema del viajante\label{subsec:Escenario-1:-Algoritmos}}

En el campo de la teor�a de complejidad computacional, el problema
del viajante o TSP por sus siglas en ingl�s, es tratado como un problema
NP-Completo y es modelado como un grafo de manera que las ciudades
son sus v�rtices, los caminos son las aristas y las distancias entre
caminos son los pesos de las aristas. Es un problema de minimizaci�n
tras la b�squeda de un recorrido completo que comienza y finaliza
en un v�rtice espec�fico y visita el resto de los v�rtices exactamente
una vez con coste m�nimo. Existen muchas variantes del problema, una
de ellas se trata del TSP sim�trico en el cual la distancia entre
un par de ciudades es la misma en cada direcci�n formando un grafo
ponderado no dirigido. Esta variante fue la versi�n considerada por
la herramienta. 

Con el fin de garantizar instancias variadas del problema, (diferentes
cantidades de ciudades, distintas representaciones y pesos), se incorpor�
a la herramienta una librer�a llamada TSPLib con m�ltiples archivos
de instancias del problema. La librer�a cuenta con 111 archivos de
formato TSP, y 32 archivos de formato OPT.TOUR que contienen el recorrido
�ptimo del archivo que referencian. Los ejemplos oscilan desde 14
y hasta 18512 ciudades con un s�lo ejemplo extremo de 85900, y utilizan
cuatro funciones de c�lculo de la distancia entre ciudades: distancia
euclidiana, distancia geom�trica, distancia pseudo euclidiana y funci�n
techo. Para la obtenci�n de las mediciones correspondientes, se realiz�
una preselecci�n de 32 archivos que comprenden ejemplos de entre 22
y 200 ciudades. Cabe destacar que fue necesaria la implementaci�n
de un parser para el uso de los archivos.

Los dataset para este dominio han sido formados tras la ejecuci�n
de seis algoritmos de resoluci�n sobre 32 entradas del problema pertenecientes
a la librer�a TSPLIB dando lugar a un total de 192 instancias por
dataset; las pruebas fueron llevadas a cabo sobre el dispositivo m�vil
Samsung Galaxy S4, y repetidas en los dispositivos S5 y S6, obteniendo
as�, tres dataset an�logos que incluyen como propiedades 7 atributos,
Id de operaci�n, nombre de componente (algoritmo), entrada del problema
(nombre de archivo), exactitud y precisi�n en el resultado, tiempo
de respuesta y valor soluci�n. Todos los dataset han sido almacenados
en archivos con formato \ac{CSV} bajo el nombre \textquoteleft 'Sx-results\textquoteright{}
seg�n el modelo de dispositivo utilizado. Los componentes involucrados
incluyen el algoritmo del vecino m�s cercano, Programaci�n Lineal,
Mejor ajuste, Kruskal, Prim, y el algoritmo de Transformaci�n Local.
El algoritmo de Backtracking fue descartado de la lista debido al
gran consumo de memoria en los dispositivos imposibilitando el desarrollo
de las pruebas. 

En cuanto a la precisi�n y exactitud de las respuestas se consideraron
tres indicadores que toman en cuenta las repeticiones en la respuesta
y la totalidad o falta de los v�rtices incluidos. El indicador TP
(\emph{true positive}), para los v�rtices incluidos en la soluci�n,
FP(\emph{false positive}), para las repeticiones de v�rtices y FN
(\emph{False Negative}) para los v�rtices no incorporados en la soluci�n.
Por lo tanto, se consider�: 

\[
Precisi\acute{o}n=\frac{TP}{FP+TP}
\]


\[
Exactitud=\frac{TP}{FP+TP+FN}
\]


A modo de ejemplo, se presenta en la figura \ref{fig:TPS-graph-problem}
un grafo con comienzo en la ciudad A. Suponiendo que un algoritmo
de resoluci�n arroja el siguiente camino soluci�n: \{A, B, D, A\},
se obtiene una precisi�n del 100\% al tener un valor de \ac{TP} =
3 y un valor de \ac{FP} = 0 ya que no hay v�rtices repetidos en la
soluci�n, sin embargo, la soluci�n no es correcta ya que no incluye
la totalidad de v�rtices del conjunto (excluye al v�rtice C). En cambio,
al calcular la exactitud se obtiene un valor del 66\% al tener un
valor de \ac{TP} = 4, \ac{FP} = 1 y \ac{FN} = 1 brindando un concepto
m�s realista. 

El uso del indicador de precisi�n o el de exactitud depender� del
contexto en el que se aplique. 

\begin{figure}[H]
\begin{centering}
\includegraphics[scale=0.55]{C:/Users/usuario/Tesisworkspace/Tesis_Standalone/tesis/images/TPS-graph-problem}
\par\end{centering}

\caption{Ejemplo de grafo para el problema del viajante.\label{fig:TPS-graph-problem}}
\end{figure}


El objetivo de este escenario es la creaci�n de modelos capaces de
predecir el tiempo de latencia de cada algoritmo de resoluci�n frente
a distintas caracter�sticas de las entradas, y modelos para predecir
la precisi�n en los caminos retornados. En la figura \ref{fig:TPS-dataset-behaviour}
expuesta a continuaci�n se muestra el comportamiento del conjunto
de datos de entrenamiento ejecutado en el dispositivo Samsung Galaxy
S4; ambos dataset restantes no son expuestos ya que presentan m�nimas
diferencias. 

\begin{figure}[H]
\begin{centering}
\includegraphics[scale=0.55]{C:/Users/usuario/Tesisworkspace/Tesis_Standalone/tesis/images/TPS-dataset-behaviour}
\par\end{centering}

\caption{Comportamiento del dataset para el escenario del problema del viajante.\label{fig:TPS-dataset-behaviour}}
\end{figure}


A primera vista la figura \ref{fig:TPS-dataset-behaviour} permite
observar una gran concentraci�n de datos sobre los extremos para los
valores de precisi�n y tiempo de respuesta. En el primer caso, aproximadamente
un 94\% de los datos toman el valor 1, y en el segundo caso, un 88\%
de los datos toman valores entre 0 y 6.77 segundos. Este comportamiento
uniforme y poco distribuido de los datos obstaculiza el desarrollo
de un buen modelo predictivo. 

Respecto al atributo \emph{accuracy}, si bien presenta m�s dispersi�n
entre los datos, aproximadamente un 82\% circunda alrededor del valor
1 y un 15\% sobre valores cercanos a 0.008. 


\subsection{Escenario 2: Servicios para detecci�n de rostros\label{subsec:Escenario-2:-Servicios}}

Hoy en d�a muchas soluciones para detecci�n de rostro se proporcionan
como servicios Web, que aunque no son tan r�pidos y usables sin conexi�n
a Internet como las bibliotecas de software, resultan generalmente
m�s precisos y proporcionan caracter�sticas muy peculiares del rostro.
Sin embargo, el rendimiento y la precisi�n de los servicios var�an
seg�n las propiedades de la imagen (tama�o, foco, oclusiones) y contexto
de ejecuci�n (capacidad de la \ac{CPU}, conexi�n de red). Por lo
tanto el objetivo de este escenario es crear modelos para predecir
el tiempo de respuesta a partir de caracter�sticas variables de la
imagen de entrada y de los par�metros de configuraci�n del componente
en cuesti�n.

Los dataset para este dominio han sido formados tras la ejecuci�n
de diferentes componentes para detecci�n de rostros en im�genes en
un �nico dispositivo m�vil (Lenovo K3 Note), por tal motivo no se
incluyen en los archivos caracter�sticas propias del m�vil ya que
significan valores constantes en la predicci�n. Los componentes involucrados
incluyen un servicio o proceso Android (Google Play Services (GMS)
face detector) y tres servicios Web (FaceRect API, Sky Biometry API
y Microsoft Face API). Estos cuatro componentes ser�n evaluados a
partir de subconjunto de 290 im�genes pertenecientes al dataset llamado
FDDB \cite{fddbTechReport} con caracter�sticas muy variadas. Cada
servicio es llamado especificando una imagen y un conjunto de par�metros
booleanos (a trav�s del valor 0 y 1) que determinan las funciones
requeridas del algoritmo determinado. En los cuatro servicios, el
tiempo de respuesta es una variable continua y registrada en milisegundos. 

A continuaci�n se detallan las principales caracter�sticas de los
servicios y los respectivos dataset creados para cada uno:


\paragraph{GMS}

La compa��a Google ha implementado una colecci�n extensa y variada
de aplicaciones para Android. Entre ellas, ofrece el paquete \textquoteleft com.google.android.gms.vision\textquoteright{}
el cual proporciona una funcionalidad com�n para trabajar con detectores
de objetos visuales. En la figura \ref{fig:GooglePlay-services}se
muestra la forma en que el objeto \emph{GoogleAPIClient} proporciona
una interfaz para conectar y hacer llamadas a cualquiera de los servicios
de Google Play disponibles tales como Google Play Games, Google Drive,
Google Maps, etc 

\begin{figure}[H]
\begin{centering}
\includegraphics[scale=0.55]{C:/Users/usuario/Tesisworkspace/Tesis_Standalone/tesis/images/GooglePlay-services}
\par\end{centering}

\caption{Esquema conceptual para el uso de servicios de Google Play Services.\label{fig:GooglePlay-services}}


\end{figure}


El algoritmo permite setear tres par�metros diferentes (\emph{allLandmarks}
, \emph{allClassifications} , \emph{accurateMode}) dando lugar a 8
configuraciones posibles por imagen. Para el servicio GMS se utilizaron
290 im�genes originando un dataset de 2320 instancias. 

Archivo Dataset: \emph{ResponseTime - Google Play Service face detector.csv}.


\paragraph{FaceRect API y Sky Biometry API}

Los servicios web \emph{SkyBiometry} y \emph{FaceRect} son consumidos
directamente desde la aplicaci�n online \emph{Mashape}, la cual ofrece
una gran variedad de aplicaciones, incluida una colecci�n de Aplicaciones
sobre detecci�n y reconocimiento de rostros. SkyBiometry y FaceRect
son aplicaciones que utilizan interfaz REST, es decir, los m�todos
son llamados a trav�s de internet usando los m�todos \ac{HTTP} standard
como GET y POST a las direcciones correspondientes. Dependiendo de
los par�metros especificados en el request, el servidor puede generar
la respuesta tanto en formato JSON como \ac{XML}. Adicionalmente,
se utiliz� el servicio GMS Visi�n de Google a partir de la librer�a
correspondiente. Los tres servicios son descritos en detalle en el
ap�ndice B.

El algoritmo de la aplicaci�n FaceRect permite setear s�lo una opci�n
(\emph{features}) por imagen, de modo que se construy� un dataset
con s�lo dos propiedades, incluyendo el tiempo de respuesta y un total
de 54 instancias. 

Archivo Dataset: \emph{ResponseTime - FaceRect API.csv}. 

Por otro lado, el algoritmo de Sky Biometry permite setear siete par�metros
diferentes dando lugar a 128 configuraciones posibles por imagen (\emph{aggressive},
\emph{gender}, \emph{glasses}, \emph{smiling}, \emph{mood}, \emph{age},\emph{
eyes}). Para el servicio Sky Biometry se utilizaron 30 im�genes originando
un dataset de 3840 instancias y se incluyen 8 propiedades, adem�s
de los par�metros configurables del tipo binario se a�ade el tiempo
de respuesta. 

Archivo Dataset: \emph{ResponseTime - SkyBiometry API.csv}. 


\paragraph{Microsoft Face API}

Face API es un servicio web en la nube que provee los algoritmos de
detecci�n y reconocimiento de rostros m�s avanzados. El servicio es
consumido a trav�s del sitio de Microsoft y las respuestas son retornadas
en formato JSON. El algoritmo permite setear seis propiedades (\emph{landmarks},
\emph{age}, \emph{gender}, \emph{facialHair}, \emph{smile} y \emph{headPos}).
El dataset est� formado por 1826 instancias y siete propiedades, adem�s
de los par�metros configurables como atributos binarios se incluye
el tiempo de respuesta. 

Archivo Dataset: \emph{ResponseTime - Microsoft Face API.csv}. 

En la figura \ref{fig:face-services-dataset-behaviour} que se muestra
a continuaci�n se expone el comportamiento de los datos del atributo
\emph{Tiempo de respuesta} para cada uno de los servicios, en referencia
con la imagen A) FaceRect API, B) Microsoft Face API, C) Google Play
Service y D) Sky Biometry API. 

\begin{figure}[H]
\begin{centering}
\includegraphics[scale=0.55]{C:/Users/usuario/Tesisworkspace/Tesis_Standalone/tesis/images/face-services-dataset-behaviour}
\par\end{centering}

\caption{Comportamiento del atributo \textquoteleft Tiempo de respuesta\textquoteright{}
de los servicios de detecci�n de rostros.\label{fig:face-services-dataset-behaviour}}
\end{figure}


El servicio Sky Biometry presenta entre sus datos un 5\% de valores
infrecuentes y en esos casos valores extremos, siendo el �nico servicio
en presentar estas caracter�sticas. 

El servicio de Microsoft Face arroj� valores de tiempo de respuesta
uniformemente distribuidos en el intervalo tomando la forma de campana
de Gauss, los datos se distribuyen sim�tricamente sobre el intervalo,
es decir, no hay presencia de sesgos hacia la izquierda o hacia la
derecha. Google Play tambi�n tiene una distribuci�n de Gauss con sesgo
hacia la izquierda. Puede observarse que presenta un intervalo muy
amplio, de modo que las propiedades configurables del algoritmo incrementan
considerablemente el tiempo de operaci�n. Sin embargo, al contrastarlo
con el servicio de FaceRect, el m�nimo de tiempo requerido por este
servicio es mucho mayor que el tiempo m�ximo arrojado por el de Google
Play, a�n as� cualquier an�lisis sobre el servicio FaceRect puede
ser precipitado, ya que se evalu� con un bajo n�mero de instancias. 


\subsection{Escenario 3: Problema de la mochila\label{subsec:Escenario-3:-Problema}}

Al igual que el problema del agente viajero, el problema de la mochila
o \textquoteleft Knapsack problem\textquoteright{} por su nombre en
ingl�s, es un problema NP-Completo de optimizaci�n combinatoria, es
decir, se busca la mejor soluci�n entre un conjunto finito de posibles
soluciones al problema. El problema simula la colocaci�n de �tems
u objetos en una mochila de tal forma que se maximice el valor de
los �tems que contiene con la restricci�n de no superar el peso (o
volumen) m�ximo que puede soportar la mochila.

La evaluaci�n de este dominio se llev� a cabo en dos modelos distintos
de dispositivos m�viles (Samsung Galaxy S3 y Lenovo K3 Note) cuyo
objetivo es crear modelos de predicci�n del tiempo de respuesta y
la optimalidad del componente, es decir, la relaci�n entre la soluci�n
encontrada y la soluci�n �ptima retornando un valor entre cero y uno.
Para este dominio se implementaron dos algoritmos de resoluci�n al
problema como componentes, el algoritmo greedy de complejidad O(n2
log(n)) y el algoritmo backtracking que encuentra la soluci�n �ptima
pero en tiempo exponencial O(2n). Ambos componentes fueron implementados
puramente en Java. Las instancias del problema se obtuvieron aleatoriamente
de las cuales se consideraron los siguientes atributos: 
\begin{lyxlist}{00.00.0000}
\item [{Primero~ID~de~Operaci�n.}]~

\begin{lyxlist}{00.00.0000}
\item [{'KnapsackWeightRatio\textquoteright :}] teniendo en cuenta W como
el peso l�mite de la mochila y bi los pesos individuales de cada �tem,
se calcula bajo la f�rmula $KWR=W/b_{1}+b_{2}+...+b_{n}$ arrojando
valores entre 0 y 1
\item [{N�mero}] de �tems en la mochila denotado como N. 
\item [{Tiempo}] de respuesta registrado en segundos. 
\end{lyxlist}
\item [{Segundo~Nombre~de~Componente.}]~

\begin{lyxlist}{00.00.0000}
\item [{Optimalidad}] o precisi�n de los componentes siempre medida respecto
a la soluci�n brindada por el algoritmo Backtracking. 
\item [{Valor}] soluci�n al problema. 
\end{lyxlist}
\end{lyxlist}
Respecto a las caracter�sticas propias del dispositivo de ejecuci�n,
s�lo se tom� en cuenta la frecuencia de CPU ya que ambos algoritmos
son single threads y corren en CPU, con la suposici�n que la frecuencia
de CPU incide sobre la predicci�n. 

Se formaron cuatro dataset para este escenario, dos por cada uno de
los dispositivos m�viles evaluados. Bajo el nombre \emph{Optimality
and Response time - dispositivo XXX} se crearon datasets de 720 instancias
(360 para cada uno de los componentes) y siete propiedades como se
describieron anteriormente (Grupo A y B). Las entradas del problema
fueron consideradas desde un m�nimo de 5 hasta un tama�o m�ximo de
24 �tems l�mite impuesto debido a la complejidad del algoritmo backtracking
y su correspondiente latencia exponencial. Por cada �tem se evaluaron
18 entradas al problema generadas aleatoriamente a partir de cuatro
valores: cantidad total de �tems, m�ximo valor y peso de un �tem,
y finalmente el peso o capacidad de la mochila. 

Por otro lado, bajo el nombre \emph{Response time - Greedy algorithm
- dispositivo XXX}, se crearon datasets de 4704 instancias y cuatro
propiedades (Grupo A) con entradas del problema de tama�o m�nimo de
5 �tems, e incrementalmente hasta un m�ximo de 200, con 24 entradas
dedicadas a cada n�mero de �tem generadas tambi�n aleatoriamente.
A continuaci�n en lafigura \ref{fig:Knapsack-dataset-behaviour} se
muestra el comportamiento de los datos del atributo \emph{Tiempo de
respuesta}, en referencia a la imagen, A) \textquoteleft Opt \& Resp
Lenovo K3\textquoteright , B) \textquoteleft Opt \& Resp Samsung Galaxy
S3\textquoteright , C) Greedy Lenovo K3 y D) Greedy Samsung S3. 

\begin{figure}[H]
\begin{centering}
\includegraphics[scale=0.55]{C:/Users/usuario/Tesisworkspace/Tesis_Standalone/tesis/images/Knapsack-dataset-behaviour}
\par\end{centering}

\caption{Comportamiento del atributo \textquoteleft Tiempo de respuesta\textquoteright{}
para el problema de la mochila. \label{fig:Knapsack-dataset-behaviour}}
\end{figure}


En la ejecuci�n de los algoritmos de Backtracking y Greedy en ambos
dispositivos (Imagen A y B) se observa una concentraci�n aproximada
del 93\% de los datos en un intervalo muy peque�o y un 5\% de los
datos se encuentran en el intervalo mayor a 1 segundo, por lo cual
podr�a concluirse que son valores at�picos o poco frecuentes. Este
comportamiento muy homog�neo de los datos, dificultar� la construcci�n
de un modelo predictivo de alta calidad. Por otro lado, los dataset
que incluyen al algoritmo Greedy, presentan m�s dispersi�n de los
datos cuya distribuci�n toma forma exponencial (Imagen C) y forma
de campana de Gauss con sesgo hacia la izquierda (Imagen D). La heterogeneidad
en los datos puede posibilitar la construcci�n de un buen modelo.
Respecto al an�lisis de los valores registrados, se puede observar
que las mismas pruebas ejecutadas en el dispositivo Samsung Galaxy
S3 consumen notoriamente m�s tiempo de operaci�n.

Por otro lado, se expone tambi�n en la figura \ref{fig:Knapsack-optimi-dataset-behaviour}
el comportamiento para el atributo optimalidad de los dataset que
incluyen ambos algoritmos. (Backtracking y Greedy). 

\begin{figure}[H]
\begin{centering}
\includegraphics[scale=0.55]{C:/Users/usuario/Tesisworkspace/Tesis_Standalone/tesis/images/Knapsack-optimi-dataset-behaviour}
\par\end{centering}

\caption{Comportamiento del atributo \textquoteleft Optimalidad\textquoteright{}
para el problema de la mochila \label{fig:Knapsack-optimi-dataset-behaviour}}
\end{figure}


Las pruebas ejecutadas en el modelo K3 de Lenovo arrojaron valores
de optimalidad muy altos, un 80\% de los datos con optimalidad del
100\%, y menos del 5\% del total adquiere valores por debajo de 0.95.
Esta cifra incluye valores extremos que representan el 1\% de la totalidad
de observaciones. Por lo tanto, la soluci�n brindada por el algoritmo
Greedy es generalmente �ptima. Por otro lado, las pruebas ejecutadas
en el dispositivo Samsung Galaxy arrojaron un grado de optimalidad
del 69\%, y s�lo un 12\% de los datos corresponden a valores entre
0.9 y 1.0, sin incluir este �ltimo. Se presume que el dataset presenta
algunas anomal�as ya que un 10\% de los datos tienen valores superiores
a 1.0, contraponiendo el concepto de optimalidad. 


\subsection{Escenario 4: Multiplicaci�n de matrices\label{subsec:Escenario-4:-Multiplicaci=0000F3n}}

El producto de matrices corresponde a problemas de clase P ya que
a diferencia de los problemas NP, su complejidad es polin�mica. El
producto entre matrices no es conmutativo, depende del orden de las
matrices intervinientes y su multiplicaci�n s�lo es posible si el
n�mero de filas de la primera matriz es igual al n�mero de columnas
de la segunda. 

Los dataset para este dominio han sido formados tras la ejecuci�n
de diferentes algoritmos de multiplicaci�n de matrices, con tama�os
variados de matrices sobre 2 modelos de dispositivos m�viles, Lenovo
K3 y Samsung Galaxy S3. Lo interesante de este dataset, es que los
algoritmos fueron implementados con diferentes librer�as y dise�ados
para ser ejecutados por diferente elementos de hardware (\ac{CPU},
\ac{GPU}, etc). 

Los componentes o algoritmos considerados para este dominio se describen
a continuaci�n: 
\begin{description}
\item [{Matrix~Multiplication}] implementaci�n simple desarrollada puramente
en lenguaje Java. 
\item [{Matrix~Multiplication~Multi~Thread}] corresponde a la misma
versi�n del componente anterior paralelizada en ocho threads de ejecuci�n. 
\item [{Matrix~Multiplication~Render~Script}] es una versi�n implementada
con RenderScript , de modo que si el dispositivo tiene un \ac{GPU}
compatible con RenderScript, �ste lo ejecutar�. Por tal raz�n el tiempo
de operaci�n es mucho m�s r�pida. 
\item [{Native~Matrix~Multiplication}] es una versi�n que usa \ac{JNI}
para que la multiplicaci�n se realice sobre c�digo nativo desarrollado
en lenguaje C++, el cual se compila espec�ficamente seg�n cada arquitectura
de \ac{CPU} (Armeabi, x86, etc). 
\item [{Matrix~Multiplication~with~Eigen}] versi�n que usa \ac{JNI}
para ejecutar la multiplicaci�n de matrices provista por la librer�a
nativa Eigen. 
\item [{Matrix~Multiplication~with~OpenCV}] versi�n que usa \ac{JNI}
para ejecutar la multiplicaci�n a trav�s de la librer�a OpenCV . 
\end{description}
Las entradas del problema utilizadas para la evaluaci�n fueron generadas
aleatoriamente a partir de las dimensiones (tama�o) de las matrices
involucradas a partir del n�mero de filas y columnas de la primer
matriz denominada matriz A y el n�mero de columnas de la segunda matriz
denominada matriz B, esta distinci�n es importante teniendo en cuenta
la propiedad de no conmutatividad del producto de matrices. Adicionalmente,
ambas matrices pueden ser configuradas externamente. De cada problema,
las propiedades que se tomaron en cuenta fueron: el n�mero de columnas
de la primer matriz (\emph{AColumn}), el n�mero de filas y el de columnas
de la segunda matriz (\emph{ARow} y \emph{BColumn}), la complejidad
temporal de la operaci�n definida como O(AColumn x ARow x BColumn),
el nombre del componente y el tiempo de respuesta de la operaci�n
registrada en segundos. No fueron incluidos atributos relativos al
dispositivo de ejecuci�n, sin embargo podr�an considerarse propiedades
como n�mero de cores de \ac{CPU}, cores de \ac{GPU}, frecuencia,
etc. ya que se piensa son buenos predictores de tiempo para algunos
algoritmos. Los dataset bajo el nombre \emph{Response time NxNxN -
Samsung Galaxy S3} y \emph{Response time NxNxN - Lenovo K3 Note} se
componen de 894 instancias lo cual comprende un total de 149 entradas
por cada uno de los seis componentes. Por otro lado, los dataset bajo
el nombre de \emph{Response time NxMxL - Sin RenderScript component
- Samsung Galaxy S3} y \emph{Response time NxMxL - Sin RenderScript
component - Lenovo K3 Note} excluyen, la versi�n implementada con
RenderScript y comprenden 745 instancias. 

A continuaci�n en la figura \ref{fig:matrix-dataset-behaviour} se
muestra el comportamiento del atributo \emph{Tiempo de respuesta}
de los dataset formados para este dominio, en referencia a la imagen,
A) Sin componente RenderScript en Lenovo K3, B) Sin componente RenderScript
en Samsung Galaxy S3, C) Todos los componentes en Lenovo K3 y D) Todos
los componentes en Samsung Galaxy S3.

\begin{figure}[H]
\begin{centering}
\includegraphics[scale=0.55]{C:/Users/usuario/Tesisworkspace/Tesis_Standalone/tesis/images/matrix-dataset-behaviour}
\par\end{centering}

\caption{Comportamiento del atributo \textquoteleft Tiempo de respuesta\textquoteright{}
para el problema de la multiplicaci�n de matrices.\label{fig:matrix-dataset-behaviour}}
\end{figure}


En todos los casos puede observarse una distribuci�n exponencial de
los datos incluyendo valores extremos o infrecuentes. Respecto a los
valores de tiempo registrados, puede notarse a simple vista que las
pruebas ejecutadas en el dispositivo Lenovo K3 consumen menor tiempo
en contraste con el dispositivo Samsung Galaxy S3. 

Lo interesante de este escenario es que permite la creaci�n de modelos
simples, por ejemplo, a partir de la t�cnica Linear Regression que
arroje los mismos o mejores resultados que otras t�cnicas m�s complejas
ya que entre los t�rminos de la funci�n de predicci�n se expresa la
complejidad del algoritmo de matrices. Incluso, se podr�an usar t�cnicas
espec�ficas para cada componentes evaluando y analizando el impacto
sobre la predicci�n. Por ejemplo, para el componente \emph{Multiplicaci�n
MultiThread} en la t�cnica Linear Regression puede considerarse el
n�mero de threads; para el componente con RenderScript puede considerarse
alguna caracter�stica propia del \ac{GPU} donde se ejecuta si implica
alguna mejora en los resultados predictivos. 


\section{Resultados y discusi�n\label{sec:Resultados-y-discusi=0000F3n}}

En esta secci�n final de la evaluaci�n se agrupan los resultados para
contrastar los escenarios entre s� con el fin de determinar relaciones
entre las t�cnicas, el contexto o dominio en el que se aplican y el
desempe�o sobre los atributos que predicen, esto es, si la t�cnica
se ajusta m�s adecuadamente a un tipo de atributo o a otro, a modo
de inferir las conclusiones pertinentes. Adem�s, se analizan detalles
sobre la performance del proceso de optimizaci�n de las t�cnicas respecto
al tiempo de c�mputo que insumen y su relaci�n con el tama�o del dataset
que procesa. 


\subsection{Resultados para el problema del viajante.\label{subsec:Resultados-para-el}}

\begin{table}[H]
\begin{centering}
\includegraphics[bb=0bp 0bp 596bp 619bp,scale=0.55]{C:/Users/usuario/Tesisworkspace/Tesis_Standalone/tesis/images/response-TSP}
\par\end{centering}

\caption{Resultados del atributo Tiempo de respuesta para el escenario \textquoteleft problema
del viajante\textquoteright .\label{tab:Resultados-del-atributo}}
\end{table}


A partir de la figura \ref{tab:Resultados-del-atributo} se desprenden
las conclusiones con respecto al primer escenario. Como primer punto
tanto el modelo de \emph{MultiLayer Perceptron} como el de \ac{SVM}
son, claramente, los mejores en cuanto a adaptaci�n al modelo. Ambos
presentan una correlaci�n con los datos de 1 (\emph{CC}), lo que representa
la exactitud de los datos calculado con respecto a los reales. Como
ya se explic� en la secci�n\ref{subsec:Ajuste-del-modelo:}, esto
puede parecernos lo m�s �ptimo, pero observando las m�tricas y las
curvas de errores, ambos modelos presentan un problema de overfitting. 

Teniendo esto en cuenta, se descartan ambos como los mejores modelos
para este tipo de problema. A su vez, se descarta el \emph{K-mean
Clusterer }ya que los niveles de correlaci�n son los mas bajos y los
errores los m�s altos.

En todos los escenarios presentados, se opt� por el caso intermedio
de adaptaci�n, considerando que el mismo es el mejor obtenido evitando
casos de overfitting o underfitting. Este se escogi� teniendo en cuenta
los valores de CC intermedios, con un error MAE aceptable considerando
los valores propios y errores promedios bajos. As�, en este caso,
el m�s eficiente es el \emph{Linear Regression. }A continuaci�n, se
presenta el comportamiento de este �ltimo con respecto a los valores
de referencia.

\begin{figure}[H]
\begin{centering}
\includegraphics[scale=0.55]{C:/Users/usuario/Tesisworkspace/Tesis_Standalone/tesis/images/linear-regression-behaviour.PNG}
\par\end{centering}

\caption{Lineas de predicci�n en la herramienta Nekonata}
\end{figure}


Con el mismo m�todo de analisis se analiz� el dataset TSPlib.csv considerando
la precisi�n del algoritmo.

\begin{table}[H]
\begin{centering}
\includegraphics[bb=0bp 0bp 596bp 619bp,scale=0.55]{C:/Users/usuario/Tesisworkspace/Tesis_Standalone/tesis/images/precision-TPS}
\par\end{centering}

\caption{Resultados del atributo precisi�n para el escenario \textquoteleft problema
del viajante\textquoteright .}
\end{table}


As�, se puede apreciar que los que presentan una mejor adaptaci�n
en este caso siguen siendo el \ac{MLP} y el \ac{SMO}. Aun as�, considerando
el analisis antes hecho, esta vez el mejor aloritmo es el \ac{SGD}. 

Si tenemos en cuenta lo visto en la secci�n \ref{subsec:Regresi=0000F3n-Lineal},
el mejor algoritmo te�rico que mejor se adapta a el problem del viajante
es el \emph{LinearRegression }es sus dos implementaciones (Ridge y
SGD). 

\begin{figure}[H]


\begin{centering}
\includegraphics[scale=0.55]{C:/Users/usuario/Tesisworkspace/Tesis_Standalone/tesis/images/errors-comparison-response-S5}
\par\end{centering}

\caption{Figura comparativa de las dos implementaciones de regresi�n lineal
para el problema del vaiajante}


\end{figure}



\subsection{Resultados para los servicios de detecci�n de rostros.}

Durante el proceso de formaci�n de los dataset, se consider� en primer
medida, la incorporaci�n de variables sobre la imagen, como el tama�o,
la intensidad de color, entre otras. Sin embargo, fueron descartadas
en la versi�n final ya que no demostraron influir sobre la precisi�n
y el tiempo de respuesta al alcanzar un grado de correlaci�n por debajo
de 0.05 entre las variables de la imagen y las variables a predecir.
S�lo los par�metros de configuraci�n con los que se invocan a la funci�n
de detecci�n de cada componente afectaron directamente al tiempo de
respuesta requerido por el servicio. Los resultados de los modelos
obtenidos para la predicci�n del atributo \emph{Tiempo de respuesta}
se exponen en el cuadro \ref{tab:Resultados-response-face}

<A�adir conclusiones>

\begin{table}[H]
\begin{centering}
\includegraphics[bb=0bp 0bp 596bp 619bp,scale=0.55]{C:/Users/usuario/Tesisworkspace/Tesis_Standalone/tesis/images/response-face}
\par\end{centering}

\caption{Resultados del atributo \textquoteleft Tiempo de respuesta\textquoteright{}
para los servicios de detecci�n de rostros.\label{tab:Resultados-response-face}}
\end{table}



\subsection{Resultados para el problema de la mochila.}

<A�adir conclusiones >

\begin{table}[H]
\begin{centering}
\includegraphics[bb=0bp 0bp 596bp 619bp,scale=0.55]{C:/Users/usuario/Tesisworkspace/Tesis_Standalone/tesis/images/response-knapsack}
\par\end{centering}

\caption{Resultados del atributo Tiempo de respuesta para el escenario \textquoteleft problema
de la mochila\textquoteright . }
\end{table}


\begin{table}[H]
\begin{centering}
\includegraphics[bb=0bp 0bp 596bp 619bp,scale=0.55]{C:/Users/usuario/Tesisworkspace/Tesis_Standalone/tesis/images/precision-knapsack}
\par\end{centering}

\caption{Resultados del atributo precisi�n para el escenario \textquoteleft problema
de la mochila\textquoteright . }
\end{table}



\subsection{Resultados para el problema de la multiplicaci�n de matrices.}

<A�adir conclusiones>

\begin{table}[H]
\begin{centering}
\includegraphics[bb=0bp 0bp 596bp 619bp,scale=0.55]{C:/Users/usuario/Tesisworkspace/Tesis_Standalone/tesis/images/reponse-matrix}
\par\end{centering}

\caption{Resultados del escenario \textquoteleft Multiplicaci�n de matrices\textquoteright{}}
\end{table}



\section{Conclusiones \label{sec:Conclusiones}}
\end{document}
\acresetall
\chapter{Conclusiones\label{chap:Conclusiones}}



En este trabajo se presentó un enfoque de aprendizaje automático para
la estimación de atributos no funcionales en componentes de aplicaciones
Android. Para darle soporte al enfoque, se implementaron dos herramientas
especificas. La primera es una biblioteca para capturar mediciones
de propiedades no funcionales durante la ejecución de componentes
y algoritmos en dispositivos Android. La segunda es una herramienta
de escritorio para entrenar modelos de predicción a partir de las
mediciones capturadas. Esta herramienta incluye varias técnicas de
regresión implementadas por la biblioteca Weka. Su diseño modular
permite incorporar fácilmente las implementaciones de otras librerías
de aprendizaje de máquina. La herramienta brinda la posibilidad de
crear modelos predictivos óptimos, adecuados a un conjunto de datos
particular y un atributo a predecir particular.

Ciertamente, el uso de modelos como criterio de calidad para la selección
de servicios y componentes se está extendiendo cada vez más, promoviendo
el desarrollo y optimización de técnicas de aprendizaje de máquina.
En primer lugar, en este trabajo se presentaron las características
principales de las técnicas que aplican aprendizaje sobre datos y
se analizaron algunas propiedades no funcionales de componentes, como
tiempo de respuesta. Luego, se estudiaron diferentes herramientas
de recolección de datos en sistemas Android examinando los principales
artefactos y mecanismos involucrados en estas herramientas.

Como resultado del creciente desarrollo de aplicaciones, la integración
de componentes de terceros se ha popularizado en diversas áreas, como
el de las aplicaciones móviles. En este sentido, se introdujeron las
nociones básicas de los dispositivos móviles, estudiando también,
el sistema operativo más difundido en estos dispositivos, Android.
Se analizaron las características más importantes del mismo, y se
presentaron los conceptos necesarios para comprender el funcionamiento
de las aplicaciones en este sistema.

La utilización de componentes y algoritmos de terceros en dispositivos
móviles presenta ciertos desafíos. Estos dispositivos tienen limitaciones
en conflicto como la energía, el acceso a la red y la capacidad de
cálculo que determinan el contexto de ejecución de estos componentes
y que afecta considerablemente los atributos de calidad y desempeño
de los mismos y de las aplicaciones que los invocan. Por lo tanto,
es importante elegir los componentes adecuados de acuerdo con su calidad
de servicio además de la funcionalidad requerida. En este sentido
se describieron algunas investigaciones recientes en el área que aplican
aprendizaje de máquina y técnicas de regresión para estimar propiedades
de desempeño de algoritmos y componentes en ejecución.

Por último, se llevo a cabo una serie de experimentos sobre 3 casos
de estudio para validar el enfoque propuesto. Los mismos fueron evaluados
con las métricas presentadas en secciones anteriores, señalando al
modelo MultiLayer Perceptron como el mejor a nivel de predicciones
bajas en error. Las estimaciones permiten predecir que en dominios
de problemas clásicos como el caso del problema de la multiplicación
de matrices y el problema del viajante métrico, el nivel de correlación
alcanzado por el algoritmo es cercano a 1 y en cuanto al error en
las predicciones a través de las métrica \ac{RMSE} , tendrán valores
inmediatamente cercanos a cero. Para el caso de servicios, aunque
las estimaciones arrojadas por el modelo presentan mas variabilidad,
su comportamiento es considerablemente superior al resto de los modelos,
en cuanto evita efectos de overfit (sobreajuste) y underfit (subajuste)
sobre los datos. Esto indica que una generalización del modelo propuesto
puede resultar útil para determinar a priori qué componentes resultarán
más eficientes, dadas ciertas características en las entradas, en
los componentes y en la operación, para la predicción del tiempo de
respuesta.

Respecto a los servicios para detección de rostros, se observa que
las técnicas tiene dificultades para aprender modelos con buena precisión.
Esto se debe principalmente a la complejidad del problema, ya que
es difícil extraer características de las imágenes de entrada que
estén correlacionadas con el tiempo de respuesta. En este caso de
estudio, el modelo K means Clusterer logra superar  al modelo MultiLayer
Perceptron. Para el servicio Microsoft, el modelo alcanza un factor
de correlación un 2\% menor que el factor determinado por el modelo
MultiLayer Perceptron, y para los servicios Google Play y SkyBiometry
 el modelo de cluster determina aún mejor la relación entre las variables
y reduce el error en las predicciones respecto al modelo MultiLayer,
que por sus resultados fue tomado como base para el análisis de las
técnicas en general. 


\section{Limitaciones}

La principal limitación que presenta el enfoque propuesto en este
trabajo es que el entrenamiento de modelos de precisión requiere una
gran cantidad de mediciones para generalizar mejor los datos. En la
evaluación presentada, las mediciones fueron capturadas sobre unos
pocos dispositivos y con un contexto de ejecución normal. Por lo tanto,
los modelos construidos solo pueden predecir el tiempo de respuesta
para esos dispositivos y contextos específicos. Idealmente la medición
debería realizarse sobre múltiples dispositivos y con diferentes contextos
de ejecución (por ejemplo, distinta disponibilidad de CPU, memoria,
y otros recursos) de tal forma que los modelos puedan estimar con
precisión el tiempo de respuesta de determinado componente en un contexto
mas amplio. Sin embargo, esto es muy costoso porque requiere disponer
de numerosos dispositivos y tiempo para capturar un conjunto de mediciones
mas exhaustiva.

Una segunda limitación parte de la incapacidad de los dispositivos
móviles para llevar adelante ejecuciones mas complejas y costosas,
que insumen gran cantidad de memoria y procesamiento. Este factor,
limita la posibilidad de llevar a cabo la etapa de aprendizaje del
enfoque sobre el mismo dispositivo móvil, obligando a desarrollar
la herramienta de aprendizaje para su ejecución en computadoras de
escritorio. 

Por última, otra limitación que se presenta responde a la ausencia
de un método automatizado para configurar dinámicamente los parámetros
de las técnicas de regresión, debiendo determinar un rango estático
y específico propio para cada técnica, para llevar a cabo el proceso
de optimización, considerando de esta forma a todos los datos de entrenamiento
por igual, sin distinción de formato, dominio, etc.




\section{Trabajos Futuros}

Cabe mencionar que existen algunas posibilidades de extensión interesantes
para este trabajo. Como primer objetivo se pretende mejorar el tiempo
de procesamiento requerido para la optimización y generación de los
modelos predictivos. Para poder lograr esto, se puede otorgar a la
aplicación la capacidad de ejecución paralela de los procesos de optimización
y construcción de los diferentes modelos. Por otro lado, se debe dar
un mejor soporte a la etapa de medición para considerar una mayor
variedad de dispositivos y contextos de ejecución sobre los que capturar
mediciones. Con este fin, se puede recurrir a algún proveedor de dispositivos
en la nube (Device-as-a-Service), como Amazon WS Device Farm\footnote{https://aws.amazon.com/es/device-farm/}
o Samsung Remote Test Lab\footnote{http://developer.samsung.com/rtlLanding.do}.
Estos servicios brindan acceso remoto a dispositivos móviles reales
distribuidos geográficamente, lo que permite capturar mediciones sobre
un grupo más amplio de dispositivos con características variadas.

Adicionalmente se pueden incorporar nuevos aspectos para mejorar y
comparar las técnicas y modelos en la herramienta de aprendizaje: 
\begin{itemize}
\item Independizar al módulo de optimización de técnicas, de diferentes
rangos de configuración para cada uno de los parámetros que incluya
la técnica. De esta forma, lograr más énfasis en los valores de cada
parámetro y explotar al máximo el desempeño de las técnicas. Incluso,
el usuario podría disponer de la posibilidad de personalizar estos
valores.
\item Informar y registrar el tiempo real consumido por cada operación de
construcción de un modelo. Operación que incluye la configuración,
optimización, entrenamiento y finalmente la evaluación del modelo
con las métricas correspondientes. 
\item Extender el análisis comparativo a nivel de librerías, aprovechando
la capacidad de la herramienta de integrar librerías de terceros.
De esta manera, se logra comparar el desempeño de una técnica conocida
en diferentes implementaciones, y seleccionar la mejor técnica independiente
de la librería que la implemente. 
\item Considerar conjuntos de datos de la misma naturaleza que los usados
durante la fase de entrenamiento, para evaluar los modelos creados.
Brindar un mecanismo de evaluación de los modelos a partir de las
métricas, y validar estos modelos creados frente a nuevos conjuntos
de datos para esclarecer el desempeño y generalización lograda. \end{itemize}

\acresetall

\selectlanguage{american}%
\newpage{}

\selectlanguage{spanish}%
\bibliographystyle{plainnat}
\addcontentsline{toc}{chapter}{\bibname}\bibliography{Bibliograf-ía}


\newpage{}

\selectlanguage{american}%
\acresetall

\selectlanguage{spanish}%
\appendix

\chapter{Implementaci�n de los algoritmos de resoluci�n de TSP \label{chap:Ambiente-de-desarrollo}}

El Problema del Agente Viajero (\emph{TSP -Travelling salesman Problem})
corresponde a uno de los problemas de optimizaci�n m�s estudiados
de la clase NP-Completos; es un problema de minimizaci�n que comienza
y termina en un v�rtice espec�fico y se visita el resto de los v�rtices
exactamente una vez. El \ac{TSP} puede ser modelado como un grafo
ponderado no dirigido, de manera que las ciudades sean los v�rtices
del grafo, los caminos son las aristas y las distancias de los caminos
son los pesos de las aristas.


\section*{Algoritmos de resoluci�n}


\subsection*{1. Best Fit - Bin Packing -}

Best fit es uno de los algoritmos heur�sticos m�s simples,brinda soluciones
�ptimas (aproximadas) aunque el algoritmo no asegura el retorno de
la mejor soluci�n. Para dominios de grafos no completos incluso, el
camino soluci�n puede contener arcos con costo infinito (ausencia
de arista). Partiendo del punto \emph{A} se analiza el trayecto m�s
corto a la siguiente ciudad, sea �sta por ejemplo la ciudad \emph{C},
ambas ciudades se a�aden al conjunto Soluci�n: \{A, C, A\} 

En cada paso del algoritmo las ciudades son seleccionadas de acuerdo
al costo de sus trayectos y son incorporadas al conjunto Soluci�n
en base al c�lculo de la distancia marginal de las intersecciones. 

La distancia marginal representa la variaci�n en el costo total teniendo
en cuenta el costo directo entre \emph{A} y \emph{C} y el costo del
camino entre \emph{A} y \emph{C} pasando por \emph{B}. (A,B,C). El
par de ciudades \emph{A} y \emph{C} son seleccionadas de forma tal
que hagan m�nimo el costo del paso por la ciudad \emph{B}.- 

Ejemplo

\begin{tabular}{|c|c|c|c|c|}
\hline 
\selectlanguage{english}%
\selectlanguage{english}%
 & B & C & D & E\tabularnewline
\hline 
A & 2 & 1 & 10 & 25\tabularnewline
\hline 
B & \selectlanguage{english}%
\selectlanguage{english}%
 & 18 & 5 & 5\tabularnewline
\hline 
C & \selectlanguage{english}%
\selectlanguage{english}%
 & \selectlanguage{english}%
\selectlanguage{english}%
 & 20 & 2\tabularnewline
\hline 
D & \selectlanguage{english}%
\selectlanguage{english}%
 & \selectlanguage{english}%
\selectlanguage{english}%
 & \selectlanguage{english}%
\selectlanguage{english}%
 & 8\tabularnewline
\hline 
\end{tabular}

S: \{ A, C, A\} 

Selecci�n ciudad E: 

Trayectos: AC - CA

En tal caso, los trayectos son an�logos como as� tambi�n la distancia
marginal: E se a�ade entre las ciudades C y A. 

S: \{ A,C,E,A\} 

Costo: 28 

En el siguiente paso la ciudad B es seleccionada: 
\begin{lyxlist}{00.00.0000}
\item [{Trayectos:}]~

\begin{lyxlist}{00.00.0000}
\item [{$AC:D=costo(A,B)+costo(B,C)-costo(A,C)=19$}]~
\item [{Resultando}] S: \{A, B, C, E, A\} Costo ahora:28+19=47 
\item [{\,}]~
\item [{$CE:D=costo(C,B)+costo(B,E)-costo(C,E)=21$}]~
\item [{Resultando}] S: \{A, C, B, E, A\} Costo ahora:28+21=49
\item [{\,}]~
\item [{$EA:D=costo(E,B)+costo(B,A)-costo(E,A)=-18$}]~
\item [{Resultando}] S: \{A, C, E, B, A\} Costo ahora:28-18=10
\end{lyxlist}
\end{lyxlist}
B es a�adida entre las ciudades del trayecto cuya distancia es menor,
en el ejemplo ilustrado, la opci�n del tercer trayecto.


\subsection*{2. Vecino m�s cercano}

Partiendo de alguna ciudad arbitraria se analizan todos los v�rtices
adyacentes y a�n no visitados, y se a�ade a la soluci�n aquel v�rtice
cuya arista de costo sea la m�nima. Tal an�lisis prosigue a partir
del �ltimo v�rtice a�adido hasta haber visitado todas las ciudades. 

A menudo, el \ac{TSP} sigue un modelo de grafo completo donde cada
par de v�rtices es conectado por una arista, en tal caso el algoritmo
brinda soluciones �ptimas partiendo desde y hacia la ciudad inicial
pasando exactamente una vez por cada ciudad restante. 

Por otro lado,en modelos donde no existe camino entre algunos pares
de ciudades, la elecci�n iterativa de la arista de m�nimo costo sin
una vista global del modelo, puede apartar v�rtices de la soluci�n
retornando soluciones parciales. 

Ejemplo

\includegraphics{\string"C:/Users/usuario/Desktop/Tesis rubbish/tesis/images/TPS-graph-example\string".png}

Soluci�n: A -> B -> C -> D -> A 

Costo: 97


\subsection*{3. Programaci�n Lineal}

El \ac{TSP} puede ser formulado de manera te�rica mediante una (o
varias) formulaciones lineales enteras usando programaci�n lineal
en enteros. Se plantean las variables del problema: 

\lstinline[basicstyle={\footnotesize}]!v�rtices: 0.. n 	Xij = 1 | Xij = 0					            																								para i,j = 0...nLa arista (i,j) pertenece(1)/ no pertenece (0) al conjunto soluci�n.Cij Distancia desde la ciudad i hasta j	     																							   para i,j = 0...n	 aristas n�mero total de aristas. 	aristasReq n�mero m�nimo de aristas para soluci�n. !


\subsection*{4. Backtracking}

Backtracking es un algoritmo general para encontrar todas o algunas
soluciones para problemas computacionales con notables restricciones
de satisfacci�n para dicho problema.El m�todo de b�squeda es incremental
de los candidatos a la soluci�n, abandonando aquellos (\emph{backtrackeando})
tan pronto como se determina que dicha soluci�n no es o va a ser v�lida. 

En el caso del \ac{TSP}, los posibles candidatos a la soluci�n se
determinan mediante el grado de incidencia de todos los nodos que
componen el grafo. Ya que la ciudad s�lo debe ser visitada una vez,
el grado de incidencia de cada ciudad debe ser igual a 2, por lo que
aquellas ciudades que no han sido visitadas (grado de incidencia 0)
o aquellas que s�lo han sido destino (grado de incidencia 1) son candidatas
a componer la soluci�n.

Pseudo-c�digo: 

\lstinline*backtracking(costoAct, minCost, lastV, currPath, bestPath, costoAct) si costoAct<minCost			//Poda por costo que se arrastra 		si !solucion(currPath) 				generarAristasFactibles(lastV) 				Por v:AristasFactibles 						costoAct+=v.costo; 						currPath= currPath + {v}; 						backtracking(costoAct, minCost, v.Destine, currPath); 						costoAct-=v.costo; 						currPath=currPath - {v}; 		sino 				bestPath=currPath;		//Setear la soluci�n ya encontrada 				minCost=costoAct;*


\subsection*{TSP sim�trico }

El TSP sim�trico es una variedad del problema TSP que para todas las
aristas se cumple la desigualdad triangular o de Minkowski: 

\[
(a,c)<(a,b)+(b,c)
\]


Considerando que no todos los grafos creados cumplen con dicha condici�n,
y que a�n as� hay que devolver un resultado �ptimo, se considera prioridad
la distancia entre los nodos y no si los mismos se repiten o no.


\subsubsection*{4. �rbol de recubrimiento }

Para resolver el \emph{Metric TSP}, se puede utilizar el m�todo de
�rbol de recubrimiento m�nimo. El mismo se puede obtener mediante
Prim o Kruskal para, luego de eso, nos aseguramos que el grafo ser�
sim�trico mediante la duplicaci�n de cada arista. Gracias a esto,
podemos recorrerlo mediante un algoritmo recursivo y obtenemos el
ciclo euleriano del mismo. Para finalizar la resoluci�n y mediante
la desigualdad antes planteada, se omiten los nodos repetidos, obteniendo
el ciclo hamiltoniano.


\subsubsection*{4.1 Prim}

El algoritmo Prim toma un tomar un v�rtice arbitrario de los ya visitados
y lo une con el v�rtice no visitado mediante el menor arco que salga
de �l.

Pseudoc�digo:

\lstinline*Prim(firstCity){ 		visited={firstCity}; 		while (visited!=allCities){ 				edge=obtenerMinEdge(visited); 				sol= sol  {edge}; 				visited= visited   {edge.destino};	} 	return sol; }*


\subsubsection*{4.2 Kruskal}

Kruskal mientras tanto tiene ordenados las aristas por costo y va
agregando a la soluci�n aquellos que ya han sido visitados con los
que no.

Pseudoc�digo:

\lstinline!Kruskal(firstCity){ 		visited={firstCity}; 		edges=getSortedEdges(); 		 e  edges{ 				if (e  (visited)x(not_visited)) 						sol=sol  {e}; 						visited=visited  {e.destino} 			} 	return sol; }!

La �nica diferencia entre ambos algoritmos de recubrimiento es la
forma de trabajo de los mismos y en consecuencia su complejidad.


\subsubsection*{4.3 Transformaciones Lineales}

Las transformaciones locales se utilizan para mejorar el hamiltoniano
obtenido por los algoritmos antes presentados. Para introducir el
algoritmo, consideremos la siguiente situaci�n:

\includegraphics{\string"C:/Users/usuario/Desktop/Tesis rubbish/tesis/images/TPS-local-transformations\string".png}

Los nodos unidos en el ciclo hamiltoniano son (A,B) ... (C,D) - l�nea
no punteada en el grafo. 

Ahora bien, para aplicar transformaciones locales se debe verificar: 

\[
(A,C)+(B,D)<(A,B)+(C,D)
\]


De suceder esto, se pueden modificar los arcos y obtener:

\includegraphics{\string"C:/Users/usuario/Desktop/Tesis rubbish/tesis/images/TPS-local-transformations2\string".png}

El resultado es un grafo con menor costo que el anterior presentado.

\chapter{Descripci�n de los servicios de detecci�n facial \label{chap:Ambiente-de-desarrollo}}


\section*{Servicio SkyBiometry}


\subsection*{En referencia al uso de la API \textendash{} Autenticaci�n }

Cada llamada a la API debe ser autorizada mediante el uso de dos claves:
api\_key y api\_secret. Ambas claves son obtenidas mediante una registraci�n
en el sitio oficial de skybiometry (\emph{www.skybiometry.com}) a
trav�s de una cuenta de email como usuario y una contrase�a.


\subsection*{L�mites de uso }

El consumo de este servicio presenta algunas limitaciones que var�an
en base a la suscripci�n particular del usuario. En general, las suscripciones
gratuitas restringen un l�mite de 100 llamadas a m�todos por hora
y de 5000 en el mes. Para el caso del m�todo POST, la limitaci�n se
presenta tanto en 100 request a la API para el procesamiento de im�genes
individuales como para 1 request con 100 im�genes a procesar. 


\subsection*{Factores ofrecidos}
\begin{lyxlist}{00.00.0000}
\item [{Respecto~al~cuadrado~del~rostro~detectado:}]~

\begin{lyxlist}{00.00.0000}
\item [{center~OBJECT~\{~x~:Float,~y~:~Float~\}}]~
\item [{width.~FLOAT.~0-100\%~del~ancho~del~rostro~respecto~al~ancho~de~la~imagen.}]~
\item [{height.~FLOAT.~0-100\%~del~largo~del~rostro~respecto~al~ancho~de~la~imagen.}]~
\end{lyxlist}
\item [{Otros~puntos~detectados:}]~

\begin{lyxlist}{00.00.0000}
\item [{mouth\_center}] OBJECT \{ x :Float, y : Float \} 
\item [{mouth\_left}] OBJECT \{ x :Float, y : Float \}
\item [{mouth\_right}] OBJECT \{ x :Float, y : Float \} 
\item [{eye\_left}] OBJECT \{ x :Float, y : Float \} 
\item [{eye\_right}] OBJECT \{ x :Float, y : Float \} 
\item [{nose}] OBJECT \{ x :Float, y : Float \} 
\item [{ear\_left}] OBJECT \{ x :Float, y : Float \} 
\item [{ear\_right}] OBJECT \{ x :Float, y : Float \} 
\item [{chin}] OBJECT \{ x :Float, y : Float \} 
\item [{yaw.}] FLOAT.Perfil. Value -90\textdegree{} a 90\textdegree{} 
\item [{roll.}] FLOAT. �ngulo de rotaci�n del rostro. Value -90\textdegree{}
a 90\textdegree{} 
\item [{pitch.}] FLOAT. Value -90\textdegree{} a 90\textdegree{} 
\end{lyxlist}
\end{lyxlist}
\begin{description}
\item [{yaw:}] �ngulo positivo en rostros donde predomina el perfil derecho
(respecto al sujeto, no a la imagen). Caso contrario, �ngulo negativo
al predominar el perfil izquierdo. 
\item [{Roll:}] Ejemplo ilustrativo. El rect�ngulo verde representa al
rostro detectado por la API cuyo valor roll es igual a -17\textdegree ,
mientras que el rect�ngulo rojo fue a�adido para ilustrar el caso
donde el �ngulo roll es igual a 0\textdegree . Una rotaci�n inversa
significa un valor roll positivo. 
\item [{\includegraphics{\string"C:/Users/usuario/Desktop/Tesis rubbish/tesis/images/face-example\string".png}}]~
\end{description}
Todos los puntos detectados descritos no especificados corresponden
a un formato JSON, cuya clave es el nombre del mismo, y el valor es
tambi�n un objeto del tipo JSON que contiene los siguientes valores:
id:INTEGER, confidence:INTEGER, y:FLOAT, x:FLOAT, a excepci�n del
valor \emph{center} cuyo objeto JSON contiene s�lo los valores: \emph{x},
\emph{y}.
\begin{description}
\item [{Nota1:}] en caso de valor de atributo no se puede determinar que
no se devuelve. Si se determina el valor que se devuelve junto con
el valor de confianza como porcentaje de 0 a 100. 
\item [{Nota2:}] La API s�lo devuelve informaci�n necesaria para representar
en forma de rect�ngulo los rostros detectados. Para el resto de opciones,
la API s�lo devuelve informaci�n del punto central en cuesti�n. 
\item [{Nota3:}] Ya que la API autom�ticamente re-escala las im�genes a
1024 pixeles para su procesamiento interno, todas las coordenadas
son provistas de forma porcentual respecto al ancho y largo de la
imagen (abstracci�n). 
\item [{Nota4:}] Nota: el atributo face es el valor por defecto y siempre
es retornado, independientemente de los atributos especificados. Si
el atributo \textquotedblleft glasses\textquotedblright{} fue requerido
adicionalmente se retorna el atributo \textquoteleft dark\_glasses\textquoteright{}
para diferenciar entre gafas oscuras y claras. El atributo mood (estado
de �nimo) se devuelve junto con 7 m�s atributos: confianza para cada
una de las emociones b�sicas, adem�s de la confianza neutral\_mood. 
\end{description}

\subsection*{Atributos faciales }

The result of the API call is a JSON or XML object containing the
requested information. Cada uno de los atributos son objetos del tipo
JSON cuya clave es el nombre del mismo y el valor es tambi�n un objeto
JSON con dos valores espec�ficos: el primero con clave \emph{value}
y el segundo con clave \emph{confidence}. El valor de confidence es
del tipo Integer para representar el porcentaje de probabilidad del
valor detectado en el campo \emph{value}.


\subsection*{Mensajes de error }

\includegraphics[scale=0.55]{\string"C:/Users/usuario/Desktop/Tesis rubbish/tesis/images/face-service-errors\string".png}


\subsection*{Post (Request en lenguaje JAVA)}

\begin{lstlisting}[language=Java,breaklines=true]
HttpResponse<JsonNode> response = Unirest.post(https://face.p.mashape.com/faces/detect?api_key=[api_key]&api_secret=[api_secret]) 
.header(X-Mashape-Key, 7rS5YDw5YHmshtdgMHP2ZYBLAljfp1OxKNzjsn1GJxNBgad6C9) 
.header(Accept, application/json) 
.field(attributes, all) 
.field(detector, Aggressive)  [ver opciones] 
.asJson();        
\end{lstlisting}

\begin{description}
\item [{Nota:}] En los espacios {[}api\_key{]} y {[}api\_secret{]} deben
colocarse las claves correspondientes disponibles con la registraci�n
de una cuenta en el sitio www.skybiometry.com \end{description}
\begin{lyxlist}{00.00.0000}
\item [{Opciones:}]~

\begin{lyxlist}{00.00.0000}
\item [{.field(\textquotedblleft files\textquotedblright ,}] Vector<File>
imagenes) 
\item [{.field(\textquotedblleft files\textquotedblright ,}] File imagen) 
\item [{.field(\textquotedblleft urls\textquotedblright ,}] \textquotedblleft \textquotedblright url\_1\textquotedblright ,
\textquotedblleft url\_2\textquotedblright , \dots ) 
\end{lyxlist}
\end{lyxlist}
\begin{description}
\item [{Nota:}] Los formatos de imagen aceptados por la API son los siguientes:
PNG, JPEG, BMP, JPEG2000.
\end{description}
La respuesta recibida por el servicio del tipo HttpResponse<JsonNode>
puede ser manipulada como un objeto del tipo JSONObject a trav�s de
la siguiente codificaci�n: 

\begin{lstlisting}[language=Java,breaklines=true]
JSONObject obj = response.getBody().getObject(); 
\end{lstlisting}



\subsection*{Parse - Overview }

La respuesta en formato JSONObject contiene en primera parte un objeto
del tipo JSONObject con clave \emph{photos} y cuyo valor, representado
en formato JSONArray, contiene toda la informaci�n detectada desde
las im�genes. Cada elemento del arreglo representa un archivo de imagen;
a su vez, estos elementos contienen un objeto del tipo JSONObject
cuya clave es \emph{tags} y su valor es un JSONArray para representar
cada una de los rostros detectados en la imagen particular; a su vez
contiene la informaci�n de ancho y largo asociado. 

En segunda parte, seguido de \emph{photos} se anexan en la respuesta
la informaci�n relacionada a la operaci�n http.

Ejemplo codificaci�n b�sica: 

\begin{lstlisting}[language=Java,breaklines=true]
JSONArray im�genes = obj.get(photos); 
for(int p=0; p<im�genes.length(); p++) { 
	//Obtenci�n de cada una de las im�genes  	
	JSONObject imagen=imagenes.getJSONObject(p);   
	//Ejemplo obtenci�n del ancho de la imagen 
	int imgWidth = (int) imagen.get(width); 
	//Obtenci�n de los rostros detectados en la imagen 
	JSONArray rostros = imagen.getJSONArray(tags); 
	for(int r=0; r<rostros.lenght(); r++) { 
		//Obtenci�n de cada uno de los rostros detectados 
		rostro=rostros.getJSONObject(r);     	 
		//Ejemplo obtenci�n de los puntos detectados JSONObject 
		mouth = rostro.getJSONObject(mouth_center); 
		int coordX= (int) mouth.get(x); 
		int coordY= (int) mouth.get(y); 
		//Ejemplo obtenci�n de los atributos faciales JSONObject 
		atributos=imagen.getJSONObject(attributes); 
		String valor = atributos.getJSONObject(mood).getString(value); 
	} 	
} 
\end{lstlisting}



\section*{Servicio FaceRect}


\subsection*{En referencia al uso de la API }

El consumo de esta API requiere la registraci�n de una cuenta en el
sitio Mashape. La registraci�n se realiza con email y contrase�a y
se realiza de manera gratuita e instant�nea. El servicio no restringe
el n�mero de request a la API pero permite el procesamiento de s�lo
una imagen por request realizado. 


\subsection*{Factores ofrecidos }
\begin{lyxlist}{00.00.0000}
\item [{Rostro~detectado:}]~

\begin{lyxlist}{00.00.0000}
\item [{Orientation:}] tipo String. Valores posibles:<frontal, profile-right,
profile-left> 
\item [{x:}] tipo Integer 
\item [{y:}] tipo Integer 
\item [{width:}] tipo Integer 
\item [{height:}] tipo Integer 
\item [{Features:}] (opcional) 
\item [{eyes}]~
\item [{nose}]~
\item [{mouth}]~
\end{lyxlist}
\end{lyxlist}
Los \emph{features} son analizados por la API s�lo en rostros cuya
orientaci�n sea frontal. El formato de respuesta de los mismos son
del tipo JSONObject en el caso de \emph{nose} y \emph{mouth}. Para
el caso de \emph{eyes} se devuelve un objeto del tipo JSONArray, cuyos
elementos (1 o 2) son del tipo JSONObject. La API representa el rostro
y los features detectados en forma de rect�ngulos, por lo que incorpora
en sus respuestas las coordenadas \emph{x}, \emph{y} (esquina superior
izquierda respecto a la imagen) y los tama�os \emph{width}, \emph{height}
del rect�ngulo en cuesti�n.


\subsection*{Post (Request en lenguaje JAVA)}

\lstinline[language=Java,breaklines=true]!HttpResponse<JsonNode> response = Unirest.post(ENDPOINT) .header(X-Mashape-Key, 7rS5YDw5YHmshtdgMHP2ZYBLAljfp1OxKNzjsn1GJxNBgad6C9) [ver opci�n correspondiente] .field(features, true) .asJson();        !

Actualmente la API soporta dos endpoints difiriendo entre s� en el
modo en que las im�genes son especificadas en el request. Mediante
el m�todo http \emph{GET} se a�ade el URL de la imagen (Endpoint:
http://apicloud-facerect.p.mashape.com/process-url.json); por otra
parte con el m�todo \emph{POST} se a�ade el archivo correspondiente
(Endpoint: http://apicloud-facerect.p.mashape.com/process-file.json)
De acuerdo con el endpoint especificado se a�ade la l�nea correspondiente: 
\begin{lyxlist}{00.00.0000}
\item [{Opciones}]~

\begin{lyxlist}{00.00.0000}
\item [{.field(\textquotedblleft url\textquotedblright ,}] \textquotedblleft url\_image\textquotedblright ) 
\item [{.field(image,}] new File(<path\_image>)) 
\end{lyxlist}
\end{lyxlist}
\begin{description}
\item [{Nota:}] Para ambos casos, la API restringe el procesamiento de
algunas im�genes, permitiendo procesar aquellas cuyo formato sea JPEG,
PNG y GIF, la resoluci�n de la misma no supere los 4096 p�xeles (4096\texttimes 4096)
y el tama�o sea inferior a 10 MBytes. 
\end{description}
La respuesta recibida por el servicio del tipo HttpResponse<JsonNode>
puede ser manipulada como un objeto del tipo JSONObject a trav�s de
la siguiente codificaci�n: 

\begin{lstlisting}[language=Java,breaklines=true]
JSONObject obj = response.getBody().getObject();
\end{lstlisting}



\subsection*{Parse - overview}

La respuesta en formato JSONObject contiene dos objetos JSON. El primero,
cuya clave es \emph{faces} contiene toda la informaci�n detectada
desde la imagen, el segundo objeto cuya clave es \emph{image} contiene
la informaci�n acerca del ancho (\emph{width}) y largo (\emph{height})
de la misma. La informaci�n detectada se presenta en un objeto del
tipo JSONArray donde cada elemento del arreglo constituye cada uno
de los rostros que ha detectada la API; Cada uno de los elementos
son JSONObject que contienen la informaci�n del rostro (\emph{orientation},
\emph{x}, \emph{y}, \emph{height}, \emph{width}) y los \emph{features}. 

Ejemplo codificaci�n b�sica 

\begin{lstlisting}[language=Java,breaklines=true]
//Ejemplo obtenci�n del ancho de la imagen 
int imgWidth = (int) obj.getJSONObject(image).get(width); 
//Obtenci�n de los rostros detectados en la imagen 
JSONArray rostros = obj.getJSONArray(faces); 
for(int r=0; r<rostros.lenght(); r++) {  	
	//Obtenci�n de cada uno de los rostros detectados 
	rostro=rostros.getJSONObject(r);     	 
	//Ejemplo obtenci�n informaci�n del rostro: 
	String orientation =(String) rostro.get(orientation); 
	int coordX=(int) rostro.get(x); 
	//Ejemplo obtenci�n de los features 
	JSONObject features = rostro.getJSONObject(features); 
	JSONArray eyes =features.getJSONArray(eyes); 
	JSONObject eye_left=eyes.getJSONObject(0); 
}
\end{lstlisting}



\section*{Servicio GMS Vision}


\subsection*{Factores ofrecidos}

El detector puede computar los siguientes atributos (accesibles a
trav�s de los distintos m�todos de la clase).

Profile: rotaci�n del rostro. FLOAT 
\begin{lyxlist}{00.00.0000}
\item [{Clasificadores:<caracter�stica~facial~presente~en~el~rostro>}]~

\begin{lyxlist}{00.00.0000}
\item [{Open}] eyes: (probabilidad) Constant value: 1 FLOAT ({*}) 
\item [{Smiling:}] (probabilidad) Constant value: 1 FLOAT ({*}) 
\item [{Height,}] width: medidos en p�xeles. FLOAT 
\item [{Position:}] Punto superior izquierdo del rostro. FLOAT 
\end{lyxlist}
\item [{Landmarks:<Puntos~de~inter�s~en~el~rostro~detectado>~LIST<Landmark>}]~

\begin{lyxlist}{00.00.0000}
\item [{BOTTOM\_MOUTH}] Constant Value: 0 CENTER 
\item [{LEFT\_CHEEK}] Constant Value: 1 CENTER 
\item [{LEFT\_EAR\_TIP}] Constant Value: 2 
\item [{LEFT\_EAR}] Constant Value: 3
\item [{MIDPOINT}] LEFT\_EYE Constant Value: 4 
\item [{CENTER}] LEFT\_MOUTH Constant Value: 5 
\item [{NOSE\_BASE}] Constant Value: 6 
\item [{MIDPOINT}] RIGHT\_CHEEK Constant Value: 7 
\item [{CENTER}] RIGHT\_EAR\_TIP Constant Value: 8 
\item [{RIGHT\_EAR}] Constant Value: 9 
\item [{MIDPOINT}] RIGHT\_EYE Constant Value: 10 
\item [{CENTER}] RIGHT\_MOUTH Constant Value: 11 
\end{lyxlist}
\item [{Angles:~<Face~Orientation>}]~

\begin{lyxlist}{00.00.0000}
\item [{EulerY}] {[}valor \textquoteleft y\textquoteright{} en la imagen{]} 
\item [{EulerZ}] {[}valor \textquoteleft r\textquoteright{} en la imagen{]}
\end{lyxlist}
\item [{({*})}] En el trabajo desarrollado se consider� el uso de valores
booleanos, determinando a los clasificadores como verdaderos para
valores de probabilidad mayor al 50\%, en casos contrario se determinaron
los mismos como falsos. \end{lyxlist}
\begin{description}
\item [{Nota:}] El �ngulo EulerZ se encuentra disponible siempre en la
detecci�n, mientras que el �ngulo EulerY s�lo si en el detector se
predetermina el modo \emph{accurate}. El concepto de \emph{left} y
\emph{right} son relativos al sujeto no a la posici�n cuando se observa
la imagen. Las posiciones son accedidas a trav�s del m�todo getPosition()
de la clase Landmark; el retorno es del tipo PointF. 
\end{description}

\subsection*{Construcci�n del detector }

\begin{lstlisting}[language=Java,breaklines=true]
FaceDetector detector = new FaceDetector.Builder(context) 
.setTrackingEnabled(false) 
.setMode(int Mode) 
.setProminentFaceOnly(true) 
.setClassificationType(int classificationType) 
.setLandmarkType(int landmarkType) 
.build(); 
\end{lstlisting}


Es necesario especificar en el detector el entorno de nuestra aplicaci�n
(mediante el uso de la clase android.utils.Context) 

Es posible obtener el entorno de la app a trav�s del m�todo \emph{getApplicationContext()}
de la clase Activity. 

El contexto es s�lo una interfaz cuya implementaci�n est� prevista
por el sistema android y permite el acceso a los recursos y clases
espec�ficas de la aplicaci�n, como as� tambi�n realizar operaciones
tales como el despacho de actividades, la difusi�n y recibimiento
de objetos del tipo Intent, etc. 

El servicio dispone de la posibilidad de elegir la preferencia de
procesamiento en la detecci�n, \emph{FAST\_MODE} prioriza la rapidez
en el an�lisis por sobre la cantidad de rostros detectados y su precisi�n,
contrariamente el servicio dispone del modo \emph{ACCURATE\_MODE}

Habilitar la opci�n \emph{tracking enabled} es recomendable para la
detecci�n de im�genes individuales no relacionadas (a diferencia de
un v�deo o una serie de im�genes fijas capturadas consecutivamente). 

Tanto los clasificadores y landmarks pueden especificarse uno a uno
a�adiendo las l�neas correspondientes en el constructor; adicionalmente
la clase FaceDetector proporciona las siguientes constantes: 
\begin{lyxlist}{00.00.0000}
\item [{\emph{ALL\_CLASSIFICATIONS}}]~
\item [{\emph{ALL\_LANDMARKS}}]~
\item [{\emph{NO\_CLASSIFICATIONS}}]~
\item [{\emph{NO\_LANDMARKS}}]~
\end{lyxlist}
El servicio no ofrece la capacidad de setar los landmark y clasificadores
bajo demanda, sino que son definidos est�ticamente al momento de crear
el detector; esta restricci�n condiciona el tiempo de respuesta del
servicio ya que el requerimiento de al menos un landmark, por ejemplo,
significa el uso de la variable \emph{FaceDetector.ALL\_LANDMARKS},
incrementando as� el tiempo de procesamiento de la imagen. 


\subsection*{Uso del detector}

El detector se puede llamar de forma sincronizada con un objeto del
tipo Frame para detectar las caras.

\begin{lstlisting}[language=Java,breaklines=true]
Frame frame = new Frame.Builder().setBitmap(<file_image>).build(); 
SparseArray<Face> faces = detector.detect(frame); 
\end{lstlisting}



\subsection*{Ejemplo codificaci�n b�sica}

\begin{lstlisting}[language=Java,breaklines=true]
for(int i=0;i<faces.size();i++){ 
	//Obtenci�n de cada uno de los rostros detectados en la imagen 
	Face rostro=faces.valueAt(i); 
	//Obtenci�n del rect�ngulo del rostro 
	Float coordX=rostro.getPosition().x; 
	Float coordY=rostro.getPosition().y; 
	Float width=rostro.getWidth(); 
	Float height=rostro.getHeight(); 
	//Obtenci�n del �ngulo EulerZ 
	Float angleZ=rostro.getEulerZ(); 
	//Obtenci�n del clasificador eyes open 
	Float probEyeRight=rostro.getIsRightEyeOpenProbability(); 
	Float probEyeLeft=rostro.getIsLeftEyeOpenProbability(); 
	//Obtenci�n de los Landmarks detectados: 
	List<Landmark> landmarks=rostro.getLandmarks(); 	 
	//Obtenci�n de cada uno de los landmark 
	for(int j=0;j<landmarks.size();j++){ 
		int LandmarkType=landmarks.get(j).getType(); 
		PointF Landmarkposition=landmarks.get(j).getPosition(); 
	} 
} 
\end{lstlisting}

\selectlanguage{english}%

\end{document}
