
\chapter{Introducción}

En los últimos años las aplicaciones móviles se han puesto en el centro
de la escena debido a la proliferación de los dispositivos móviles
y su creciente capacidad de cómputo y almacenamiento\cite{Zunino2011}.
Estos dispositivos pasaron de ser terminales con capacidades limitadas,
generalmente de propósito específico, como agendas electrónicas o
teléfonos celulares, a ser pequeñas computadoras de propósito general
con grandes capacidades de procesamiento, almacenamiento y acceso
a Internet, como tablets y smartphones. 

Los dispositivos móviles de hoy en día utilizan sistemas operativos
similares a las computadoras personales, como Android y iOS. Por lo
tanto, pueden ejecutar software al que hace unos años atrás solo se
encontraba en dichas computadoras. Esto, sumado a su costo accesible,
pequeño tamaño, movilidad y ubicuidad de las conexiones móviles de
alta velocidad, ha impulsado el desarrollo de las aplicaciones móviles
para una amplia variedad de fines, incluyendo entretenimiento, juegos,
comunicaciones, redes sociales, comercio electrónico, turismo, educación,
y mucho más. 

Para reducir los costos y tiempos de desarrollo, es común la reutilización
de software mediante la integración de componentes de terceros. Un
componente\cite{Petritsch2016} es una entidad de software en tiempo
de ejecución que encapsula un servicio, es decir, un conjunto de funciones
y datos, a través de una interfaz específica \ac{API}. Los componentes
se pueden clasificar de acuerdo al ambiente en el que residen durante
su ejecución, distinguiéndose así aquellos que se ejecutan en nodos
remotos, como los denominados Servicios Web \cite{Erickson2009},
de aquellos que residen en el dispositivo, como procesos en segundo
plano, bibliotecas de enlace dinámico, o simples objetos Java. 

Las funcionalidades que implementan estos componentes buscan satisfacer
las necesidades comunes de muchas aplicaciones, como el procesamiento
de texto e imágenes, almacenamiento de datos en la nube, identificación
de usuarios, algoritmos de optimización, etc. La misma funcionalidad
suele ser ofrecida por componentes alternativos que difieren en sus
propiedades no-funcionales o atributos de calidad \cite{Addison2003}.
Estas propiedades son los aspectos que utilizan los desarrolladores
para juzgar su funcionamiento, tales como performance (por ej., tiempo
de respuesta), disponibilidad (por ej., tasa de errores) o precisión
de la respuesta (para el caso de aquellos componentes que procesan
datos de entrada para obtener un resultado de salida). 

Los dispositivos móviles tienen limitaciones en conflicto como la
energía, el acceso a la red y la capacidad de cálculo que determinan
el contexto de ejecución de estos componentes y que afecta considerablemente
los atributos de calidad de los mismos y de las aplicaciones que los
invocan. Por lo tanto, es importante elegir los componentes adecuados
de acuerdo con su calidad de servicio además de la funcionalidad requerida. 

Sin embargo, estos componentes suelen ser \emph{cajas negras} para
los desarrolladores de aplicaciones móviles, que tienen acceso a la
definición de sus \ac{API}s pero no así a su implementación interna,
por lo que no cuentan con información de sus atributos dinámicos para
elegir el componente adecuado en cada contexto de ejecución.


\section{Motivación\label{sec:Motivaci=0000F3n}}

Para predecir la performance de un sistema, algunos enfoques definen
un modelo del mismo como función de agregación que considera el desempeño
individual de sus componentes. Así, por ejemplo, el tiempo de respuesta
de un \emph{mashup} de servicios Web puede determinarse como la suma
de los tiempos de respuesta de los servicios invocados \cite{Rosenberg2009}.
Lo mismo se puede aplicar sobre otras propiedades de arquitecturas
y aplicaciones que involucran el ensamblado de diferentes componentes
\cite{Crnkovic2011}\cite{Sanchez2015}. 

Muchos estudios se enfocan en el análisis y predicción de los aspectos
dinámicos de componentes individuales, siendo las técnicas de aprendizaje
de máquina, y en particular las de regresión, las más utilizadas.
La mayor parte de estos estudios se enfocan en algoritmos de optimización
e inteligencia artificial \cite{Hutter2014}, donde se destaca el
trade-off entre los tiempos de ejecución y la precisión o calidad
de las respuestas obtenidas. En estos trabajos, se obtienen mediciones
a partir de sucesivas ejecuciones de los algoritmos y se entrenan
modelos con estas mediciones. Los modelos son funciones que dado determinado
algoritmo e instancia de un problema permiten estimar su desempeño.
Generalmente, diferentes algoritmos se desempeñan mejor que otros
para distintas instancias, por lo que los modelos resultan útiles
como criterio de decisión en la selección y parametrización automática
de algoritmos, la asignación óptima de tareas en contextos Grid, entre
otros escenarios. 

La predicción de performance también se ha extendido a servicios Web
\cite{Zheng2013} en una modalidad conocida como predicción colaborativa,
en donde las mediciones son recolectadas y compartidas por múltiples
nodos distribuidos en todo el mundo. En este caso, los modelos permiten
determinar latencia, disponibilidad, y otras propiedades de un servicio
a partir de la ubicación geográfica del cliente que lo invoca y otras
características del contexto. 

En las aplicaciones móviles, donde el contexto de ejecución y disponibilidad
de los recursos puede variar rápidamente (cambio de red de acceso
a internet, reducción de la batería, uso limitado de memoria y \ac{CPU},
etc) y diferentes componentes consumen estos recursos de diferente
forma, es interesante el uso de estos modelos como criterio de calidad
para la selección de servicios y componentes, tanto en tiempo de desarrollo
como en tiempo de ejecución. Por esta razón, se plantea la posibilidad
de desarrollar un enfoque para construir y evaluar modelos de predicción
en el contexto de aplicaciones móviles. 


\section{Objetivos y Solución propuesta\label{sec:Objetivos-Soluci=0000F3n-propuesta}}

El objetivo del trabajo consiste en desarrollar un enfoque para la
elaboración de modelos de predicción de propiedades dinámicas de componentes
accedidos por aplicaciones móviles. Basándonos en el hecho de que
Android es el sistema operativo más difundido para dispositivos móviles,
el enfoque se pondrá a prueba sobre diferentes casos de estudio en
este sistema operativo. 

El enfoque se basa en un proceso de aprendizaje de máquina. Dado un
conjunto de componentes que implementan la misma funcionalidad y de
los cuales queremos predecir propiedades dinámicas, el método propuesto
es el siguiente: 
\begin{enumerate}
\item Usar conocimiento del dominio para seleccionar características del
contexto y de los datos de entrada del componente que puedan ser indicativos
de su desempeño. 
\item Generar un conjunto de datos de entrada representativos del espacio
de entrada para la evaluación de los componentes. 
\item Ejecutar los componentes con las entradas generadas y tomar mediciones
de las características identificadas en el punto 1 más las propiedades
de interés a predecir: tiempo de respuesta, calidad de la respuesta,
etc. 
\item Usar estas mediciones con técnicas de aprendizaje de máquina para
entrenar y evaluar modelos de predicción. 
\end{enumerate}
El enfoque propuesto se pondrá a prueba sobre grupos de componentes
y servicios reales que implementan funcionalidades de interés para
desarrolladores de aplicaciones móviles. Esta evaluación no sólo involucrará
diferentes dominios, sino también diferentes técnicas, como regresiones
y redes neuronales, sobre diferentes propiedades de interés. Para
llevar a cabo la medición de los componentes se implementará una herramienta
de test de performance para la plataforma Android, y se utilizará
software de aprendizaje de máquina como Weka \cite{Hall2009}. 

En conclusión, se espera poder proveer un análisis de los dominios
y técnicas consideradas, como así también compararlas tanto en términos
de precisión como también en cuanto a su generalización a diferentes
contextos de ejecución (dispositivos) y datos de entrada


\section{Organización\label{sec:Organizaci=0000F3n}}

El resto del trabajo se organiza en 5 capítulos. A continuación se
da un breve resumen de los temas que se abordan en cada uno de ellos. 

En el capítulo\ref{chap:Marco-Teorico} se presenta el marco teórico,
donde se definen los conceptos utilizados a lo largo de todo el informe,
tales como: Android, desempeño, precisión, aprendizaje de máquina,
regresión, modelos, componentes. 

En el capítulo\ref{chap:Trabajos-Relacionados} se presentan herramientas
de benchmarks tanto para el sistema Android como para Java y se exponen
algunos trabajos relacionados desde diferentes perspectivas, predicciones
que incluyen modelos de performance y precisión como aproximación
de éxito de los resultados, no sólo en algoritmos de optimización
sino también en servicios Web. 

En el capítulo \ref{chap:Enfoque-y-Herramientas} se describe el enfoque
y las herramientas utilizadas. Se detalla la arquitectura e implementación
de las mismas, ahondando en las decisiones de diseño que se consideran
importantes para el entendimiento y reutilización de las mismas. 

En el capítulo \ref{chap:Evaluaci=0000F3n} se presenta la evaluación
de los modelos obtenidos a partir de todo el framework presentado.
Se presentan las variables consideradas en los escenarios evaluados,
los modelos que han sido analizados y las conclusiones alcanzadas,
presentando justificativo para la selección de los mejores modelos
predictivos alcanzados. Mediante el presente trabajo se han alcanzado
modelos de predicción para el tiempo de respuesta con un coeficiente
de correlación de 0.73 en promedio y un error medio de 3.19 milisegundos
en la predicción.

Finalmente, en el capítulo \ref{chap:Conclusiones} se exponen las
conclusiones del trabajo realizado, las limitaciones actuales del
enfoque y la herramienta, y posibles líneas de trabajo futuro.
