
\chapter{Implementación de los algoritmos de resolución de TSP \label{chap:Ambiente-de-desarrollo}}

El Problema del Agente Viajero (\emph{TSP -Travelling salesman Problem})
corresponde a uno de los problemas de optimización más estudiados
de la clase NP-Completos; es un problema de minimización que comienza
y termina en un vértice específico y se visita el resto de los vértices
exactamente una vez. El \ac{TSP} puede ser modelado como un grafo
ponderado no dirigido, de manera que las ciudades sean los vértices
del grafo, los caminos son las aristas y las distancias de los caminos
son los pesos de las aristas.


\section*{Algoritmos de resolución}


\subsection*{1. Best Fit - Bin Packing -}

Best fit es uno de los algoritmos heurísticos más simples,brinda soluciones
óptimas (aproximadas) aunque el algoritmo no asegura el retorno de
la mejor solución. Para dominios de grafos no completos incluso, el
camino solución puede contener arcos con costo infinito (ausencia
de arista). Partiendo del punto \emph{A} se analiza el trayecto más
corto a la siguiente ciudad, sea ésta por ejemplo la ciudad \emph{C},
ambas ciudades se añaden al conjunto Solución: \{A, C, A\} 

En cada paso del algoritmo las ciudades son seleccionadas de acuerdo
al costo de sus trayectos y son incorporadas al conjunto Solución
en base al cálculo de la distancia marginal de las intersecciones. 

La distancia marginal representa la variación en el costo total teniendo
en cuenta el costo directo entre \emph{A} y \emph{C} y el costo del
camino entre \emph{A} y \emph{C} pasando por \emph{B}. (A,B,C). El
par de ciudades \emph{A} y \emph{C} son seleccionadas de forma tal
que hagan mínimo el costo del paso por la ciudad \emph{B}.- 

Ejemplo

\begin{tabular}{|c|c|c|c|c|}
\hline 
\selectlanguage{english}%
\selectlanguage{english}%
 & B & C & D & E\tabularnewline
\hline 
A & 2 & 1 & 10 & 25\tabularnewline
\hline 
B & \selectlanguage{english}%
\selectlanguage{english}%
 & 18 & 5 & 5\tabularnewline
\hline 
C & \selectlanguage{english}%
\selectlanguage{english}%
 & \selectlanguage{english}%
\selectlanguage{english}%
 & 20 & 2\tabularnewline
\hline 
D & \selectlanguage{english}%
\selectlanguage{english}%
 & \selectlanguage{english}%
\selectlanguage{english}%
 & \selectlanguage{english}%
\selectlanguage{english}%
 & 8\tabularnewline
\hline 
\end{tabular}

S: \{ A, C, A\} 

Selección ciudad E: 

Trayectos: AC - CA

En tal caso, los trayectos son análogos como así también la distancia
marginal: E se añade entre las ciudades C y A. 

S: \{ A,C,E,A\} 

Costo: 28 

En el siguiente paso la ciudad B es seleccionada: 
\begin{lyxlist}{00.00.0000}
\item [{Trayectos:}]~

\begin{lyxlist}{00.00.0000}
\item [{$AC:D=costo(A,B)+costo(B,C)-costo(A,C)=19$}]~
\item [{Resultando}] S: \{A, B, C, E, A\} Costo ahora:28+19=47 
\item [{\,}]~
\item [{$CE:D=costo(C,B)+costo(B,E)-costo(C,E)=21$}]~
\item [{Resultando}] S: \{A, C, B, E, A\} Costo ahora:28+21=49
\item [{\,}]~
\item [{$EA:D=costo(E,B)+costo(B,A)-costo(E,A)=-18$}]~
\item [{Resultando}] S: \{A, C, E, B, A\} Costo ahora:28-18=10
\end{lyxlist}
\end{lyxlist}
B es añadida entre las ciudades del trayecto cuya distancia es menor,
en el ejemplo ilustrado, la opción del tercer trayecto.


\subsection*{2. Vecino más cercano}

Partiendo de alguna ciudad arbitraria se analizan todos los vértices
adyacentes y aún no visitados, y se añade a la solución aquel vértice
cuya arista de costo sea la mínima. Tal análisis prosigue a partir
del último vértice añadido hasta haber visitado todas las ciudades. 

A menudo, el \ac{TSP} sigue un modelo de grafo completo donde cada
par de vértices es conectado por una arista, en tal caso el algoritmo
brinda soluciones óptimas partiendo desde y hacia la ciudad inicial
pasando exactamente una vez por cada ciudad restante. 

Por otro lado,en modelos donde no existe camino entre algunos pares
de ciudades, la elección iterativa de la arista de mínimo costo sin
una vista global del modelo, puede apartar vértices de la solución
retornando soluciones parciales. 

Ejemplo

\includegraphics{\string"C:/Users/usuario/Desktop/Tesis rubbish/tesis/images/TPS-graph-example\string".png}

Solución: A -> B -> C -> D -> A 

Costo: 97


\subsection*{3. Programación Lineal}

El \ac{TSP} puede ser formulado de manera teórica mediante una (o
varias) formulaciones lineales enteras usando programación lineal
en enteros. Se plantean las variables del problema: 

\begin{lstlisting}[basicstyle={\footnotesize},breaklines=true,language=Java,captionpos=t,tabsize=3,frame=no,keywordstyle={\color{blue}},commentstyle={\color{gray}},stringstyle={\color{red}},numbers=left,numberstyle={\tiny},numbersep=5pt,breaklines=true,showstringspaces=false,basicstyle={\footnotesize},emph={label}]
vértices: 0.. n 	
Xij = 1 | Xij = 0 //para i,j = 0...n
La arista (i,j) pertenece(1)/ no pertenece (0) al conjunto solución.
Cij Distancia desde la ciudad i hasta j //para i,j = 0...n	 
aristas número total de aristas. 	
aristasReq número mínimo de aristas para solución. 
\end{lstlisting}



\subsection*{4. Backtracking}

Backtracking es un algoritmo general para encontrar todas o algunas
soluciones para problemas computacionales con notables restricciones
de satisfacción para dicho problema.El método de búsqueda es incremental
de los candidatos a la solución, abandonando aquellos (\emph{backtrackeando})
tan pronto como se determina que dicha solución no es o va a ser válida. 

En el caso del \ac{TSP}, los posibles candidatos a la solución se
determinan mediante el grado de incidencia de todos los nodos que
componen el grafo. Ya que la ciudad sólo debe ser visitada una vez,
el grado de incidencia de cada ciudad debe ser igual a 2, por lo que
aquellas ciudades que no han sido visitadas (grado de incidencia 0)
o aquellas que sólo han sido destino (grado de incidencia 1) son candidatas
a componer la solución.

Pseudo-código: 

\begin{lstlisting}[language=Java,numbers=left,numberstyle={\tiny},basicstyle={\footnotesize},breaklines=true,captionpos=t,frame=no,keywordstyle={\color{blue}},commentstyle={\color{gray}},stringstyle={\color{red}},numbersep=5pt,emph={label}]
backtracking(costoAct, minCost, lastV, currPath, bestPath, costoAct) 
si costoAct<minCost //Poda por costo que se arrastra 	
	si !solucion(currPath) 		
		generarAristasFactibles(lastV) 		
		Por v:AristasFactibles 			
			costoAct+=v.costo; 			
			currPath= currPath + {v}; 			
			backtracking(costoAct, minCost, v.Destine, currPath); 			
			costoAct-=v.costo; 			
			currPath=currPath - {v}; 	
	sino 		
		bestPath=currPath; //Setear la solución ya encontrada 		
		minCost=costoAct;
\end{lstlisting}



\subsection*{TSP simétrico }

El TSP simétrico es una variedad del problema TSP que para todas las
aristas se cumple la desigualdad triangular o de Minkowski: 

\[
(a,c)<(a,b)+(b,c)
\]


Considerando que no todos los grafos creados cumplen con dicha condición,
y que aún así hay que devolver un resultado óptimo, se considera prioridad
la distancia entre los nodos y no si los mismos se repiten o no.


\subsubsection*{4. Árbol de recubrimiento }

Para resolver el \emph{Metric TSP}, se puede utilizar el método de
árbol de recubrimiento mínimo. El mismo se puede obtener mediante
Prim o Kruskal para, luego de eso, nos aseguramos que el grafo será
simétrico mediante la duplicación de cada arista. Gracias a esto,
podemos recorrerlo mediante un algoritmo recursivo y obtenemos el
ciclo euleriano del mismo. Para finalizar la resolución y mediante
la desigualdad antes planteada, se omiten los nodos repetidos, obteniendo
el ciclo hamiltoniano.


\subsubsection*{4.1 Prim}

El algoritmo Prim toma un tomar un vértice arbitrario de los ya visitados
y lo une con el vértice no visitado mediante el menor arco que salga
de él.

Pseudocódigo:

\begin{lstlisting}[language=Java,numbers=left,numberstyle={\tiny},basicstyle={\footnotesize},breaklines=true,captionpos=t,frame=no,keywordstyle={\color{blue}},commentstyle={\color{gray}},stringstyle={\color{red}},numbersep=5pt,emph={label}]
Prim(firstCity){ 	
	visited={firstCity}; 	
	while (visited!=allCities){ 		
		edge=obtenerMinEdge(visited); 		
		sol= sol ∪ {edge}; 		
		visited= visited  ∪ {edge.destino};
	} 	
return sol; }
\end{lstlisting}



\subsubsection*{4.2 Kruskal}

Kruskal mientras tanto tiene ordenados las aristas por costo y va
agregando a la solución aquellos que ya han sido visitados con los
que no.

Pseudocódigo:

\begin{lstlisting}[language=Java,numbers=left,numberstyle={\tiny},basicstyle={\footnotesize},breaklines=true,captionpos=t,frame=no,keywordstyle={\color{blue}},commentstyle={\color{gray}},stringstyle={\color{red}},numbersep=5pt,emph={label}]
Kruskal(firstCity){ 	
	visited={firstCity}; 	
	edges=getSortedEdges(); 	
	∀ e ∈ edges{ 		
		if (e ∈ (visited)x(not_visited)) 			
			sol=sol ∪ {e}; 			
			visited=visited ∪ {e.destino} 		
	} 
	return sol; }
\end{lstlisting}


La única diferencia entre ambos algoritmos de recubrimiento es la
forma de trabajo de los mismos y en consecuencia su complejidad.


\subsubsection*{4.3 Transformaciones Lineales}

Las transformaciones locales se utilizan para mejorar el hamiltoniano
obtenido por los algoritmos antes presentados. Para introducir el
algoritmo, consideremos la siguiente situación:

\includegraphics{\string"C:/Users/usuario/Desktop/Tesis rubbish/tesis/images/TPS-local-transformations\string".png}

Los nodos unidos en el ciclo hamiltoniano son (A,B) ... (C,D) - línea
no punteada en el grafo. 

Ahora bien, para aplicar transformaciones locales se debe verificar: 

\[
(A,C)+(B,D)<(A,B)+(C,D)
\]


De suceder esto, se pueden modificar los arcos y obtener:

\includegraphics{\string"C:/Users/usuario/Desktop/Tesis rubbish/tesis/images/TPS-local-transformations2\string".png}

El resultado es un grafo con menor costo que el anterior presentado.
