
\chapter{Conclusiones\label{chap:Conclusiones}}

En este trabajo se presentó un modelo de aprendizaje automático para
la estimación de atributos no funcionales en componentes de aplicaciones
Android, un framework de predicción con técnicas de regresión que
permite una gran abstraccióna las implementaciones de bajo nivel de
las librerías de aprendizaje de máquina. Este modelo brinda la posibilidad
de crear modelos predictivos óptimos, adecuados a un conjunto de datos
particular y un atributo a predecir particular.

Ciertamente, el uso de modelos como criterio de calidad para la selección
de servicios y componentes se está extendiendo cada vez más, promoviendo
el desarrollo y optimización de técnicas de aprendizaje de máquina.
En primer lugar, en este trabajo se presentaron las características
principales de las técnicas que aplican aprendizaje sobre los datos,
describiendo en forma detallada los algoritmos y el desafío de optimizar
la configuración de los paramétros. Luego, se estudiaron diferentes
herramientas de recolección de datos en sistemas Android examinando
los principales artefactos y mecanismos involucrados en estas herramientas
y se analizaron dos indicadores principales de performance de componentes,
el tiempo de respuesta y precisión en los resultados. 

Como resultado del creciente desarrollo de aplicaciones, hoy en día,
la integración de componentes de terceros se ha popularizado en diversas
áreas, como por ejemplo la de los dispositivos móviles. En este sentido,
se introdujeron las nociones básicas de los dispositivos móviles,
estudiando también, el sistema operativo más difundido en estos dispositivos,
Android. Se analizaron las características más importantes del mismo,
y se presentaron los conceptos necesarios para comprender el funcionamiento
de las aplicaciones en este sistema.

La utilización de componentes de terceros en dispositivos móviles
presenta ciertos desafíos. Estos dispositivos tienen limitaciones
en conflicto como la energía, el acceso a la red y la capacidad de
cálculo que determinan el contexto de ejecución de estos componentes
y que afecta considerablemente los atributos de calidad de los mismos
y de las aplicaciones que los invocan. Por lo tanto, es importante
elegir los componentes adecuados de acuerdo con su calidad de servicio
además de la funcionalidad requerida. En este sentido se describieron
las investigaciones actuales en el área. Se presentaron los trabajos
que apuntaban a la determinación de relaciones entre las estimaciones
de los atributos de calidad y las propiedades inhatas de los diferentes
dominios o casos de aplicación presentados. 

Por último, se presentaron los experimentos efectuados para validar
el enfoque propuesto. Los mismos fueron evaluados con las métricas
presentadas en secciones anteriores, señalando al modelo MultiLayer
Perceptron como el mejor a nivel de predicciones bajas en error. Las
estimaciones permiten predecir que en dominios de problemas P y NP
como el caso del problema de la multiplicacion de matrices y el problema
del viajante métrico, el nivel de correlación alcanzado por el algoritmo
es inmediatamente inferior al 100\% y en cuanto al error en las predicciones
a través de las métrica \ac{RMSE} , tendrán valores inmediatamente
cercanos o iguales a cero. Para el caso de servicios, aunque las estimaciones
arrojadas por el modelo presentan mas variabilidad, su comportamiento
es considerablemente superior al resto de los modelos, en cuanto evita
efectos de overfit y underfit sobre los datos. Esto indica que una
generalización del modelo propuesto puede resultar útil para determinar
a priori qué componentes resultarán más eficiente, dadas ciertas características
en las entradas, en los componentes y en la operación, tanto para
la predicción del tiempo de respuesta como para la presición.

Por otro lado, las evaluaciones señalan al modelo Linear Regression
como la mejor opción a nivel de generalización en el aprendizaje y
buena adaptación a diferentes formatos de dataset. El factor de correlación
obtenido por este modelo para problemas de complejidad polinómica
alcanza resultados tan precisos como el modelo MultiLayer Perceptron
a un costo computacional extremadamente inferior. Respecto al caso
de estudio de problema NP, el modelo fue capaz de alcanzar valores
de correlacioón entre las variables mayores al 70\%, un valor positivo
teniendo en cuenta un conjunto estrecho de datos de entrenamiento
(cercano a las 200 instancias), y un tiempo de procesamiento reducido
a minutos, tanto para la predicción del tiempo de respuesta como para
la precisión. Incluso, los resultados arrojados fueron superiores
a otras técnicas. Respecto a los servicios, se observa que el modelo
tiene dificultades para aprender, aunque la pobreza de los dataset
formados concentra la mayor responsabilidad. En estos ambientes, el
modelo K means Clusterer logra superar estas dificultades y obtiene
resultados superiores incluso, al modelo MultiLayer Perceptron. Para
el servicio Microsoft conformado por más de 1800 instancias, el modelo
alcanza un factor de correlacion un 2\% menor que el factor determinado
por el modelo MultiLayer Perceptron, y para los servicios Google Play
(2320 instancias) y SkyBiometry (3840 instancias) el modelo de cluster
determina aún mejor la relación entre las variables y reduce el error
en las predicciones respecto al modelo MultiLayer, que por sus resultados
fue tomado como base para el análisis de las técnicas en general. 


\section{Limitaciones}

La principal limitación que presenta el modelo propuesto en este trabajo
parte del problema de la falta de recursos suficientes de los dispositivos
móviles para llevar adelante ejecuciones complejas y costosas, que
insumen gran cantidad de memoria y procesamiento. Este factor, impide
la posibilidad de serializar las primeras dos fases del enfoque propuesto,
obligando a dividir el enfoque en dos herramientas desarrolladas para
ambientes distintos. 

Otra limitación que se presenta responde a la ausencia de un método
automatizado para configurar dinámicamente los parámetros de las tecnicas
de regresión, debiendo determinar un rango estático y específico propio
para cada técnica, para llevar a cabo el proceso de optimización,
considerando de esta forma a todos los datos de entrenamiento por
igual, sin distinción de formato, dominio, etc.

Finalmente, es importante destacar que las estimaciones arrojadas
para los servicios remotos de detección facial, no reflejan conclusiones
claras y determinísticas. Este problema parte de la heterogeneidad
en la formación de los datos, lo que dificulta realizar comparaciones
entre ellos. Por otro lado, no se incoporaron atributos que reflejen
datos reales sobre los rostros en las imágenes ni atributos sobre
los resultados retornados por el servicio, de manera que la precisión
no puede ser analizada, ni contrastada con el tiempo de respuesta
obtenido.


\section{Trabajos Futuros}

Cabe mencionar que existen algunas posibilidades de extensión interesantes
para este trabajo. Como primer objetivo se pretende mejorar el tiempo
de procesamiento requerido para la optimización y generación de los
modelos predictivos. Para poder lograr esto, se le puede otorgar a
la aplicación la capacidad de soportar eficientemente múltiples hilos
de ejecución y distribuir o paralelizar los procesos de optimización
y construcción de los diferentes modelos. Por otro lado, se deben
incoportar dataset con la suficiente expresividad para describir de
mejor manera los diferentes casos de aplicación analizados y derivar
en conclusiones más precisas.

Adicionalmente se pretende incorporar nuevos aspectos para la mejora
y contraste de técnicas. Entre los aspectos a considerar al extender
el proceso de optimización se procura: 
\begin{itemize}
\item Independizar al módulo de optimización de técnicas, de diferentes
rangos de configuración para cada uno de los parámetros que incluya
la técnica. De esta forma, lograr más énfasis en los valores de cada
parámetro y explotar al máximo el desempeño de las técnicas. Incluso,
el usuario podría disponer de la posibilidad de personalizar estos
valores.
\item Informar y registrar el tiempo real consumido por cada operación de
construcción de un modelo. Operación que incluye la configuración,
optimización, entrenamiento y finalmente la construccion del modelo
con la evaluación de las métricas correspondientes. 
\item Extender el análisis comparativo a nivel de librerías, aprovechando
la capacidad de la herramienta de integrar librerías de terceros.
De esta manera, se logra comparar el desempeño de una técnica conocida
en diferentes implementaciones, y seleccionar la mejor técnica independiente
de la librería que la implemente. 
\item Considerar conjuntos de datos de la misma naturaleza que los usados
durante la fase de entrenamiento, para evaluar los modelos creados.
Brindar un mecanismo de evaluación de los modelos a partir de las
métricas, y validar estos modelos creados frente a nuevos conjuntos
de datos para enclarecer el desempeño y generalización lograda. 
\end{itemize}
\selectlanguage{english}%
\selectlanguage{english}%

