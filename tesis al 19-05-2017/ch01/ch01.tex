
\chapter{Introducción}


\section{Motivación\label{sec:Motivaci=0000F3n}}

En los últimos años las aplicaciones móviles se han puesto en el centro
de la escena debido a la proliferación de los dispositivos móviles
y su creciente capacidad de cómputo y almacenamiento\cite{Zunino2011}.
Estos dispositivos pasaron de ser terminales con capacidades limitadas,
generalmente de propósito específico, como agendas electrónicas o
teléfonos celulares, a ser pequeñas computadoras de propósito general
con grandes capacidades de procesamiento, almacenamiento y acceso
a Internet, como tablets y smartphones. 

Los dispositivos móviles de hoy en día utilizan sistemas operativos
similares a las computadoras personales, como Android y iOS. Por lo
tanto, pueden ejecutar software al que hace unos años atrás solo se
encontraba en dichas computadoras. Esto, sumado a su costo accesible,
pequeño tamaño, movilidad y ubicuidad de las conexiones móviles de
alta velocidad, ha impulsado el desarrollo de las aplicaciones móviles
para una amplia variedad de fines, incluyendo entretenimiento, juegos,
comunicaciones, redes sociales, comercio electrónico, turismo, educación,
y mucho más. 

Para reducir los costos y tiempos de desarrollo, es común la re-utilización
de software mediante la integración de componentes de terceros. Un
componente \citep{Petritsch2016} es una entidad de software en tiempo
de ejecución que encapsula un servicio, es decir, un conjunto de funciones
y datos, a través de una interfaz específica \ac{API}. Los componentes
se pueden clasificar de acuerdo al ambiente en el que residen durante
su ejecución, distinguiéndose así aquellos que se ejecutan en nodos
remotos, como los denominados Servicios Web \cite{Erickson2009},
de aquellos que residen en el dispositivo, como procesos en segundo
plano, bibliotecas de enlace dinámico, o simples objetos Java. 

Las funcionalidades que implementan estos componentes buscan satisfacer
las necesidades comunes de muchas aplicaciones, como el procesamiento
de texto e imágenes, almacenamiento de datos en la nube, identificación
de usuarios, algoritmos de optimización, etc. La misma funcionalidad
suele ser ofrecida por componentes alternativos que difieren en sus
propiedades no-funcionales o atributos de calidad \cite{Addison2003}.
Estas propiedades son los aspectos que utilizan los desarrolladores
para juzgar su funcionamiento, tales como performance (por ej., tiempo
de respuesta), disponibilidad (por ej., tasa de errores) o precisión
de la respuesta. 

Los dispositivos móviles tienen limitaciones en conflicto como la
energía, el acceso a la red y la capacidad de cálculo que determinan
el contexto de ejecución de estos componentes y que afecta los atributos
de calidad de los mismos y de las aplicaciones que los invocan. Por
lo tanto, es importante elegir los componentes adecuados de acuerdo
con su calidad de servicio además de la funcionalidad requerida. 

Sin embargo, estos componentes suelen ser \emph{cajas negras} para
los desarrolladores de aplicaciones móviles, que tienen acceso a la
definición de sus \ac{API}s pero no así a su implementación interna,
por lo que no cuentan con información de sus atributos dinámicos para
elegir el componente adecuado en cada contexto de ejecución.

Teniendo en cuenta lo antes dicho, en este trabajo se propone estudiar
y aplicar técnicas de aprendizaje de máquina \cite{Mitchell2013}
 para construir modelos de predicción de las propiedades dinámicas
de componentes de software en dispositivos móviles. 

En las aplicaciones móviles, donde la disponibilidad de recursos puede
variar rápidamente (cambio de red de acceso a Internet, capacidad
limitada de memoria y \ac{CPU}, etc) y teniendo en cuenta la forma
en que los componentes consumen estos recursos, es interesante el
uso de estos modelos como criterio de calidad para la selección de
servicios y componentes candidatos, tanto en tiempo de desarrollo
como en tiempo de ejecución.


\section{Objetivos y Solución propuesta\label{sec:Objetivos-Soluci=0000F3n-propuesta}}

El objetivo del trabajo consiste en desarrollar un enfoque para la
elaboración de modelos de predicción de propiedades dinámicas, como
tiempo de respuesta y precisión, de componentes ejecutados sobre dispositivos
móviles. Basándonos en el hecho de que Android es el sistema operativo
más difundido para dispositivos móviles, el enfoque se pondrá a prueba
sobre diferentes casos de estudio en este sistema operativo. 

El enfoque se basa en un proceso de aprendizaje de máquina. Dado un
conjunto de componentes, como algoritmos o servicios Web, que implementan
la misma funcionalidad y de los cuales queremos predecir alguna propiedad
dinámica, el método propuesto es el siguiente: 
\begin{enumerate}
\item Usar conocimiento del caso de uso para seleccionar características
del contexto y de los datos de entrada del componente que puedan ser
indicativos de esta propiedad. 
\item Generar un conjunto de datos de entrada representativos del espacio
de entrada para la evaluación de los componentes. 
\item Ejecutar los componentes con las entradas generadas y tomar mediciones
de las características identificadas en el punto 1 más la propiedad
de interés a predecir: tiempo de respuesta, precisión de la respuesta,
etc. 
\item Usar estas mediciones con técnicas de aprendizaje de máquina para
entrenar y evaluar modelos de predicción de la propiedad. 
\end{enumerate}
El enfoque propuesto se pondrá a prueba sobre grupos de algoritmos
y servicios Web reales que implementan funcionalidades de interés
para desarrolladores de aplicaciones móviles. Esta evaluación no sólo
involucrará diferentes casos de estudio, sino también diferentes técnicas,
como regresiones y redes neuronales, sobre diferentes propiedades
de interés. Para llevar a cabo la medición de los componentes se implementó
un framework que simplifica la ejecución de pruebas y mediciones de
performance en la plataforma Android, y para el entrenamiento y evaluación
de modelos se implementó una segunda herramienta que utiliza software
de aprendizaje de máquina como Weka\cite{Hall2009}.


\section{Organización\label{sec:Organizaci=0000F3n}}

El resto del trabajo se organiza en 5 capítulos. A continuación se
da un breve resumen de los temas que se abordan en cada uno de ellos. 

En el capítulo\ref{chap:Marco-Teorico} se presenta el marco teórico,
donde se definen los conceptos utilizados a lo largo de todo el informe,
tales como: Android, desempeño, precisión, aprendizaje de máquina,
regresión, modelos, componentes. 

En el capítulo\ref{chap:Trabajos-Relacionados} se presentan algunos
trabajos relacionados desde diferentes perspectivas: herramientas
de medición de performance para el sistema Android, y trabajos que
involucran predicción de performance con técnicas de aprendizaje de
máquina.

En el capítulo \ref{chap:Enfoque-y-Herramientas} se describe el enfoque
y las herramientas propuestas. Se detalla la arquitectura e implementación
de las mismas, ahondando en las decisiones de diseño que se consideran
mas importantes. 

En el capítulo \ref{chap:Evaluaci=0000F3n} se presenta la evaluación
del enfoque sobre tres casos de estudio. Se presentan las propiedades
consideradas en los escenarios evaluados, los modelos que han sido
generados y analizados, y los resultados alcanzados.

Finalmente, en el capítulo \ref{chap:Conclusiones} se exponen las
conclusiones del trabajo realizado, las limitaciones encontradas del
enfoque y las herramientas, y posibles líneas de trabajo futuro.
